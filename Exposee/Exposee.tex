%%% File-Information {{{
%%% Filename: template_bericht.tex
%%% Purpose: lab report, technical report, project report
%%% Time-stamp: <2004-06-30 18:19:32 mp>
%%% Authors: The LaTeX@TUG-Team [http://latex.tugraz.at/]:
%%%          Karl Voit (vk), Michael Prokop (mp), Stefan Sollerer (ss)
%%% History:
%%%   20050914 (ss) correction of "backref=true" to "backref" due to hyperref documentation
%%%   20040630 (mp) added comments to foldmethod at end of file
%%%   20040625 (vk,ss) initial version
%%%
%%% }}}
%%%%%%%%%%%%%%%%%%%%%%%%%%%%%%%%%%%%%%%%%%%%%%%%%%%%%%%%%%%%%%%%%%%%%%%%%%%%%%%%
%%% main document {{{

\documentclass[
a4paper,     %% defines the paper size: a4paper (default), a5paper, letterpaper, ...
% landscape,   %% sets the orientation to landscape
% twoside,     %% changes to a two-page-layout (alternatively: oneside)
% twocolumn,   %% changes to a two-column-layout
% headsepline, %% add a horizontal line below the column title
% footsepline, %% add a horizontal line above the page footer
% titlepage,   %% only the titlepage (using titlepage-environment) appears on the first page (alternatively: notitlepage)
% parskip,     %% insert an empty line between two paragraphs (alternatively: halfparskip, ...)
% leqno,       %% equation numbers left (instead of right)
% fleqn,       %% equation left-justified (instead of centered)
% tablecaptionabove, %% captions of tables are above the tables (alternatively: tablecaptionbelow)
% draft,       %% produce only a draft version (mark lines that need manual edition and don't show graphics)
% 10pt         %% set default font size to 10 point
% 11pt         %% set default font size to 11 point
12pt         %% set default font size to 12 point
]{scrartcl}  %% article, see KOMA documentation (scrguide.dvi)



%%%%%%%%%%%%%%%%%%%%%%%%%%%%%%%%%%%%%%%%%%%%%%%%%%%%%%%%%%%%%%%%%%%%%%%%%%%%%%%%
%%%
%%% packages
%%%

%%%
%%% encoding and language set
%%%

%%% ngerman: language set to new-german
\usepackage{ngerman}

%%% babel: language set (can cause some conflicts with package ngerman)
%%%        use it only for multi-language documents or non-german ones
%\usepackage[ngerman]{babel}

%%% inputenc: coding of german special characters
\usepackage[utf8]{inputenc}

%%% fontenc, ae, aecompl: coding of characters in PDF documents
\usepackage[T1]{fontenc}
\usepackage{ae,aecompl}

%%%
%%% technical packages
%%%

%%% amsmath, amssymb, amstext: support for mathematics
%\usepackage{amsmath,amssymb,amstext}

%%% psfrag: replace PostScript fonts
\usepackage{psfrag}

%%% listings: include programming code
%\usepackage{listings}

%%% units: technical units
%\usepackage{units}

%%%
%%% layout
%%%

%%% scrpage2: KOMA heading and footer
%%% Note: if you don't use this package, please remove 
%%%       \pagestyle{scrheadings} and corresponding settings
%%%       below too.
\usepackage[automark]{scrpage2}


%%%
%%% PDF
%%%

\usepackage{ifpdf}

%%% Should be LAST usepackage-call!
%%% For docu on that, see reference on package ``hyperref''
\ifpdf%   (definitions for using pdflatex instead of latex)

  %%% graphicx: support for graphics
  \usepackage[pdftex]{graphicx}

  \pdfcompresslevel=9

  %%% hyperref (hyperlinks in PDF): for more options or more detailed
  %%%          explanations, see the documentation of the hyperref-package
  \usepackage[%
    %%% general options
    pdftex=true,      %% sets up hyperref for use with the pdftex program
    %plainpages=false, %% set it to false, if pdflatex complains: ``destination with same identifier already exists''
    %
    %%% extension options
    backref,      %% adds a backlink text to the end of each item in the bibliography
    pagebackref=false, %% if true, creates backward references as a list of page numbers in the bibliography
    colorlinks=false,   %% turn on colored links (true is better for on-screen reading, false is better for printout versions)
    %
    %%% PDF-specific display options
    bookmarks=true,          %% if true, generate PDF bookmarks (requires two passes of pdflatex)
    bookmarksopen=false,     %% if true, show all PDF bookmarks expanded
    bookmarksnumbered=false, %% if true, add the section numbers to the bookmarks
    %pdfstartpage={1},        %% determines, on which page the PDF file is opened
    pdfpagemode=None         %% None, UseOutlines (=show bookmarks), UseThumbs (show thumbnails), FullScreen
  ]{hyperref}


  %%% provide all graphics (also) in this format, so you don't have
  %%% to add the file extensions to the \includegraphics-command
  %%% and/or you don't have to distinguish between generating
  %%% dvi/ps (through latex) and pdf (through pdflatex)
  \DeclareGraphicsExtensions{.pdf}

\else %else   (definitions for using latex instead of pdflatex)

  \usepackage[dvips]{graphicx}

  \DeclareGraphicsExtensions{.eps}

  \usepackage[%
    dvips,           %% sets up hyperref for use with the dvips driver
    colorlinks=false %% better for printout version; almost every hyperref-extension is eliminated by using dvips
  ]{hyperref}

\fi


%%% sets the PDF-Information options
%%% (see fields in Acrobat Reader: ``File -> Document properties -> Summary'')
%%% Note: this method is better than as options of the hyperref-package (options are expanded correctly)
\usepackage{color}
\definecolor{Navy}{rgb}{0,0,0.5}
\definecolor{Gray}{gray}{0.5}
\definecolor{dunkelgrau}{rgb}{0.8,0.8,0.8}
\hypersetup{
  pdftitle={}, %%
  pdfauthor={}, %%
  pdfsubject={}, %%
  pdfcreator={Accomplished with LaTeX2e and pdfLaTeX with hyperref-package.}, %% 
  pdfproducer={}, %%
  pdfkeywords={} %%
  colorlinks=true, 			% false: boxed links; true: colored links
	linkcolor=Navy,          		% color of internal links
	citecolor=Gray,        			% color of links to bibliography
	filecolor=magenta,      		% color of file links
	urlcolor=blue,           			% color of external links
	linkbordercolor={1 1 1}, 		% set to white
	citebordercolor={1 1 1} 		% set to white
}


%%%%%%%%%%%%%%%%%%%%%%%%%%%%%%%%%%%%%%%%%%%%%%%%%%%%%%%%%%%%%%%%%%%%%%%%%%%%%%%%
%%%
%%% user defined commands
%%%

%%% \mygraphics{}{}{}
%% usage:   \mygraphics{width}{filename_without_extension}{caption}
%% example: \mygraphics{0.7\textwidth}{rolling_grandma}{This is my grandmother on inlinescates}
%% requires: package graphicx
%% provides: including centered pictures/graphics with a boldfaced caption below
%% 
\newcommand{\mygraphics}[3]{
  \begin{center}
    \includegraphics[width=#1, keepaspectratio=true]{#2} \\
    \textbf{#3}
  \end{center}
}

%%%%%%%%%%%%%%%%%%%%%%%%%%%%%%%%%%%%%%%%%%%%%%%%%%%%%%%%%%%%%%%%%%%%%%%%%%%%%%%%
%%%
%%% define the titlepage
%%%

% \subject{}   %% subject which appears above titlehead
\titlehead{\includegraphics[width=0.2\textwidth]{logo.png}} %% special heading for the titlepage

%%% title
\title{Exposee zur Masterarbeit}
\subtitle{Performante Speicherung und Verarbeitung von Geodaten}

%%% author(s)
\author{Kurt Junghanns (59886)}

%%% date
\date{Leipzig, der \today{}}

\publishers{Betreuer: M. Sc. Volkmar Herbst}

% \thanks{} %% use it instead of footnotes (only on titlepage)

% \dedication{} %% generates a dedication-page after titlepage


%%% uncomment following lines, if you want to:
%%% reuse the maketitle-entries for hyperref-setup
%\newcommand\org@maketitle{}
%\let\org@maketitle\maketitle
%\def\maketitle{%
%  \hypersetup{
%    pdftitle={\@title},
%    pdfauthor={\@author}
%    pdfsubject={\@subject}
%  }%
%  \org@maketitle
%}


%%%%%%%%%%%%%%%%%%%%%%%%%%%%%%%%%%%%%%%%%%%%%%%%%%%%%%%%%%%%%%%%%%%%%%%%%%%%%%%%
%%%
%%% set heading and footer
%%%

%%% scrheadings default: 
%%%      footer - middle: page number
\pagestyle{scrheadings}

%%% user specific
%%% usage:
%%% \position[heading/footer for the titlepage]{heading/footer for the rest of the document}

%%% heading - left
% \ihead[]{}

%%% heading - center
% \chead[]{}

%%% heading - right
% \ohead[]{}

%%% footer - left
% \ifoot[]{}

%%% footer - center
% \cfoot[]{}

%%% footer - right
% \ofoot[]{}



%%%%%%%%%%%%%%%%%%%%%%%%%%%%%%%%%%%%%%%%%%%%%%%%%%%%%%%%%%%%%%%%%%%%%%%%%%%%%%%%
%%%
%%% begin document
%%%

\begin{document}

% \pagenumbering{roman} %% small roman page numbers

%%% include the title
% \thispagestyle{empty}  %% no header/footer (only) on this page
 \maketitle

%%% start a new page and display the table of contents
% \newpage
 \tableofcontents

%%% start a new page and display the list of figures
% \newpage
% \listoffigures

%%% start a new page and display the list of tables
% \newpage
% \listoftables

%%% display the main document on a new page 
% \newpage

% \pagenumbering{arabic} %% normal page numbers (include it, if roman was used above)

%%%%%%%%%%%%%%%%%%%%%%%%%%%%%%%%%%%%%%%%%%%%%%%%%%%%%%%%%%%%%%%%%%%%%%%%%%%%%%%%
%%%
%%% begin main document
%%% structure: \section \subsection \subsubsection \paragraph \subparagraph
%%%


\section{Motivation}

Die Agri Con GmbH verwaltet als Akteur im Bereich \glqq Precision Farming\grqq \ täglich mehrere Millionen geografische Punktdaten.
Diese Daten werden von aktiven Landwirtschaftsmaschinen und durch die Verarbeitung durch firmeninterne und firmenexterene Mitarbeiter sowie Systeme erzeugt.
Weiterhin fallen dadurch Vektor- und Rasterdaten an, welche gespeichert und anschließend verarbeitet werden müssen.

- Daten sind essentiell für den Betrieb
- Notwendigkeit von Speicherung und schneller mächtiger Verarbeitung
- Nicht nur Agri Con steht vor dieser Notwendigkeit
- normalerweise Oracle DB oder PostGIS, keine Alternativen bekannt
- da viele Daten und verteilt -> NoSQL interessant



\section{Fragestellung}

Täglich anfallende geographische Daten sollen in einer zu definierenden Umgebung im Gigabytebereich persistent gespeichert werden.
Außerdem sollen diese Daten mit geringer Laufzeit aggregiert und verarbeitet werden.
Die Verarbeitung der Daten soll anhand der geographischen Informationen mit Hilfe dafür vorgesehener Funktionen der Umgebung erfolgen.
Das Ergebnis der Verarbeitung soll stets eine Menge von aufbereiteten Geodaten sein, welche von GIS direkt interpretiert werden können.

Die Daten fallen dynamisch im Tagesbetrieb an.
Dabei handelt sich um Raster- sowie Vektordaten.
Es ist davon auszugehen, dass innerhalb eines Monats schätzungsweise 30GB an neuen Daten anfallen.
Dabei sind bereits 200 GB an Daten vorhanden.

Die Umgebung muss somit große Mengen an Daten raumbezogen ablegen können.

An die Datenhaltung werden folgende Anforderungen gestellt:
\begin{itemize}
\item gleichzeitige und verteilte Datenablage auf mehreren virtuellen Maschinen
\item Erhalt der geographischen Informationen der Daten
\item persistente Datenhaltung von statischen und dynamischen Daten
\item Versionierung der dynamischen Daten
\item performante Aggregation soll bereits durch intelligente Datenablage ermöglicht werden
\end{itemize}

Gespeicherte Daten sind speziell gefiltert zu aggregieren.
Dabei soll die Filterung anhand von Meta- und Geoinformationen erfolgen.
Neben der Datenhaltung soll auch die Verarbeitung verteilt durchführbar sein.
Die Laufzeit ist eine essentielle Anforderung, wie auch der Umfang der geographischen Funktionen, wobei die Qualität der Funktionen die Laufzeit wesentlich beeinflusst.



\section{Stand der Forschung}

In der Agri Con GmbH werden täglich angefallene Daten nachts verarbeitet und stehen nur in verarbeiteter Form bereit.
Zur Datenhaltung und -verarbeitung wird PostGreSQL mit der Erweiterung PostGIS verwendet. Auch R wird als Programmiersprache zur Verarbeitung eingesetzt.

Denkbare Systeme sind folgende:
\begin{itemize}
\item PostGIS mit Optimierung durch Reorganisation der Datenbank und verteilte Verwendung\footnote{Dazu zählt Multi-Master Replikation, Load-Balancing und clustering}
\item MongoDB
\item GeoCouch
\item Oracle
\item Hadoop, Hive und ArcGis
\item Hadoop mit einer der Datenbanken
\end{itemize}


Momentan existieren komplette Systeme oder Teilsysteme für das beschriebene Szenario.
Eine Lösung ist clustering von PostGIS oder Oracles Geodatenbank.
Eine andere ist Hadoop-GIS, ein komplettes System welches Hadoop verwendet und eine eigene Engine zur Verarbeitung bereitstellt, oder SpatialHadoop, welches einen Hadoop Aufsatz zur Verarbeitung darstellt.

\section{Methodik und Vorgehen}

Für dieses Szenario sind die technischen Möglichkeiten zu analysieren und zu vergleichen.
Darin sollen besonders die Möglichkeiten der geographischen Speicherung und Verarbeitung der Daten und deren Beziehungen betrachtet werden.
Der Vergleich soll anschließend in einer Empfehlung für das dargestellte Szenario münden.
Damit die Aussagefähigkeit des Vergleiches und der Empfehlung für die Umsetzung in der Agri Con GmbH hoch ist, sind Studien in Form von empirischen Leistungsvergleichen der Laufzeit und des Funktionsumfanges durchzuführen.

Die in der Umgebung vorhandenen geographischen Funktionen sind zu nutzen und fehlende zu implementieren oder durch Erweiterungen verfügbar zu machen.


\section{Zeitplan}

Die Thesis wird zum ersten Oktober 2014 angemeldet und ist spätestens am 31 März abzugeben.





\section{vorläufige Gliederung}

1. Einleitung\\
1.1 Motivation\\
1.2 Zielsetzung\\
2. Grundlagen\\
2.1 geografische Datenverarbeitung\\
2.1.1 Bezugssysteme\\
2.1.2 Geometriearten\\
2.1.2 PostGIS\\
2.2 NoSQL\\
2.2.1 Abgrenzung zu relationalen Systemen\\
2.2.2 NoSQL GIS\\
2.3 Leistungstests\\
3. Ausgangssituation\\
4. System 1\\
4.1 Überblick\\
4.2 Datenhaltung\\
4.3 Verarbeitung\\
4.4 Testumgebung\\
4.5 Zusammenfassung\\
5. System 2\\
6. System 3\\
7. Vergleich\\
8. Schlussfolgerung\\



\section{Literatur}
\begin{itemize}
\item \url{http://blogs.esri.com/esri/arcgis/2012/05/02/mongodb-example-code-for-adding-a-nosql-plug-in-data-source/}
\item \url{http://www.paolocorti.net/2009/12/06/using-mongodb-to-store-geographic-data/}
\item \url{http://www-users.cs.umn.edu/~shekhar/talk/2012/12.6.sbd.duke.pdf}
\item \url{http://esri.github.io/gis-tools-for-hadoop/}
\item \url{http://www.vldb.org/pvldb/vol6/p1009-aji.pdf}
\item \url{https://www.youtube.com/watch?v=_JCPf89s-NI} plus \url{http://de.slideshare.net/Hadoop_Summit/grailer-hochmuth-june27515pmroom212v3}
\item \url{http://video.arcgis.com/watch/2394/big-data-using-arcgis-with-apache-hadoop}
\item \url{http://spatialhadoop.cs.umn.edu/}
\item \url{http://backstopmedia.booktype.pro/big-data-dictionary/preface/}
\item \url{http://www.rasdaman.de/}
\item \url{http://en.wikipedia.org/wiki/SciDB}

\end{itemize}

%%%
%%% end main document
%%%
%%%%%%%%%%%%%%%%%%%%%%%%%%%%%%%%%%%%%%%%%%%%%%%%%%%%%%%%%%%%%%%%%%%%%%%%%%%%%%%%

% \appendix  %% include it, if something (bibliography, index, ...) follows below

%%%%%%%%%%%%%%%%%%%%%%%%%%%%%%%%%%%%%%%%%%%%%%%%%%%%%%%%%%%%%%%%%%%%%%%%%%%%%%%%
%%%
%%% bibliography
%%%
%%% available styles: abbrv, acm, alpha, apalike, ieeetr, plain, siam, unsrt
%%%
% \bibliographystyle{plain}

%%% name of the bibliography file without .bib
%%% e.g.: literatur.bib -> \bibliography{literatur}
% \bibliography{FIXXME}

\end{document}
%%% }}}
%%% END OF FILE
%%%%%%%%%%%%%%%%%%%%%%%%%%%%%%%%%%%%%%%%%%%%%%%%%%%%%%%%%%%%%%%%%%%%%%%%%%%%%%%%
%%% Notice!
%%% This file uses the outline-mode of emacs and the foldmethod of Vim.
%%% Press 'zi' to unfold the file in Vim.
%%% See ':help folding' for more information.
%%%%%%%%%%%%%%%%%%%%%%%%%%%%%%%%%%%%%%%%%%%%%%%%%%%%%%%%%%%%%%%%%%%%%%%%%%%%%%%%
%% Local Variables:
%% mode: outline-minor
%% OPToutline-regexp: "%% .*"
%% OPTeval: (hide-body)
%% emerge-set-combine-versions-template: "%a\n%b\n"
%% End:
%% vim:foldmethod=marker
