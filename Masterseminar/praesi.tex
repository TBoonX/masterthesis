% Dieser Text ist urheberrechtlich geschützt
% Er stellt einen Auszug eines von mir erstellten Referates dar
% und darf nicht gewerblich genutzt werden
% die private bzw. Studiums bezogen Nutzung ist frei
% Januar 2006
% Autor: Sascha Frank 
% Universität Freiburg 
% www.informatik.uni-freiburg.de/~frank/
% www.namsu.de/


\documentclass{beamer}
%%\usetheme{Warsaw-seahorse}
%\usepackage{german}
%\usepackage{ngerman}
\usepackage[utf8]{inputenc}
\usepackage[ngerman]{babel}
\usepackage{hyperref}

\usetheme{Warsaw}
%\usecolortheme{seahorse}
%\usefonttheme{serif}
\useinnertheme{rectangles}
%\usepackage{bookman}
\setbeamercovered{transparent}

%\setbeamertemplate{navigation symbols}{}
%\setbeamertemplate{footline}{}
%\setbeamertemplate{headline}{}


\usepackage{color} % used for comments
\usepackage{listings}%Quellcode

\definecolor{Navy}{rgb}{0,0,0.5}
\definecolor{Gray}{gray}{0.5}
\definecolor{dunkelgrau}{rgb}{0.8,0.8,0.8}
\definecolor{hellgrau}{rgb}{0.95,0.95,0.95}
\definecolor{hellgrau2}{rgb}{0.93,0.93,0.93}
\definecolor{red}{rgb}{1,0, 0.2}
\definecolor{green}{rgb}{0,1,0.2}

\lstset {
  language=xml,
  basicstyle={\footnotesize\ttfamily},
  numbers=none,
  aboveskip=5mm,
  belowskip=5mm,
  showstringspaces=false,
  columns=flexible,
  keywordstyle={\bfseries\color{green}},
  commentstyle={\color{Red}\textit},
  stringstyle={\color{red}},
  frame=single,
  breaklines=true,
  breakatwhitespace=true,
  tabsize=4,
  morekeywords={xmlns:rdf, xmlns:rdfs, rdf:about, rdf:resource}  % <-- adding custom keywords
}
\lstset {
  language=sparql,
  basicstyle={\footnotesize\ttfamily},
  numbers=left,
  aboveskip=5mm,
  belowskip=5mm,
  showstringspaces=false,
  columns=flexible,
  keywordstyle={\bfseries\color{green}},
  commentstyle={\color{hellgrau}\textit},
  stringstyle={\color{red}},
  frame=single,
  breaklines=true,
  breakatwhitespace=true,
  tabsize=4,
  morekeywords={PREFIX, select, SELECT, where, WHERE, Filter, filter, FILTER}  % <-- adding custom keywords
}


\begin{document}

\title{Masterseminar}
\subtitle{Untersuchung und Optimierung geografischer Informationssysteme zur Verarbeitung agrartechnischer Kennzahlen} 
\author{Kurt Junghanns, B.Sc.} 
\date{\today}
%\logo{\includegraphics[scale=0.08]{logo-SF}}

\begin{frame}
\titlepage
\end{frame}

\begin{frame}
\frametitle{Inhaltsverzeichnis}\tableofcontents
%Themenfindung
%Datenquellen
%Schema (Verlinkung nennen)
%ETL (Integration; Datentransformation wichtig)
%Ergbnis (ohne business intelligence)
%    linguistisches matching möglich (Levenshtein-distanz oder N-Gram Ähnlihckeit)
\end{frame}

\section{Einleitung}
\begin{frame}\frametitle{Einleitung}
\underline{Betreuer:}\\
M. Sc. Volkmar Herbst\\
Prof. Dr. rer. nat. Thomas Riechert\\
\vspace{\baselineskip}
\underline{Unternehmen:}\\
Agricon GmbH\\
\vspace{\baselineskip}
\underline{Abgabedatum:}\\
28.3.2015
\end{frame}

\section{Aufgabenstellung}
\begin{frame}\frametitle{Aufgabenstellung}
\textit{Untersuchung und Optimierung geografischer Informationssysteme zur Verarbeitung agrartechnischer Kennzahlen:}\\

\begin{enumerate}
\item Untersuchung bestehender Systeme anhand von Qualitätsmerkmalen
\item Auswahl eines Frameworks
\item Architekturentwurf
\item prototypische Implementierung %plus Bewertung
\end{enumerate}
\end{frame}

\section{Einordnung}
\begin{frame}\frametitle{Einordnung} 
\underline{Interesse:}\\
\begin{itemize}
\item OpenSource Alternativen unbekannt
\item NoSQL Eignung
\item Verringerung der Laufzeit
\end{itemize}

\underline{Anwendung:}\\
\begin{itemize}
\item Entlastung der Datenbank
\item Persistierung der originalen Daten
\item Verringerung der Laufzeit
\end{itemize}
\end{frame}

\section{Grundlagen}
\begin{frame}\frametitle{gegebene Grundlagen zur Lösung} 
\begin{itemize}
\item Referenzsystem %Ist Stand
\item Testdaten
\item Anforderungen
\item Ausgabemodul (UMN MapServer)
\end{itemize}
\end{frame}

\begin{frame}\frametitle{theoretische Grundlagen} 
\begin{itemize}
\item Softwarequalität\footnote{[Wall2001]}
\item Softwaremetriken\footnote{[Fent1997]}
\item Funktionstests\footnote{[Ludw2007]}
\item Leistungstests\footnote{[Hans1995]}
\item Nutzwertanalyse
%nicht Capability Maturity Model (Reifegradmodell) (kurz CMM), da es um Entwicklung/Verbesserung von Software geht
%nicht Control Objectives for Information and Related Technology (COBIT), da Unternehmen nicht in die Analyse einfließen soll
%Analytic Hierarchy Process (AHP) [auch nicht geeignet, da keine Teams und Zeitmanagement] zu detailiert, da maximal 1 System die notwendigen Anforderungen erfüllt und somit keine detailierte Abschätzung/Bewertung der systeme gegeben werden muss
%
\item Guidelines der Systeme
\end{itemize}
\end{frame}

%\section{Eigentanteil}
% Vorhandene Systeme in diesem Anwendungsszenario vergleichen
% erweiterung der Syteme durch eigenen Code und andere Systeme
% 

\section{Anforderungen}
\begin{frame}\frametitle{Anforderungen}
Die prototypische Umsetzung ist Wunschkriterium.\\ %da Eignung der Systeme zu Beginn nicht bekannt
Ziel ist bei Eignung von Systemen das mit der bestmöglichen Eignung zum Einsatz in der Firma darzustellen.
\end{frame}

\begin{frame}\frametitle{Qualitätskriterien}
\begin{itemize}
\item Funktionsumfang: parallele Verarbeitung, Gruppierungs-, Filter-, Verschneidungs- sowie Overlayfunktionen und Geostatistik
\item Interoperabilität: Schnittstellen für PostgreSQL und UMN MapServer
\item Fehlertoleranz: Unabhängigkeit der Verarbeitungsprozesse
\item Dokumentation: seriöse Dokumentation der Installation, Verwendung und Wartung
\item Zeitverhalten: Laufzeiten unter denen des Ist-Standes\footnote{Anhängig von Art der Berechnung und Menge der Daten}
\end{itemize}
\end{frame}

\section{Lösungsansatz}
%modellieren der Aufgabe, den Kontext und die Ergebnisse


\begin{frame}\frametitle{Ansätze zur Lösung}
\underline{Softwareauswahl:}\\
Allgemeines Vorgehen bei Bewertung von Software mit Hilfe von Qualitätsmerkmalen und -kriterien.\\

\underline{Bewertung:}\\
Softwaremetriken mit Leistungs- und Funktionstests.
\end{frame}
%aus Studium:
% Messung des Speedup aus MessagePassing
% 

\section{Stand}
\begin{frame}\frametitle{erreichte Ergebnisse}
%Gantt
\begin{figure}[htb]
  \begin{center}
    \includegraphics[width=1\hsize]{Plan.png}
  \end{center}
  %\caption[a]{}
\end{figure}
\end{frame}


\section{Ausblick}
\begin{frame}\frametitle{Ausblick}
\begin{itemize}
\item Softwaremetriken spezifizieren
\item Systeme auswählen
\item ausgewählte Systeme mit Metriken bewerten
\item Prototyp entwerfen
\item Werkzeugauswahl
\item Prototyp bewerten
\end{itemize}
\end{frame}

\end{document}
