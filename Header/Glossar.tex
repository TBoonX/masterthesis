%Befehle für Glossar
\newglossaryentry{bonitur}
{
  name=Bonitur,
  description={landwirtschaftliche Beurteilung des Ackers und der Pflanzen zum Zwecke der Planung des Einsatzes von Dünger, Pestiziden, Fungiziden und Herbiziden.},
  plural=Bonituren
}
\newglossaryentry{umn}
{
  name=UMN MapServer,
  description={Mapserver des OGC unter MIT Lizenz, Erstentwicklung durch Universität von Minnesota, welcher als CGI Modul und für verschiedene Sprachen  bereitsteht.},
  plural=UMN MapServer
}
\newglossaryentry{geoserver}
{
  name=GeoServer,
  description={Ist ein freier OGC konformer Mapserver der Open Source Geospatial Foundation, geschrieben in Java.},
  plural=GeoServer
}
\newglossaryentry{spark}
{
  name=Spark,
  description={Apache Spark steht unter der Apache License 2.0 und ist ein Framework zur Datenverarbeitung in Clustersystemen. Es tritt mit Hadoop in Konkurrenz und arbeitet mit HDFS, Apache Cassandra, OpenStack Swift, Amazon S3 und Accumulo zusammen.},
  plural=Spark
}
\newglossaryentry{storm}
{
  name=Storm,
  description={Apache Storm ist ein Framework speziell für Stapelverarbeitung von Datenströmen durch verteilte Prozesse. Es steht unter der Apache License 2.0},
  plural=Storm
}
\newglossaryentry{pig}
{
  name=Pig,
  description={Als Apache Projekt dient Pig zur Abstraktion von Java MapReduce Jobs in der Sprache Pig Latin. Ziel ist eine Vereinfachung von MapReduce mit der gleichzeitigen Einbindung externer Funktionen.},
  plural=Pig
}
\newglossaryentry{cascading}
{
  name=Cascading,
  description={Das Java Framework Cascading steht unter der Apache License dient der Erstellung komplexer Datenverarbeitungsabläufe. Dafür wird MapReduce indirekt in vereinfachter Form zugänglich gemacht.},
  plural=Cascading
}
\newglossaryentry{r}
{
  name=R,
  description={Die plattformunabhängige Programmiersprache R steht unter der GNU General Public License ist wird für statistisches Rechnen und dessen grafische Aufbereitung verwendet.},
  plural=R
}
\newglossaryentry{scala}
{
  name=Scala,
  description={Scala ist eine objektorientierte funktionale Programmiersprache mit einem statischen Typsystem und ist auf der JVM und LLVM lauffähig.},
  plural=Scala
}
\newglossaryentry{wcs_glos}
{
  name=Web Coverage Service,
  description={Ein OGC konformer Dienst zum Abruf von multi-dimensionalen Daten mit Zeit- und Raumbezug. Diese sind über eine eigene Syntax mit ihren Metadaten abrufbar.},
  plural=Web Coverage Services
}
\newglossaryentry{wps_glos}
{
  name=Web Processing Service,
  description={Dieser OGC konformer Dienst ermöglicht die räumliche Analyse von Daten im geografischen Kontext. Dazu stellt der Dienst Clients Vorschriften und Modelle zur Verfügung.},
  plural=Web Processing Services
}
\newglossaryentry{prec_farm}
{
  name=Precision Farming,
  description={Bedeutet eine individuelle Betrachtung und Bewirtschaftung einzelner Teile von Flurstücken, wodurch Unterschiede des Bodens und die variierende Ertragsfähigkeit innerhalb einer Nutzfläche berücksichtigt werden.},
  plural=Precision Farming
}
\newglossaryentry{epsg-code}
{
  name=EPSG-Code,
  description={Dies ist eine vier- bis fünfstellige Ziffer zur eindeutigen Identifikation von räumlichen Referenzsystemen. Sie werden von der EPSG herausgegeben und finden weltweit Anwendung.},
  plural=EPSG-Codes
}
\newglossaryentry{esxi}
{
  name=VMware ESXi,
  description={ESXi ist ein Hypervisor Stufe 1 des Unternehmens VMware und wird als Betriebssystem installiert.},
  plural=VMware ESXi
}
\newglossaryentry{vsphere}
{
  name=VMware vSphere,
  description={vSphere ist eine Sammlung von Systemen und Werkzeugen des Unternehmens VMware zur umfassenden Virtualisierung.},
  plural=VMware vSphere
}