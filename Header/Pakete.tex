% Anpassung an Landessprache
\usepackage{ngerman}
\usepackage[ngerman]{babel}
 
% Verwenden von Sonderzeichen und Silbentrennung
\usepackage[utf8]{inputenc}	
\usepackage[T1]{fontenc}			
\usepackage{textcomp} 
\usepackage{ae,aecompl}		                                                                    		% Euro-Zeichen und andere
\usepackage[babel,german=quotes]{csquotes}						% Anf�hrungszeichen
\RequirePackage[ngerman=ngerman-x-latest]{hyphsubst} 	% erweiterte Silbentrennung

%Schriftart
\usepackage{mathptmx}

% Befehle aus AMSTeX f�r mathematische Symbole z.B. \boldsymbol \mathbb
\usepackage{amsmath,amsfonts}

% Zeilenabst�nde und Seitenr�nder 
\usepackage{setspace}
\usepackage{geometry}

% Einbinden von JPG-Grafiken
\usepackage{graphicx}

% zum Umflie�en von Bildern
% Verwendung unter http://de.wikibooks.org/wiki/LaTeX-Kompendium:_Baukastensystem#textumflossene_Bilder
\usepackage{floatflt}

% Verwendung von vordefinierten Farbnamen zur Colorierung
% Palette und Verwendung unter http://kitt.cl.uzh.ch/kitt/CLinZ.CH/src/Kurse/archiv/LaTeX-Kurs-Farben.pdf
\usepackage[usenames,dvipsnames]{color} 

% Tabellen
\usepackage{array}
\usepackage{longtable}

% einfache Grafiken im Code
% Einf�hrung unter http://www.math.uni-rostock.de/~dittmer/bsp/pstricks-bsp.pdf
\usepackage{pstricks}

% Quellcodeansichten
\usepackage{verbatim}
\usepackage{moreverb} 											% f�r erweiterte Optionen der verbatim Umgebung
% Befehle und Beispiele unter http://www.ctex.org/documents/packages/verbatim/moreverb.pdf
\usepackage{listings} 											% f�r angepasste Quellcodeansichten siehe
% Kurzeinf�hrung unter http://blog.robert-kummer.de/2006/04/latex-quellcode-listing.html

% verlinktes und Farblich angepasstes Inhaltsverzeichnis
\usepackage[pdftex,
colorlinks=true,
linkcolor=InterneLinkfarbe,
urlcolor=ExterneLinkfarbe]{hyperref}
\usepackage[all]{hypcap}

% Glossar und Abbildungsverzeichnis
\usepackage[
xindy,          %indexing phase
nonumberlist, %keine Seitenzahlen anzeigen
acronym,      %ein Abk�rzungsverzeichnis erstellen
toc          %Eintr�ge im Inhaltsverzeichnis
]      %im Inhaltsverzeichnis auf section-Ebene erscheinen
{glossaries}
%\usepackage{acrodefplural}

% URL verlinken, lange URLs umbrechen
\usepackage{url}

% sorgt daf�r, dass Leerzeichen hinter parameterlosen Makros nicht als Makroendezeichen interpretiert werden
\usepackage{xspace}

% Beschriftungen f�r Abbildungen und Tabellen
\usepackage{caption}

% Entwicklerwarnmeldungen entfernen
\usepackage{scrhack}