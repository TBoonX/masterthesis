%allgemeine Formatangaben
\documentclass[
 a4paper, 										% Papierformat
 12pt,												% Schriftgr��e
 ngerman, 										% f�r Umlaute, Silbentrennung etc.
 titlepage,										% es wird eine Titelseite verwendet
 bibliography=totoc,					% Literaturverzeichnis im Inhaltsverzeichnis auff�hren
 listof=totoc,								% Verzeichnisse im Inhaltsverzeichnis auff�hren
 oneside, 										% einseitiges Dokument
 captions=nooneline,					% einzeilige Gleitobjekttitel ohne Sonderbehandlung wie mehrzeilige Gleitobjekttitel behandeln
 numbers=noenddot,						% �berschriften-??Nummerierung ohne Punkt am Ende
 parskip=half									% zwischen Abs�tzen wird eine halbe Zeile eingef�gt
 ]{scrbook}
 
 %Tiefe Inhaltsverzeichnis
 \setcounter{tocdepth}{1}

\newcommand{\titel}{Untersuchung freier verteilter geografischer Informationssysteme zur Verarbeitung agrartechnischer Kennzahlen}
\newcommand{\untertitel}{Am Beispiel des aktuellen Standes bei Agri~Con}
\newcommand{\abschlussart}{Master of Science (M.Sc.)}
\newcommand{\arbeit}{Masterarbeit}
\newcommand{\hochschule}{Hochschule f\"ur Technik, Wirtschaft und Kultur}
\newcommand{\fachbereich}{Fakult\"at Informatik, Mathematik und Naturwissenschaften}
\newcommand{\autor}{Kurt Junghanns}
\newcommand{\studiengang}{Masterstudiengang Informatik}
\newcommand{\matrikelnr}{59886}
\newcommand{\erstgutachter}{Prof. Dr. rer. nat. Thomas Riechert}
\newcommand{\zweitgutachter}{M. Sc. Volkmar Herbst}
\newcommand{\ort}{Leipzig}
\newcommand{\Satz}{Satz: \,\LaTeX{}} 			% einbinden von pers�nlichen Daten

% Anpassung an Landessprache
\usepackage{ngerman}
\usepackage[ngerman]{babel}
 
% Verwenden von Sonderzeichen und Silbentrennung
\usepackage[utf8]{inputenc}	
\usepackage[T1]{fontenc}			
\usepackage{textcomp} 
\usepackage{ae,aecompl}		                                                                    		% Euro-Zeichen und andere
\usepackage[babel,german=quotes]{csquotes}						% Anf�hrungszeichen
\RequirePackage[ngerman=ngerman-x-latest]{hyphsubst} 	% erweiterte Silbentrennung

%Schriftart
\usepackage{mathptmx}

% Befehle aus AMSTeX f�r mathematische Symbole z.B. \boldsymbol \mathbb
\usepackage{amsmath,amsfonts}

% Zeilenabst�nde und Seitenr�nder 
\usepackage{setspace}
%Angepasste Seitenränder
\usepackage[left=40mm,right=20mm,top=13mm,bottom=13mm,headsep=10mm,includeheadfoot]{geometry}

% Einbinden von JPG-Grafiken
\usepackage{graphicx}

% zum Umflie�en von Bildern
% Verwendung unter http://de.wikibooks.org/wiki/LaTeX-Kompendium:_Baukastensystem#textumflossene_Bilder
\usepackage{floatflt}

% Verwendung von vordefinierten Farbnamen zur Colorierung
% Palette und Verwendung unter http://kitt.cl.uzh.ch/kitt/CLinZ.CH/src/Kurse/archiv/LaTeX-Kurs-Farben.pdf
\usepackage[usenames,dvipsnames,svgnames,table,x11names]{xcolor} 
%\definecolor{GreyBlue}{HTML}{}
\colorlet{GrayBlue}{blue!50!gray}

% Tabellen
\usepackage{array}
\usepackage{longtable}

% einfache Grafiken im Code
% Einf�hrung unter http://www.math.uni-rostock.de/~dittmer/bsp/pstricks-bsp.pdf
\usepackage{pstricks}

% Quellcodeansichten
\usepackage{verbatim}
\usepackage{moreverb} 											% f�r erweiterte Optionen der verbatim Umgebung
% Befehle und Beispiele unter http://www.ctex.org/documents/packages/verbatim/moreverb.pdf
\usepackage{listings} 											% f�r angepasste Quellcodeansichten siehe
% Kurzeinf�hrung unter http://blog.robert-kummer.de/2006/04/latex-quellcode-listing.html
\lstset{basicstyle=\ttfamily,
  showstringspaces=false,
  commentstyle=\color{red},
  keywordstyle=\color{blue}
}
\lstset{breaklines}

% verlinktes und Farblich angepasstes Inhaltsverzeichnis
\usepackage[pdftex,
colorlinks=true,
linkcolor=InterneLinkfarbe,
urlcolor=ExterneLinkfarbe]{hyperref}
\usepackage[all]{hypcap}

% Glossar und Abbildungsverzeichnis
\usepackage[
xindy,          %indexing phase
nonumberlist, %keine Seitenzahlen anzeigen
acronym,      %ein Abk�rzungsverzeichnis erstellen
toc          %Eintr�ge im Inhaltsverzeichnis
]      %im Inhaltsverzeichnis auf section-Ebene erscheinen
{glossaries}
%\usepackage{acrodefplural}

% URL verlinken, lange URLs umbrechen
\usepackage{url}

% sorgt daf�r, dass Leerzeichen hinter parameterlosen Makros nicht als Makroendezeichen interpretiert werden
\usepackage{xspace}

% Beschriftungen f�r Abbildungen und Tabellen
\usepackage{caption}

% Entwicklerwarnmeldungen entfernen
\usepackage{scrhack}

% Rechnen in latex
\usepackage{fp}

% Dia Abbildungen
\usepackage{tikz}

% Anzahl an Tabellen und Abbildung zählen
\usepackage[figure,table]{totalcount}

% caption gruppierungen
\usepackage{subcaption}					% einbinden der verwendeten Latex-Pakete


\onehalfspacing 							% 1,5facher Zeilenabstand

\definecolor{InterneLinkfarbe}{rgb}{0.1,0.1,0.3} 	% Farbliche Absetzung von externen Links
\definecolor{ExterneLinkfarbe}{rgb}{0.1,0.1,0.7}	% Farbliche Absetzung von internen Links

% Einstellungen f�r Fu�noten:
\captionsetup{font=footnotesize,labelfont=sc,singlelinecheck=true,margin={5mm,5mm}}

% Stil der Quellenangabe
\bibliographystyle{alphadin}

% Stil Zitate
\renewenvironment{quote}
               {\list{}{\rightmargin\leftmargin}%
                \item\relax\small\begin{itshape}\ignorespaces}
               {\unskip\unskip\end{itshape}\endlist}

%Ausschluss von Schusterjungen
\clubpenalty = 10000
%Ausschluss von Hurenkindern
\widowpenalty = 10000

% Befehle, die Umlaute ausgeben, f�hren zu Fehlern, wenn sie hyperref als Optionen �bergeben werden
\hypersetup{
    pdftitle={\titel},
    pdfauthor={\autor},
    pdfcreator={\autor},
    pdfsubject={\titel},
    pdfkeywords={\titel},
}

% Beispiel f�r eine Listings-Codeumbebungen
% Bei mehreren Definitionen empfielt sich das auslagern in eine externe Datei
\lstloadlanguages{Java,HTML}
\lstset{
	frame=tb,
	framesep=5pt,
	basicstyle=\footnotesize\ttfamily,
	showstringspaces=false,
	keywordstyle=\ttfamily\bfseries\color{CadetBlue},
	identifierstyle=\ttfamily,
	stringstyle=\ttfamily\color{OliveGreen},
	commentstyle=\color{GrayBlue},
	rulecolor=\color{Gray},
	xleftmargin=5pt,
	xrightmargin=5pt,
	aboveskip=\bigskipamount,
	belowskip=\bigskipamount
} 

%Den Punkt am Ende jeder Beschreibung deaktivieren
\renewcommand*{\glspostdescription}{}

%Glossar-Befehle anschalten
\makeglossaries
%\glsenablehyper
%Eine Abkuerzung mit Glossareintrag
%\newacronym{AD}{AD}{Active Directory\protect\glsadd{glos:AD}}


% Abkuerzungen
\newacronym{mvcc}{MVCC}{Multi Version Currency Control}
\newacronym{acid}{ACID}{Atomicity, Consistency, Isolation und Durability}
\newacronym{base}{BASE}{Basically Available, Soft state, Eventual consistency}
\newacronym{gis}{GIS}{Geoinformationssystem}
\newacronym{hdfs}{HDFS}{Hadoop File System}
%\acrodefplural{giss}[GIS]{Geoinformationssysteme}


%Befehle für Glossar
\newglossaryentry{bonitur}
{
  name=Bonitur,
  description={landwirtschaftliche Beurteilung des Ackers und der Pflanzen zum Zwecke der Planung des Einsatzes von Dünger, Pestiziden, Fungiziden und Herbiziden.},
  plural=Bonituren
}
\newglossaryentry{umn}
{
  name=UMN MapServer,
  description={Mapserver des OGC unter MIT Lizenz, Erstentwicklung durch Universität von Minnesota, welcher als CGI Modul und für verschiedene Sprachen  bereitsteht.},
  plural=UMN MapServer
}
\newglossaryentry{geoserver}
{
  name=GeoServer,
  description={Ist ein freier OGC konformer Mapserver der Open Source Geospatial Foundation, geschrieben in Java.},
  plural=GeoServer
}
\newglossaryentry{spark}
{
  name=Spark,
  description={Apache Spark steht unter der Apache License 2.0 und ist ein Framework zur Datenverarbeitung in Clustersystemen. Es tritt mit Hadoop in Konkurrenz und arbeitet mit HDFS, Apache Cassandra, OpenStack Swift, Amazon S3 und Accumulo zusammen.},
  plural=Spark
}
\newglossaryentry{storm}
{
  name=Storm,
  description={Apache Storm ist ein Framework speziell für Stapelverarbeitung von Datenströmen durch verteilte Prozesse. Es steht unter der Apache License 2.0},
  plural=Storm
}
\newglossaryentry{pig}
{
  name=Pig,
  description={Als Apache Projekt dient Pig zur Abstraktion von Java MapReduce Jobs in der Sprache Pig Latin. Ziel ist eine Vereinfachung von MapReduce mit der gleichzeitigen Einbindung externer Funktionen.},
  plural=Pig
}
\newglossaryentry{cascading}
{
  name=Cascading,
  description={Das Java Framework Cascading steht unter der Apache License dient der Erstellung komplexer Datenverarbeitungsabläufe. Dafür wird MapReduce indirekt in vereinfachter Form zugänglich gemacht.},
  plural=Cascading
}
\newglossaryentry{r}
{
  name=R,
  description={Die plattformunabhängige Programmiersprache R steht unter der GNU General Public License ist wird für statistisches Rechnen und dessen grafische Aufbereitung verwendet.},
  plural=R
}
\newglossaryentry{scala}
{
  name=Scala,
  description={Scala ist eine objektorientierte funktionale Programmiersprache mit einem statischen Typsystem und ist auf der JVM und LLVM lauffähig.},
  plural=Scala
}
\newglossaryentry{wcs_glos}
{
  name=Web Coverage Service,
  description={Ein OGC konformer Dienst zum Abruf von multi-dimensionalen Daten mit Zeit- und Raumbezug. Diese sind über eine eigene Syntax mit ihren Metadaten abrufbar.},
  plural=Web Coverage Services
}
\newglossaryentry{wps_glos}
{
  name=Web Processing Service,
  description={Dieser OGC konformer Dienst ermöglicht die räumliche Analyse von Daten im geografischen Kontext. Dazu stellt der Dienst Clients Vorschriften und Modelle zur Verfügung.},
  plural=Web Processing Services
}
\newglossaryentry{prec_farm}
{
  name=Precision Farming,
  description={Bedeutet eine individuelle Betrachtung und Bewirtschaftung einzelner Teile von Flurstücken, wodurch Unterschiede des Bodens und die variierende Ertragsfähigkeit innerhalb einer Nutzfläche berücksichtigt werden.},
  plural=Precision Farming
}
\newglossaryentry{epsg-code}
{
  name=EPSG-Code,
  description={Dies ist eine vier- bis fünfstellige Ziffer zur eindeutigen Identifikation von räumlichen Referenzsystemen. Sie werden von der EPSG herausgegeben und finden weltweit Anwendung.},
  plural=EPSG-Codes
}
\newglossaryentry{esxi}
{
  name=VMware ESXi,
  description={ESXi ist ein Hypervisor Stufe 1 des Unternehmens VMware und wird als Betriebssystem installiert.},
  plural=VMware ESXi
}
\newglossaryentry{vsphere}
{
  name=VMware vSphere,
  description={vSphere ist eine Sammlung von Systemen und Werkzeugen des Unternehmens VMware zur umfassenden Virtualisierung.},
  plural=VMware vSphere
}



% Beschriftung der Abbildungen
\captionsetup[figure]{labelfont={},textfont={}}
% Beschriftung der Tabelle
\captionsetup[table]{labelfont={},textfont={}}


% Rechnungsvorschriften paket fp
%Ein Rechenbefehl
\FPset\Gesamtsumme{0}
\newcommand{\psum}[1]{%
% Addition ausführen
\FPadd\0\Gesamtsumme{#1}\global\let\Gesamtsumme\0%
#1
}