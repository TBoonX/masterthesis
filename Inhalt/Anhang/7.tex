\label{appendix:funktionstests} 
\capstartfalse
%Tabellen
\begin{table}[h!]
\centering
\small
\begin{tabular}{p{2.8cm}|p{12cm}}
\textbf{Testfall:} & Vorhandene Schnittstelle zu PostgreSQL. \\ \hline
\textbf{Beschreibung:} & Ein direkter Datenaustausch muss mit PostgreSQL möglich sein. Dabei sollen Datenbankkonfigurationen und Daten übertragen und verwendet werden können. \\ \hline
\textbf{Testdaten:} & Zufällige Einträge aus farm.farms. Das Datenbankschema wird zunächst ohne die Daten übernommen. \\ \hline
\textbf{Sollergebnis:} & Die Schemata sind mit SQL create erzeugbar und die Daten werden ohne Fehler und ohne Umwandlung eingefügt. \\ \hline
\textbf{Ist Ergebnis:} &  \\ \hline
\textbf{Bestanden:} & Nein \\
\end{tabular}
\caption*{FT01}
\end{table}

\begin{table}[h!]
\centering
\small
\begin{tabular}{p{2.8cm}|p{12cm}}
\textbf{Testfall:} & Vorhandene Schnittstelle zu \Gls{umn}. \\ \hline
\textbf{Beschreibung:} & Das System muss im \Gls{umn} als Datenquelle nutzbar sein. \\ \hline
\textbf{Testdaten:} & Mapfile zur Anzeige beispielhafter Schläge aus farm.fields der farmid 1177. Darin muss ein Ausschnitt mit 3 Schlägen zu sehen sein. \\ \hline
\textbf{Sollergebnis:} & Alle Schläge werden angezeigt. \\ \hline
\textbf{Ist Ergebnis:} &  \\ \hline
\textbf{Bestanden:} & Nein \\
\end{tabular}
\caption*{FT02}
\end{table}

\begin{table}[h!]
\centering
\small
\begin{tabular}{p{2.8cm}|p{12cm}}
\textbf{Testfall:} & Datenaustausch nach PostGIS Format. \\ \hline
\textbf{Beschreibung:} & Daten sind in einem PostGIS Format zu übertragen und zu speichern. \\ \hline
\textbf{Testdaten:} & Zufällige Einträge aus farm.fields und n.nsensorlogs. Das Datenbankschema wird zunächst ohne die Daten übernommen. \\ \hline
\textbf{Sollergebnis:} & Einträge aus den Tabellen werden ohne Umwandlung direkt in die Datenbank geschrieben. \\ \hline
\textbf{Ist Ergebnis:} &  \\ \hline
\textbf{Bestanden:} & Nein \\
\end{tabular}
\caption*{FT03}
\end{table}

\begin{table}[h!]
\centering
\small
\begin{tabular}{p{2.8cm}|p{12cm}}
\textbf{Testfall:} & Umwandlung zwischen Koordinatenreferenzsystemen mit EPSG Codes 4326 und 3857. \\ \hline
\textbf{Beschreibung:} & Vorhandene räumliche Daten werden zwischen 4326 nach 3857 umgewandelt. \\ \hline
\textbf{Testdaten:} & Zufällige Einträge aus nutrients.samples. \\ \hline
\textbf{Sollergebnis:} & Einträge werden mit st\_{}transform von 4326 nach 3857 und anders herum umgewandelt. \\ \hline
\textbf{Ist Ergebnis:} &  \\ \hline
\textbf{Bestanden:} & Nein \\
\end{tabular}
\caption*{FT04}
\end{table}

\capstarttrue