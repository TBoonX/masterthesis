\label{appendix:pgxcctlconfig} 
\begin{lstlisting}[language=bash,caption={Konfigurationsdatei pgxc-ctl}]
#!/usr/bin/env bash
#
# Postgres-XC Configuration file for pgxc_ctl utility. 
#
# Configuration file can be specified as -c option from pgxc_ctl command.   Default is
# $PGXC_CTL_HOME/pgxc_ctl.org.
#
# This is bash script so you can make any addition for your convenience to configure
# your Postgres-XC cluster.
#
# Please understand that pgxc_ctl provides only a subset of configuration which pgxc_ctl
# provide.  Here's several several assumptions/restrictions pgxc_ctl depends on.
#
# 1) All the resources of pgxc nodes has to be owned by the same user.   Same user means
#    user with the same user name.  User ID may be different from server to server.
#    This must be specified as a variable $pgxcOwner.
#
# 2) All the servers must be reacheable via ssh without password.   It is highly recommended
#    to setup key-based authentication among all the servers.
#
# 3) All the databases in coordinator/datanode has at least one same superuser.  Pgxc_ctl
#    uses this user to connect to coordinators and datanodes.   Again, no password should
#    be used to connect.  You have many options to do this, pg_hba.conf, pg_ident.conf and
#    others.  Pgxc_ctl provides a way to configure pg_hba.conf but not pg_ident.conf.   This
#    will be implemented in the later releases.
#
# 4) Gtm master and slave can have different port to listen, while coordinator and datanode
#    slave should be assigned the same port number as master.
#
# 5) Port nuber of a coordinator slave must be the same as its master.
#
# 6) Master and slave are connected using synchronous replication.  Asynchronous replication
#    have slight (almost none) chance to bring total cluster into inconsistent state.
#    This chance is very low and may be negligible.  Support of asynchronous replication
#    may be supported in the later release.
#
# 7) Each coordinator and datanode can have only one slave each.  Cascaded replication and
#    multiple slave are not supported in the current pgxc_ctl.
#
# 8) Killing nodes may end up with IPC resource leak, such as semafor and shared memory.
#    Only listening port (socket) will be cleaned with clean command.
#
# 9) Backup and restore are not supported in pgxc_ctl at present.   This is a big task and
#    may need considerable resource.
#
#========================================================================================
#
#
# pgxcInstallDir variable is needed if you invoke "deploy" command from pgxc_ctl utility.
# If don't you don't need this variable.
pgxcInstallDir=$HOME/pgxc
#---- OVERALL -----------------------------------------------------------------------------
#
pgxcOwner=$USER			# owner of the Postgres-XC databaseo cluster.  Here, we use this
						# both as linus user and database user.  This must be
						# the super user of each coordinator and datanode.
pgxcUser=$pgxcOwner		# OS user of Postgres-XC owner

tmpDir=/tmp					# temporary dir used in XC servers
localTmpDir=$tmpDir			# temporary dir used here locally

configBackup=n					# If you want config file backup, specify y to this value.
configBackupHost=pgxc-linker	# host to backup config file
configBackupDir=$HOME/pgxc		# Backup directory
configBackupFile=pgxc_ctl.bak	# Backup file name --> Need to synchronize when original changed.

#---- GTM ------------------------------------------------------------------------------------

# GTM is mandatory.  You must have at least (and only) one GTM master in your Postgres-XC cluster.
# If GTM crashes and you need to reconfigure it, you can do it by pgxc_update_gtm command to update
# GTM master with others.   Of course, we provide pgxc_remove_gtm command to remove it.  This command
# will not stop the current GTM.  It is up to the operator.


#---- GTM Master -----------------------------------------------

#---- Overall ----
gtmName=gtm
gtmMasterServer=ThinkCenter
gtmMasterPort=6660
gtmMasterDir=$HOME/data_gtm

#---- Configuration ---
gtmExtraConfig=none			# Will be added gtm.conf for both Master and Slave (done at initilization only)
gtmMasterSpecificExtraConfig=none	# Will be added to Master's gtm.conf (done at initialization only)

#---- GTM Slave -----------------------------------------------

# Because GTM is a key component to maintain database consistency, you may want to configure GTM slave
# for backup.

#---- Overall ------
gtmSlave=n					# Specify y if you configure GTM Slave.   Otherwise, GTM slave will not be configured and
							# all the following variables will be reset.
gtmSlaveName=gtmSlave
gtmSlaveServer=		# value none means GTM slave is not available.  Give none if you don't configure GTM Slave.
gtmSlavePort=20001			# Not used if you don't configure GTM slave.
gtmSlaveDir=$HOME/pgxc/nodes/gtm	# Not used if you don't configure GTM slave.
# Please note that when you have GTM failover, then there will be no slave available until you configure the slave
# again. (pgxc_add_gtm_slave function will handle it)

#---- Configuration ----
gtmSlaveSpecificExtraConfig=none # Will be added to Slave's gtm.conf (done at initialization only)

#---- GTM Proxy -------------------------------------------------------------------------------------------------------
# GTM proxy will be selected based upon which server each component runs on.
# When fails over to the slave, the slave inherits its master's gtm proxy.  It should be
# reconfigured based upon the new location.
#
# To do so, slave should be restarted.   So pg_ctl promote -> (edit postgresql.conf and recovery.conf) -> pg_ctl restart
#
# You don't have to configure GTM Proxy if you dont' configure GTM slave or you are happy if every component connects
# to GTM Master directly.  If you configure GTL slave, you must configure GTM proxy too.

#---- Shortcuts ------
gtmProxyDir=$HOME/data_gtm_proxy

#---- Overall -------
gtmProxy=y				# Specify y if you conifugre at least one GTM proxy.   You may not configure gtm proxies
						# only when you dont' configure GTM slaves.
						# If you specify this value not to y, the following parameters will be set to default empty values.
						# If we find there're no valid Proxy server names (means, every servers are specified
						# as none), then gtmProxy value will be set to "n" and all the entries will be set to
						# empty values.
gtmProxyNames=(gtm_pxy1)	# No used if it is not configured
gtmProxyServers=(ThinkCenter)			# Specify none if you dont' configure it.
gtmProxyPorts=(20001)				# Not used if it is not configured.
gtmProxyDirs=($gtmProxyDir)	# Not used if it is not configured.

#---- Configuration ----
gtmPxyExtraConfig=none		# Extra configuration parameter for gtm_proxy.  Coordinator section has an example.
gtmPxySpecificExtraConfig=(none none none none)

#---- Coordinators ----------------------------------------------------------------------------------------------------

#---- shortcuts ----------
coordMasterDir=$HOME/data_coordinator
coordSlaveDir=$HOME/pgxc/nodes/coord_slave
coordArchLogDir=$HOME/pgxc/nodes/coord_archlog

#---- Overall ------------
coordNames=(coord1)		# Master and slave use the same name
coordPorts=(6661)			# Master ports
poolerPorts=(20010)			# Master pooler ports
coordPgHbaEntries=(192.168.1.0/24)				# Assumes that all the coordinator (master/slave) accepts
												# the same connection
												# This entry allows only $pgxcOwner to connect.
												# If you'd like to setup another connection, you should
												# supply these entries through files specified below.
# Note: The above parameter is extracted as "host all all 0.0.0.0/0 trust".   If you don't want
# such setups, specify the value () to this variable and suplly what you want using coordExtraPgHba
# and/or coordSpecificExtraPgHba variables.
#coordPgHbaEntries=(::1/128)	# Same as above but for IPv6 addresses

#---- Master -------------
coordMasterServers=(ThinkCenter)		# none means this master is not available
coordMasterDirs=($coordMasterDir)
coordMaxWALsernder=1	# max_wal_senders: needed to configure slave. If zero value is specified,
						# it is expected to supply this parameter explicitly by external files
						# specified in the following.	If you don't configure slaves, leave this value to zero.
coordMaxWALSenders=($coordMaxWALsernder)
						# max_wal_senders configuration for each coordinator.

#---- Slave -------------
coordSlave=n			# Specify y if you configure at least one coordiantor slave.  Otherwise, the following
						# configuration parameters will be set to empty values.
						# If no effective server names are found (that is, every servers are specified as none),
						# then coordSlave value will be set to n and all the following values will be set to
						# empty values.
coordSlaveSync=y		# Specify to connect with synchronized mode.
coordSlaveServers=(node07 node08 node09 node06)			# none means this slave is not available
coordSlavePorts=(20004 20005 20004 20005)			# Master ports
coordSlavePoolerPorts=(20010 20011 20010 20011)			# Master pooler ports
coordSlaveDirs=($coordSlaveDir $coordSlaveDir $coordSlaveDir $coordSlaveDir)
coordArchLogDirs=($coordArchLogDir $coordArchLogDir $coordArchLogDir $coordArchLogDir)

#---- Configuration files---
# Need these when you'd like setup specific non-default configuration 
# These files will go to corresponding files for the master.
# You may supply your bash script to setup extra config lines and extra pg_hba.conf entries 
# Or you may supply these files manually.
coordExtraConfig=coordExtraConfig	# Extra configuration file for coordinators.  
						# This file will be added to all the coordinators'
						# postgresql.conf
# Pleae note that the following sets up minimum parameters which you may want to change.
# You can put your postgresql.conf lines here.
cat > $coordExtraConfig <<EOF
#================================================
# Added to all the coordinator postgresql.conf
# Original: $coordExtraConfig
log_destination = 'stderr'
logging_collector = on
log_directory = 'pg_log'
listen_addresses = '*'
max_connections = 100
EOF

# Additional Configuration file for specific coordinator master.
# You can define each setting by similar means as above.
coordSpecificExtraConfig=(none)
coordExtraPgHba=none	# Extra entry for pg_hba.conf.  This file will be added to all the coordinators' pg_hba.conf
coordSpecificExtraPgHba=(none)

#----- Additional Slaves -----
#
# Please note that this section is just a suggestion how we extend the configuration for
# multiple and cascaded replication.   They're not used in the current version.
#
coordAdditionalSlaves=n		# Additional slave can be specified as follows: where you
coordAdditionalSlaveSet=(cad1)		# Each specifies set of slaves.   This case, two set of slaves are
											# configured
cad1_Sync=n		  		# All the slaves at "cad1" are connected with asynchronous mode.
							# If not, specify "y"
							# The following lines specifies detailed configuration for each
							# slave tag, cad1.  You can define cad2 similarly.
cad1_Servers=(node08 node09 node06 node07)	# Hosts
cad1_dir=$HOME/pgxc/nodes/coord_slave_cad1
cad1_Dirs=($cad1_dir $cad1_dir $cad1_dir $cad1_dir)
cad1_ArchLogDir=$HOME/pgxc/nodes/coord_archlog_cad1
cad1_ArchLogDirs=($cad1_ArchLogDir $cad1_ArchLogDir $cad1_ArchLogDir $cad1_ArchLogDir)


#---- Datanodes -------------------------------------------------------------------------------------------------------

#---- Shortcuts --------------
datanodeMasterDir=$HOME/data
datanodeSlaveDir=$HOME/pgxc/nodes/dn_slave
datanodeArchLogDir=$HOME/pgxc/nodes/datanode_archlog

#---- Overall ---------------
#primaryDatanode=datanode1				# Primary Node.
# At present, xc has a priblem to issue ALTER NODE against the primay node.  Until it is fixed, the test will be done
# without this feature.
primaryDatanode=datanode1				# Primary Node.
datanodeNames=(datanode1)
datanodePorts=(6663)	# Master ports
datanodePoolerPorts=(20012)	# Master pooler ports
datanodePgHbaEntries=(192.168.1.0/24)	# Assumes that all the coordinator (master/slave) accepts
										# the same connection
										# This list sets up pg_hba.conf for $pgxcOwner user.
										# If you'd like to setup other entries, supply them
										# through extra configuration files specified below.
# Note: The above parameter is extracted as "host all all 0.0.0.0/0 trust".   If you don't want
# such setups, specify the value () to this variable and suplly what you want using datanodeExtraPgHba
# and/or datanodeSpecificExtraPgHba variables.
#datanodePgHbaEntries=(::1/128)	# Same as above but for IPv6 addresses

#---- Master ----------------
datanodeMasterServers=(ThinkCenter)	# none means this master is not available.
													# This means that there should be the master but is down.
													# The cluster is not operational until the master is
													# recovered and ready to run.	
datanodeMasterDirs=($datanodeMasterDir)
datanodeMaxWalSender=1								# max_wal_senders: needed to configure slave. If zero value is 
													# specified, it is expected this parameter is explicitly supplied
													# by external configuration files.
													# If you don't configure slaves, leave this value zero.
datanodeMaxWALSenders=($datanodeMaxWalSender)
						# max_wal_senders configuration for each datanode

#---- Slave -----------------
datanodeSlave=n			# Specify y if you configure at least one coordiantor slave.  Otherwise, the following
						# configuration parameters will be set to empty values.
						# If no effective server names are found (that is, every servers are specified as none),
						# then datanodeSlave value will be set to n and all the following values will be set to
						# empty values.
datanodeSlaveServers=(node07 node08 node09 node06)	# value none means this slave is not available
datanodeSlavePorts=(20008 20009 20008 20009)	# value none means this slave is not available
datanodeSlavePoolerPorts=(20012 20013 20012 20013)	# value none means this slave is not available
datanodeSlaveSync=y		# If datanode slave is connected in synchronized mode
datanodeSlaveDirs=($datanodeSlaveDir $datanodeSlaveDir $datanodeSlaveDir $datanodeSlaveDir)
datanodeArchLogDirs=( $datanodeArchLogDir $datanodeArchLogDir $datanodeArchLogDir $datanodeArchLogDir )

# ---- Configuration files ---
# You may supply your bash script to setup extra config lines and extra pg_hba.conf entries here.
# These files will go to corresponding files for the master.
# Or you may supply these files manually.
datanodeExtraConfig=none	# Extra configuration file for datanodes.  This file will be added to all the 
							# datanodes' postgresql.conf
datanodeSpecificExtraConfig=(none none none none)
datanodeExtraPgHba=none		# Extra entry for pg_hba.conf.  This file will be added to all the datanodes' postgresql.conf
datanodeSpecificExtraPgHba=(none none none none)

#----- Additional Slaves -----
datanodeAdditionalSlaves=n	# Additional slave can be specified as follows: where you
# datanodeAdditionalSlaveSet=(dad1 dad2)		# Each specifies set of slaves.   This case, two set of slaves are
											# configured
# dad1_Sync=n		  		# All the slaves at "cad1" are connected with asynchronous mode.
							# If not, specify "y"
							# The following lines specifies detailed configuration for each
							# slave tag, cad1.  You can define cad2 similarly.
# dad1_Servers=(node08 node09 node06 node07)	# Hosts
# dad1_dir=$HOME/pgxc/nodes/coord_slave_cad1
# dad1_Dirs=($cad1_dir $cad1_dir $cad1_dir $cad1_dir)
# dad1_ArchLogDir=$HOME/pgxc/nodes/coord_archlog_cad1
# dad1_ArchLogDirs=($cad1_ArchLogDir $cad1_ArchLogDir $cad1_ArchLogDir $cad1_ArchLogDir)

#---- WAL archives -------------------------------------------------------------------------------------------------
walArchive=n	# If you'd like to configure WAL archive, edit this section.
				# Pgxc_ctl assumes that if you configure WAL archive, you configure it
				# for all the coordinators and datanodes.
				# Default is "no".   Please specify "y" here to turn it on.
#
#		End of Configuration Section
#
#==========================================================================================================================

#========================================================================================================================
# The following is for extension.  Just demonstrate how to write such extension.  There's no code
# which takes care of them so please ignore the following lines.  They are simply ignored by pgxc_ctl.
# No side effects.
#=============<< Beginning of future extension demonistration >> ========================================================
# You can setup more than one backup set for various purposes, such as disaster recovery.
walArchiveSet=(war1 war2)
war1_source=(master)	# you can specify master, slave or ano other additional slaves as a source of WAL archive.
					# Default is the master
wal1_source=(slave)
wal1_source=(additiona_coordinator_slave_set additional_datanode_slave_set)
war1_host=node10	# All the nodes are backed up at the same host for a given archive set
war1_backupdir=$HOME/pgxc/backup_war1
wal2_source=(master)
war2_host=node11
war2_backupdir=$HOME/pgxc/backup_war2
#=============<< End of future extension demonistration >> ========================================================


\end{lstlisting} 
 
