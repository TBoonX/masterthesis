
\chapter*{Abstrakt}
\label{sec:Abstrakt}
%\begin{abstract}
%handlungsempfehlung
%Open Source - Quelloffen nicht vergessen!
In dieser Arbeit werden Quelloffene verteilte geografische Informationssysteme für den Anwendungsfall des Einsatzes bei der Agri~Con untersucht, bewertet und ausgewählt.
%Was ist Agri Con? - auch nennen warum nur freie software
Die Anforderungen der Agri~Con zur Auswahl eines Systems werden in Form von Qualitätsmetriken messbar gemacht, wodurch mit einer Nutzwertanalyse eine Bewertung möglich ist.
An Hand dieser Bewertung erfolgt die Auswahl eines geeigneten Systems.
%Anschließend erfolgt eine genauere Untersuchung des Systems.
Dieses wird anschließend genauer untersucht.
In dieser Untersuchung wird auf die Verwendung und den möglichen Einsatz bei Agri~Con eingegangen.
Weiterhin werden Tests zur Feststellung von ausgewählter Funktionalität und Leistungsfähigkeit bezüglich Schwächen des aktuellen Systems durchgeführt.
%Vergkeichbar! - mit VM vergleichbar gemacht - konkret werden
Die Arbeit schließt mit einer Zusammenfassung und Bewertung der gewonnenen Ergebnisse ab.

Im Ergebnis der Arbeit wird gezeigt, dass Funktionalität von Postgres-XL für den Anwendungsfall ausreichend ist, bei der Prämisse, dass Trigger in der aktuellen version nicht unterstützt wird.
Hinsichtlich der Leistungsfähigkeit konnte im Gegensatz zum Ist-Stand kein signifikanter Leistungszuwachs festgestellt werden.
Signifikant meint ... %
%dass das aktuelle System nicht durch zu Hilfenahme oder durch Austausch von GIS Frameworks verbessert werden kann.
Das dabei untersuchte Framework Postgres-XL ist für den Anwendungsfall nicht geeignet.
Außerdem kann diese Arbeit als Handlungsempfehlung zur Analyse von Frameworks bei ähnlichen Anwendungsfällen verwendet werden.