
\chapter*{Abstrakt}
\label{sec:Abstrakt}
%\begin{abstract}
%handlungsempfehlung
In dieser Arbeit werden quelloffene verteilte geografische Informationssysteme für den Anwendungsfall des Einsatzes bei der Agri~Con GmbH untersucht, bewertet und ausgewählt.
%Was ist Agri Con? - auch nennen warum nur freie software
Die Agri~Con GmbH agiert als Berater, Verkäufer und Dienstleister für Agrarunternehmen die ihre Produktionsprozesse durch den Einsatz von \Gls{prec_farm}-Technologien optimieren wollen.
Dabei steht die kostengünstige und effiziente Datenhaltung und -verarbeitung zunehmend im Mittelpunkt.
Die Anforderungen der Agri~Con GmbH zur Auswahl eines Systems werden in Form von Qualitätsmetriken messbar gemacht, wodurch eine Bewertung mit einer Nutzwertanalyse möglich ist.
Diese Nutzwertanalyse wird auf ausgewählte Frameworks wie GeoMesa, Postgres-XL und Rasdaman angewandt.
Anhand dieser Bewertung erfolgt die Auswahl des Frameworks mit dem höchsten Nutzwert, hier Postgres-XL.
%Anschließend erfolgt eine genauere Untersuchung des Systems.
Das ausgewählte Framework wird anschließend genauer untersucht und es wird die Verwendung sowie der mögliche praktische Einsatz bei der Agri~Con GmbH bewertet.
Weiterhin werden Tests zur Feststellung von ausgewählter Funktionalität und Leistungsfähigkeit bezüglich identifizierter Schwächen des aktuellen Systems durchgeführt.
Um eine Vergleichbarkeit zu gewährleisten, wurde das aktuelle System in eine virtuelle Maschine exportiert und Postgres-XL in eine identisch konfigurierte virtuelle Maschine installiert.
%Vergleichbar! - mit VM vergleichbar gemacht - konkret werden
Die Arbeit endet mit einer Zusammenfassung, einer Wertung der Ergebnisse und einem Ausblick auf die zukünftige Handhabung der räumlichen Daten bei der Agri~Con GmbH.

Im Ergebnis der Arbeit wird gezeigt, dass die untersuchte Funktionalität von Postgres-XL für den Anwendungsfall ausreichend ist, jedoch ein produktiver Einsatz unter den konkreten Praxisbedingungen nicht empfohlen werden kann.
Gegen einen produktiven Einsatz spricht, dass Trigger in der aktuellen Version nicht unterstützt werden und die Programmiersprache R nicht verteilt verwendet werden kann.
Das Ergebnis hinsichtlich der Leistungsfähigkeit ist kein signifikanter Leistungszuwachs im Gegensatz zum aktuellen Stand.
Signifikant meint 30\%{} bis 100\%{} Leistungszuwachs bei 200\%{} Hardwareeinsatz.
%dass das aktuelle System nicht durch zu Hilfenahme oder durch Austausch von GIS Frameworks verbessert werden kann.
Ein weiteres Ergebnis ist, dass diese Arbeit als Handlungsempfehlung zur Analyse von Frameworks bei ähnlichen Anwendungsfällen verwendet werden kann.