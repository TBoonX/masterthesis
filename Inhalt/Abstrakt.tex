
\chapter*{Abstrakt}
\label{sec:Abstrakt}
%\begin{abstract}
%handlungsempfehlung
In dieser Arbeit werden verteilte geografische Informationssysteme für den Anwendungsfall des Ist-Standes bei Agri~Con untersucht, bewertet und ausgewählt.
Die Anforderungen zur Auswahl eines Systems werden in Form von Qualitätskriterien messbar gemacht, wodurch mit einer Nutzwertanalyse eine Bewertung möglich ist.
An Hand dieser Bewertung Erfolg die Auswahl und weitere Untersuchung eines Systems.
In dieser Untersuchung wird auf die Verwendung und den möglichen Einsatz bei Agri~Con eingegangen.
Weiterhin werden Tests zur Feststellung von ausgewählter Funktionalität und Leistungsfähigkeit bezüglich Schwächen des Ist-Standes durchgeführt.
Die Arbeit endet mit einer Zusammenfassung und Bewertung der gewonnenen Ergebnisse.

Ergebnis ist, dass der Ist-Stand nicht durch zu Hilfenahme oder durch Austausch von Frameworks verbessert werden kann.
Das dabei untersuchte Framework Postgres-XL ist für den Anwendungsfall nicht geeignet.
Außerdem kann diese Arbeit als Handlungsempfehlung zur Analyse von Frameworks bei ähnlichen Anwendungsfällen verwendet werden.