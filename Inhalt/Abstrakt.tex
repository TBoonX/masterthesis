
\chapter*{Abstrakt}
\label{sec:Abstrakt}
%\begin{abstract}
%handlungsempfehlung
In dieser Arbeit werden quelloffene verteilte geografische Informationssysteme für den Anwendungsfall des Einsatzes bei der Agri~Con GmbH untersucht, bewertet und ausgewählt.
%Was ist Agri Con? - auch nennen warum nur freie software
Die Agri~Con GmbH agiert als Berater, Verkäufer und Dienstleister für Agrarunternehmer mit dem Schwerpunkt der Modernisierung der Landwirtschaftlichen Vorgänge, um ein niedrigen Verhältnis von Aufwand zu Ertrag zu erhalten.
Dieses mittelständische Unternehmen zielt auf eine kostengünstige und effiziente Datenhaltung und -verarbeitung ab.
Die Anforderungen der Agri~Con GmbH zur Auswahl eines Systems werden in Form von Qualitätsmetriken messbar gemacht, wodurch mit einer Nutzwertanalyse eine Bewertung möglich ist.
Diese Nutzwertanalyse wird auf die Frameworks GeoMesa, Postgres-XL und Rasdaman angewandt.
An Hand dieser Bewertung erfolgt die Auswahl des Frameworks mit dem höchten Nutzwert, hier Postgres-XL.
%Anschließend erfolgt eine genauere Untersuchung des Systems.
Dieses wird anschließend genauer untersucht, wobei auf die Verwendung und den möglichen Einsatz bei der Agri~Con GmbH eingegangen wird.
Weiterhin werden Tests zur Feststellung von ausgewählter Funktionalität und Leistungsfähigkeit bezüglich Schwächen des aktuellen Systems durchgeführt.
Um die Vergleichbarkeit zu gewährleisten, wurde das aktuelle System in eine virtuelle Maschine exportiert und Postgres-XL in eine identische virtuelle Maschine installiert.
%Vergleichbar! - mit VM vergleichbar gemacht - konkret werden
Die Arbeit endet mit einer Zusammenfassung, einer Empfehlung bzw. Wertung der Ergebnisse und einem Ausblick auf die zukünftige Handhabung der räumlichen Daten bei der Agri~Con GmbH.

Im Ergebnis der Arbeit wird gezeigt, dass die untersuchte Funktionalität von Postgres-XL für den Anwendungsfall ausreichend ist, jedoch ein produktiver Einsatz nicht empfohlen werden kann.
Gegen einen produktiven Einsatz spricht, dass Trigger in der aktuellen Version nicht unterstützt werden und die Erweiterung Plr nicht verteilt verwendet werden kann.
Das Ergebnis hinsichtlich der Leistungsfähigkeit ist kein signifikanter Leistungszuwachs im Gegensatz zum aktuellen Stand.
Signifikant meint 30\%{} bis 100\%{} Leistungszuwachs bei 200\%{} Hardwareeinsatz.
%dass das aktuelle System nicht durch zu Hilfenahme oder durch Austausch von GIS Frameworks verbessert werden kann.
Ein weiteres Ergebnis ist, dass diese Arbeit als Handlungsempfehlung zur Analyse von Frameworks bei ähnlichen Anwendungsfällen verwendet werden kann.