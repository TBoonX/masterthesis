\chapter{Gegenüberstellung}

Anhand der unter \ref{qualitätsmetriken} erstellten Metriken sind die Frameworks GeoMesa, ESRI GIS Tools for Hadopp und  zu vergleichen.
Der Vergleich findet im Rahmen einer groben Nutzwertanalyse statt.
Hierbei werden keine Daten von durchgeführten Tests herangezogen, sondern es wird anhand der Spezifikation der einzelnen Frameworks untersucht.

%TODO: Nutzertanalyse: Struktur, Bewertungssystem, erklären

Tabelle \ref{table:Wertungsmassstab} zeigt die für die Nutzwertanalyse notwendige Wertung der einzelnen Metriken.
\begin{table}[h]
\centering
\begin{tabular}{l|l}
\textbf{Metrik} & \textbf{Gewichtung} \\ \hline
Richtigkeit & 10 \\ \hline
Interoperabilität & 22 \\ \hline
Funktionsumfang & 18 \\ \hline
Fehlertoleranz & 8 \\ \hline
Dokumentation & 20 \\ \hline
Zeitverhalten & 16 \\ \hline
Modifizierbarkeit & 16
\end{tabular}
\caption{Bewertungsmaßstab}
\label{table:Wertungsmassstab}
\end{table}

Für jedes Framework wird eine Nutzwertanalyse durchgeführt und die dazugehörigen Tabellen dazu präsentiert.

\section{GeoMesa}

\begin{table}[h]
\centering
\begin{tabular}{l|l|l|l}
\textbf{Metrik} & \textbf{maximaler Wert} & \textbf{erreichter Wert} & \textbf{gewichteter Wert} \\ \hline
Funktionsumfang & \FPtrunc\Gesamtsumme\Gesamtsumme{0}\FPprint\Gesamtsumme & 44 & 0,14*44 \\ \hline
 &  &  &  \\ \hline
 &  &  &  \\ \hline
\textbf{Summe:} &  &  & 
\end{tabular}
\caption{Nutzwertanalyse GeoMesa}
\label{table:nutzwertanalyse-geomesa}
\end{table}