\chapter{Einleitung}


\section{Motivation}

Die Agri Con GmbH verwaltet als Akteur im Bereich \glqq Precision Farming\grqq\ täglich mehrere Millionen geografische Punktdaten. Diese Daten werden von aktiven Landwirtschaftsmaschinen und durch die Verarbeitung durch firmeninterne und firmenexterne Mitarbeiter sowie Systeme erzeugt. Weiterhin fallen dadurch indirekt Vektor- und Rasterdaten an, welche gespeichert und anschließend verarbeitet werden müssen.
Aus den Quelldaten werden Vektordaten für beispielsweise Verteilung der Grunddüngung erzeugt. Rasterdaten werden für \glqq N-Düngung\grqq\  verwendet, was unter anderem die Biomasse, die Nährstoffaufnahme und die Nährstoffverteilung beinhaltet.
Diese Menge an Daten ist essentiell für den Betrieb, weshalb diese strukturiert gespeichert und kostengünstig verarbeitet werden müssen. Nicht nur Agri Con steht vor dieser Notwendigkeit, sondern der Großteil der Unternehmen, die sich mit komplexen Geodaten beschäftigen, wie Monsanto, Google, Facebook, ESRI, OpenGEO, etc.


% Zur Verbreitung von Postgis leider nichts gefunden


\section{Zielsetzung}

Die aktuellen Werkzeuge kommen an ihre Grenzen wenn große Datenmengen zur Laufzeit bearbeitet werden müssen. Es ist zu untersuchen welche Vorteile andere Datenhaltungssysteme bieten bzw. welche alternativen Herangehensweisen wie NoSQL, die Verwendung von caching und die verteilte Datenhaltung verwendet werden können.
Dafür sind existierende \Glspl{gis} zu untersuchen und deren Eignung für den in Kapitel \ref{chapter:ausgangsszenario} beschriebenen Anwendungsfall festzustellen. Die Schwerpunkte der Untersuchung sind die Möglichkeiten und die Leistungsfähigkeit der räumlichen Datenverarbeitung und nicht die Formen der Datendarstellung.
Dabei werden NoSQL und Open-Source Systeme bevorzugt behandelt.
Aus geeigneten Frameworks wird eines ausgewählt. Dieses wird speziell untersucht und eine prototypische Installation\footnote{Dabei kann eine Installation aus mehreren Frameworks bestehen und eigens implementierte Funktionalitäten enthalten} erstellt.
Schlussendlich soll eine Entscheidungsgrundlage anhand von Qualitätsmerkmalen für die teilweise Ersetzung des Ist-Standes gegeben werden.
% Die Auswahl zwischen vorhandenen Systemen nach ausgesuchten Merkmalen soll für ähnliche  Untersuchungen als Handelsempfehlung dienen.

\section{Aufbau der Arbeit}


Zu Beginn werden theoretischen Grundlagen zu Datenbanken, geographischer Datenverarbeitung, NoSQL und Leistungstests festgehalten.
Anschließend definiert Kapitel 3 das Ausgangsszenario, für welches die Systeme analysiert und getestet werden sollen.
Darauf folgend bewertet eine Nutzwertanalyse ausgewählte Frameworks nach den Anforderungen des Anwendungsfalles.
%Die darauf folgenden Kapitel stellen die ausgewählten Systeme unter den Gesichtspunkten Aufbau, Installation, Datenimport, Verarbeitung, Schnittstelle und Leistungstest dar.
Das vorletzte Kapitel stellt das ausgewählte Framework unter den Punkten Aufbau, Installation, Datenimport, Verarbeitung, Schnittstelle und Leistungstest dar.
Die Thesis endet mit einer Zusammenfassung, einer Empfehlung bzw. Wertung der Ergebnisse und einem Ausblick auf die zukünftige Handhabung der räumlichen Daten bei Agricon.