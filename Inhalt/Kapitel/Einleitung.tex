\chapter{Einleitung}


\section{Motivation}

Die Agri~Con GmbH verwaltet als Akteur im Bereich \Gls{prec_farm} täglich mehrere Millionen räumliche Daten. Diese Daten werden von mit moderner Technik ausgestatteten Landwirtschaftsmaschinen und durch die Verarbeitung durch firmeninterne und firmenexterne Mitarbeiter sowie Systeme erzeugt. Weiterhin fallen dadurch indirekt Vektor- und Rasterdaten an, welche gespeichert und anschließend verarbeitet werden.
Aus den Quelldaten werden Vektordaten für beispielsweise Verteilung der Grunddüngung erzeugt. Rasterdaten werden für die Stickstoffdüngung oder auch \glqq N-Düngung\grqq\ genannt verwendet, was unter anderem die Biomasse, die Nährstoffaufnahme und die Nährstoffverteilung beinhaltet.
Diese Menge an Daten ist essentiell für das Unternehmen und dessen Kunden, weshalb diese strukturiert gespeichert und kostengünstig verarbeitet werden müssen. Nicht nur Agri~Con steht vor dieser Notwendigkeit, sondern der Großteil der Unternehmen, die sich mit komplexen Geodaten beschäftigen, wie Monsanto, Google, Facebook, ESRI, OpenGEO, etc.


% Zur Verbreitung von Postgis leider nichts gefunden


\section{Zielsetzung}

Die aktuellen Werkzeuge erfüllen nicht alle Anforderungen, wenn große Datenmengen zur Laufzeit bearbeitet werden müssen. Es ist zu untersuchen welche Vorteile alternative Datenhaltungssysteme bieten bzw. welche alternativen Herangehensweisen wie NoSQL und die verteilte Datenhaltung geeignet sind.
Dafür sind existierende \Glspl{gis} zu untersuchen und deren Eignung für den in Kapitel \ref{chapter:ausgangsszenario} beschriebenen Anwendungsfall festzustellen. Die Schwerpunkte der Untersuchung sind die Möglichkeiten und die Leistungsfähigkeit der räumlichen Datenverarbeitung und nicht die Formen der Datendarstellung.
Dabei werden NoSQL und Open-Source Frameworks bevorzugt untersucht.
Aus geeigneten wird eines ausgewählt. Dieses wird speziell untersucht und eine prototypische Installation\footnote{Dabei kann eine Installation aus mehreren Frameworks bestehen und eigens implementierte Funktionalitäten enthalten} erstellt.
Schlussendlich soll eine Entscheidungsgrundlage anhand von bewerteten Qualitätsmerkmalen und der dargestellt Entwurf für die teilweise Ersetzung des Ist-Standes gegeben werden.
% Die Auswahl zwischen vorhandenen Systemen nach ausgesuchten Merkmalen soll für ähnliche  Untersuchungen als Handelsempfehlung dienen.

\section{Aufbau der Arbeit}


Zu Beginn werden theoretische Grundlagen zu Datenbanken, geographischer Datenverarbeitung, NoSQL und Tests festgehalten.
Daran schließt sich die Darstellung und Begründung des methodischen Vorgehens an.
Anschließend definiert Kapitel \ref{chapter:ausgangsszenario} das Ausgangsszenario, für welches die Frameworks analysiert und getestet werden sollen, sowie die dazugehörigen Anforderungen.
Darauf folgend bewertet eine Nutzwertanalyse ausgewählte Frameworks nach den Anforderungen des Anwendungsfalles.
%Die darauf folgenden Kapitel stellen die ausgewählten Systeme unter den Gesichtspunkten Aufbau, Installation, Datenimport, Verarbeitung, Schnittstelle und Leistungstest dar.
Das vorletzte Kapitel stellt das ausgewählte Framework unter den Punkten Aufbau, Installation, Datenimport, Verarbeitung und Schnittstellen dar und präsentiert den Entwurf sowie die Umsetzung des Prototypen.
Dazu werden Funktions- und Leistungstest durchgeführt und ausgewertet.
Die Arbeit endet mit einer Zusammenfassung, einer Empfehlung bzw. Wertung der Ergebnisse und einem Ausblick auf die zukünftige Handhabung der räumlichen Daten bei Agri~Con.