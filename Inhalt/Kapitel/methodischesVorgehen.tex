\chapter{methodisches Vorgehen}

- Grundlagen sind notwendig zur Bewertung der Systeme
- die exakte Definition der Anforderungen und des Ist-Standes ist Voraussetzung zur Bewertung
- die Struktur der Untersuchung der systeme dient dem einheitlichen vergleich und der Nachvollziehbarkeit
- konkret: aufbau ist wichtig  zur beurteilung der leistungsfähigkeit; Installation und Schnittstelle sollen aufzeigen, dass das system unter der definierten Umgebung lauffähig ist; Datenimport zeigt die integrationsfähigkeit; Verarbeitung als wesentlicher Punkt, neben Aufbau Voraussetzung für Leistung, ; Leistungstests als quelle zum Vergleich der leistung
- Verweis auf bestehende vergleiche (literatur)

%Dan:
%du wirst bestimmt dinge finden wie "Benchmark von Datenbanken"
% und ich meine ein gis ist ja nur eine DB (mit effizienter Datenhaltung für geographische Figuren) mit erweiterten Funktionen im gis kontext
%zu bewertung von software/is würde ich auf softwaretechnik bücher verweisen ... die du verwenden kannst, um dir eine eigene Liste von  Bewertungskritierien für deinen Anwendungsfall zu erstellen