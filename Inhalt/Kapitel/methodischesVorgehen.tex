\chapter{methodisches Vorgehen}

- Grundlagen sind notwendig zur Bewertung der Systeme
- die exakte Definition der Anforderungen und des Ist-Standes ist Voraussetzung zur Bewertung
- die Struktur der Untersuchung der systeme dient dem einheitlichen vergleich und der Nachvollziehbarkeit
- konkret: aufbau ist wichtig  zur beurteilung der leistungsfähigkeit; Installation und Schnittstelle sollen aufzeigen, dass das system unter der definierten Umgebung lauffähig ist; Datenimport zeigt die integrationsfähigkeit; Verarbeitung als wesentlicher Punkt, neben Aufbau Voraussetzung für Leistung, ; Leistungstests als quelle zum Vergleich der leistung
- Verweis auf bestehende vergleiche (literatur)

%Dan:
%du wirst bestimmt dinge finden wie "Benchmark von Datenbanken"
% und ich meine ein gis ist ja nur eine DB (mit effizienter Datenhaltung für geographische Figuren) mit erweiterten Funktionen im gis kontext
%zu bewertung von software/is würde ich auf softwaretechnik bücher verweisen ... die du verwenden kannst, um dir eine eigene Liste von  Bewertungskritierien für deinen Anwendungsfall zu erstellen

Das Thema "Untersuchung und Optimierung von verteilten geografischen Informationssystemen zur Verarbeitung agrartechnischer Kennzahlen" besteht aus vier Unteraufgaben:\\
\textbf{Untersuchung bestehender Frameworks anhand von Qualitätsmerkmalen}\\
Aus gegebenen Anforderungen\footnote{siehe \ref{Anforderungen}} sind Qualitätsmerkmale zu erstellen und die Mindestmenge an notwendiger Qualität zu definieren.
Darauf aufbauend sind Qualitätsmetriken zu erstellen, welche die einzelnen Qualitäten messbar machen.
Einer Menge von scheinbar geeigneten Frameworks ist anhand der definierten Metriken zu untersuchen.
Dabei sind jedoch nur die wesentlichen Qualitäten zu untersuchen.


\textbf{Auswahl eines Frameworks}\\
Aus der untersuchten Menge ist eines auszuwählen.

Auf Grund der sehr konkreten Anforderungen an das gesuchte Framework, scheint es unwahrscheinlich das mehrere Frameworks die Mindestanforderungen erfüllen..
Dies ist eine subjektive Einschätzung des Autors, was es in dieser Teilaufgabe zu beweisen gilt.
Deshalb ist die Auswahl des Frameworks anhand dessen Spezifikation durchzuführen. % TODO: Knappheit an geeigneten Frameworks verdeutlichen - vllt. spezielles Anwednungsgebiet der Frameworks benennen

\textbf{Entwurf eines Prototypen}\\
Das ausgewählte Framework ist detailliert zu untersuchen.
Aus dieser Untersuchung soll ein Entwurf zum Einsatz dieses Frameworks entstehen.
Dabei ist besonders dessen Architektur zu beleuchten, eine Konfiguration enthalten und fehlende Funktionalitäten mit nachträglicher Implementierung...

\textbf{Prototypische Implementierung}\\
Der Entwurf wird schlussendlich umgesetzt und anhand der Metriken mit Funktions- und Leistungstests bewertet.
Diese detaillierte Bewertung erfolgt im Rahmen einer Nutzwertanalyse.
Ziel ist dabei die Eignung des Prototypen hinsichtlich des geforderten Einsatzzweckes darzustellen.