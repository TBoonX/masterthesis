\chapter{Realisierung mit Postgres-XL}
\label{chapter:postgresxl}

\section{Verwendung}
%Indexstrukturen nicht vergessen!
%Mehrrechner-Datenbanksystem?
%replikationsverfahren wichtig?
%auf Fehlen von Triggern eingehen

\subsection{Installation}
%mit Systemvoraussetzungen
\subsubsection{Systemvoraussetzungen}
Die Dokumentation\footnote{\url{http://files.postgres-xl.org/documentation/install-requirements.html}} verwendet hierbei deckungsgleich die offizielle Dokumentation zu den Systemanforderungen von PostgreSQL\footnote{\url{http://www.postgresql.org/docs/9.2/static/install-requirements.html}}.
Dabei wird ein Linux Betriebssystem, 155MB freien Festplattenspeicher für die Übersetzung, Installation und Erstellung eines leeren Datenbankclusters sowie eine Menge von Paketen genannt.
Diese ist: GNU make, gcc, tar und zlib als benötigte sowie libperl oder libpython als optionale Pakete.
Zusätzlich werden für die Erzeugung der Dokumentation oder über Übersetzung des Quellcodes weitere Pakete benötigt.
Diese Anforderungen setzen eine Standard Installation voraus.
Neben Linux Derivaten wird auch FreeBSD und Max OS X unterstützt.
Die Prozessorarchitektur von Intel wird unterstützt, andere sind laut Dokumentation ebenso verwendbar.
Im Rahmen dieser Arbeit konnte Postgres-XL auch auf einem Raspberry Pi 1 Model B, basierend auf einem ARMv6 Prozessor und dem Linux Derivat Raspbian, installiert und verwendet werden.

\subsubsection{Installation}
Postgres-XL steht als RPM und direkt als Quellcode bereit.
Davon verfügbare RPM Pakete sind jedoch von Mai 2014 und somit veraltet.
Im Rahmen dieser Arbeit wurde der aktuelle Quellcode von github verwendet.
Die Übersetzung des Quellcodes erfolgt mit einer im Linux Umfeld oft verwendeten configure, make und make install Routine.
Der aktuelle Quellcode wird mit dem Kommandozeilen- und Versionsverwaltungstool git auf den Computer geladen und die darin enthaltene Datei configure mit zusätzlichen Parametern zum Zwecke der Einrichtung der anschließenden Installation ausgeführt.
Die Ausführung von make übersetzt den Quellcode und der Parameter install kopiert die Übersetzungen in die mit configure festgelegten Ordner.
Im Anhang \ref{appendix:install} ist das Skript zur Installation von Postgres-XL auf dem Testsystem zu sehen.
Das Installationsskript muss für andere Systemumgebungen angepasst werden, da Pfade und notwendige Pakete unterschiedlich sein können.
Weiterhin ist zu erwähnen, dass Änderungen an der Kernel-Konfiguration vorzunehmen  sind, jedoch ebenso abhängig von der Systemumgebung.
Im Testsystem musste der Wert des für jede Anwendung nutzbaren geteilten Speichers erhöht werden, um PostgreSQL starten zu können.\\
Um das Kommandozeilentool pgxc\_{}ctl nutzen zu können, muss im Quellcode Ordner in ./contrib/pgxc\_{}ctl gewechselt und dort make sowie make install ausgeführt werden. damit wird das Tool übersetzt und in den in der vorangegangenen Installation festgelegten Ordner kopiert.

\subsubsection{Einrichtung}
Postgres-Xl ist für den Einsatz in einem Cluster konzipiert.
So muss die Installation für jeden Knoten vorgenommen werden.
Die Einrichtung der einzelnen Knoten variiert je nach Art des Knotens, wobei ein Knoten entweder eine GTM Instanz oder mehrere Coordinator sowie DataNodes Instanzen mit einer GTM-Proxy Instanz enthält.
Jede Instanz kann automatisiert mit pgxc\_{}ctl oder manuell erstellt und konfiguriert werden.
Das Testsystem wurde mit pgxc\_{}ctl eingerichtet.
Voraussetzung der Nutzung von pgxc\_{}ctl ist der Zugang zu allen Knoten per SSH ohne Passwortabfrage für den selben Benutzer und die Vergabe eindeutiger Hostnames an die Knoten.
Ist dies gegeben, kann pgxc\_{}ctl in der Kommandozeile gestartet werden.
Mit dem Kommando \textit{prepare config} erzeugt pgxc\_{}ctl eine Konfigurationsdatei unter dem in der Umgebungsvariable PGXC\_{}CTL\_{}HOME festgelegten Ordner.
In dieser Datei werden alle Elemente des Clusters definiert.
Dazu zählt: GTM, GTM-Proxys, Coordinators und DataNodes.
Außerdem können zu allen vier Typen Slaves definiert werden, welche bei Ausfall des Elementes dessen Aufgaben übernehmen.
So wird pro Element das Arbeitsverzeichnis, der Name, der Host, der Port, optionale Konfigurationsparameter, pg\_{}hba Einträge, pgPool Port und der Ordner der Logdateien festgelegt.
Eine detaillierte Beschreibung befindet sich in der Dokumentation.\footnote{\url{http://files.postgres-xl.org/documentation/pgxc-ctl.html}}
Anhang \ref{appendix:pgxcctlconfig} enthält eine beispielhafte Konfigurationsdatei.
%schlussendliche Konfigurationdatei mit aufführen

\subsection{Schnittstelle}
%Datenimport gehört dazu
Der Zugriff auf Daten eines Postgres-XL Clusters erfolgt über die Coordinators.
Dazu sind Programme und Tools aus dem PostgreSQL Umfeld zu verwenden.
Dazu zählen \Gls{jdbc}, das Kommandozeilentool psql und das grafische Programm zur Datenbankverwaltung pgAdminIII.
Ebenso sind die SQL Befehle bis auf ein paar Ausnahmen deckungsgleich.
Diese Ausnahmen beziehen sich auf die Verteilung der Daten und Verwaltung des Clusters.
Beispielsweise das SQL Statement \textit{Create Table tblname (serial id, text data) Distributed by Hash(id);} weicht durch die Ergänzung \textit{Distributed by} vom PostgreSQL SQL Syntax ab.
Jede Tabelle wird in jeder PostgreSQL Instanz erzeugt.
Dabei werden die Daten entweder nach einem Attribut verteilt oder zwischen den Datenbankinstanzen gespiegelt.
Das Schlüsselwort \textit{Distributed} veranlasst eine Verteilung der Daten, \textit{Replicate} dagegen eine Replikation der Tabelle über alle Nodes.
Weiterhin zu erwähnen ist der Befehl \textit{Create Node nodename With (TYPE=, HOST=, PORT=)}, welcher direkt als SQL Statement verwendet werden kann und dem Cluster einen Knoten hinzufügt.
Analog dazu existiert der Befehl \textit{Drop Node nodename} zum entfernen eines Knotens.
Wurde das Cluster verändert, ist dies mit \textit{Select * From pgxc\_{}pool\_{}reload();} für alle Knoten zu propagieren.
Weitere Abweichungen sind der Dokumentation zu entnehmen.\footnote{siehe \url{http://files.postgres-xl.org/documentation/sql-commands.html}}

Daten wurden mit dem Kommandozeilentool pg\_{}dump aus dem Ist-Stand in das Testsystem überführt.
Der Ist-Zustand verwendet PostgreSQL Version 9.3.
Um Daten zwischen verschiedenen Versionen auszutauschen, muss der Zwischenstand der Datenbank im Textformat erstellt und dieser angepasst als SQL Anweisungen in das andere System eingespielt werden.

%DBLink -> Kombatibiltätsprobleme? nicht festgestellt
%Übertragung im Prototyp???

%psql -c "copy (select list of column  from table_name ) to stdin " dbanme | psql -c "table_name(specify the column ) from stdout " targetDB
%https://stackoverflow.com/questions/14797327/copy-data-between-two-tables-in-postgresql-using-dblink-sql

\subsection{Verarbeitung}
%SQL
%PL/R
%R -> auf welchen Knoten wird das ausgeführt? Coordinator?
%PostGIS
Die Datenverarbeitung erfolgt analog des Ist-Standes bei Agri~Con.
Mit der Installation\footnote{Skript siehe \ref{appendix:postgis}} von PostGIS als Erweiterung, können die vorhandenen SQL Funktionen übernommen werden, ebenso die R Bibliotheken des speziellen Krigings.
Dafür ist neben R als Programmiersprache auf den Systemen, Pl/R als Erweiterung in Postgres-XL, zu installieren.
Die Installation von Pl/R mit dem Quellcode als Grundlage ist in Anhang \ref{appendix:plr} zu finden.

%\section{Prototyp}
%ERM festhalten

\section{Entwurf}
%Übernahme des Schemas
%spezielle Daten werden übernommen -> vllt. replikation möglich?
%eine Idee: quelldaten (punkte) in pgxc schreiben, dort Verarbeitung starten und Ergebnis auf pg kopieren
%zwei Datenquellen in Programmen ist kompliziert, server in pg nimmt performanz?
%vollständige Umstellung auf pgxc kann zu kombatibilitätsproblemen führen (Funktionen) und replikation zu hetzner wird schwierig

%Überlegen, ob konkreter Prototyp nicht zu kompliziert ist, und man stattdessen eine allgemeine Aussage zur Leistungsfähigkeit gegenüber PostgreSQL  trift

%welche Funktionen benötigen Daten aus beiden DBs?
%Gibt es Quelldaten die nur verarbeitet und nicht direkt gelesen werden?
%Die Replikation zu Hetzner mit berücksichtigen?
%Mail Service  -> existiert eine Übersicht?
%Maßnahmen Service?

\section{Implementierung}
%SQL mit dblnk skizzieren

\section{Tests}
%pgbench http://www.postgresql.org/docs/devel/static/pgbench.html

\subsection{Testumgebung}
Der Cluster wird mit Virtualisierung erstellt.
Dafür steht ein IBM Rack Server x3850 M2 zur Verfügung, mit folgender Ausstattung:
vier Xeon E7330 Quad-Core mit 2,4 GHz, 32GB DDR2 RAM, vier 500GB 2,5 Zoll SATA Festplatten von Western Digital mit 7.200 U/min und einem MR10k Raid-Controller.
Als Virtualisierungssoftware kommt \Gls{esxi} in der kostenlosen Version 5.5 mit \Gls{vsphere} als Testversion 6 zum Einsatz.
Mit dieser Virtualisierungslösung ist es möglich, Ressourcen explizit und ausschließlich einer VM zuzuordnen.
Ziel dieser unter Abbildung \ref{fig:physAufb} dargestellten Testumgebung ist es Postgres-XL mit PostgreSQL des Ist Standes zu vergleichen, einen homogenen Cluster zu erzeugen und Aussagen über die Skalierbarkeit von Postgres-XL zu treffen.
Die VMs sind in Abbildung \ref{fig:VMAufb} dargestellt.
\begin{figure}[h!]
\centering
% Graphic for TeX using PGF
% Title: /home/tboonx/Dokumente/Studium/Masterarbeit/masterthesis/Abbildungen/Testsystem_physischerAufbau.dia
% Creator: Dia v0.97.2
% CreationDate: Sun Mar 22 16:47:12 2015
% For: tboonx
% \usepackage{tikz}
% The following commands are not supported in PSTricks at present
% We define them conditionally, so when they are implemented,
% this pgf file will use them.
\ifx\du\undefined
  \newlength{\du}
\fi
\setlength{\du}{15\unitlength}
\begin{tikzpicture}
\pgftransformxscale{1.000000}
\pgftransformyscale{-1.000000}
\definecolor{dialinecolor}{rgb}{0.000000, 0.000000, 0.000000}
\pgfsetstrokecolor{dialinecolor}
\definecolor{dialinecolor}{rgb}{1.000000, 1.000000, 1.000000}
\pgfsetfillcolor{dialinecolor}
\pgfsetlinewidth{0.100000\du}
\pgfsetdash{}{0pt}
\pgfsetdash{}{0pt}
\pgfsetbuttcap
\pgfsetmiterjoin
\pgfsetlinewidth{0.100000\du}
\pgfsetbuttcap
\pgfsetmiterjoin
\pgfsetdash{}{0pt}
\definecolor{dialinecolor}{rgb}{0.000000, 0.000000, 0.000000}
\pgfsetstrokecolor{dialinecolor}
\draw (38.088600\du,26.070243\du)--(38.088600\du,30.624686\du)--(39.606748\du,30.624686\du)--(39.606748\du,26.070243\du)--cycle;
\pgfsetlinewidth{0.010000\du}
\pgfsetbuttcap
\pgfsetmiterjoin
\pgfsetdash{}{0pt}
\definecolor{dialinecolor}{rgb}{0.000000, 0.000000, 0.000000}
\pgfsetstrokecolor{dialinecolor}
\draw (38.088600\du,26.070243\du)--(38.088600\du,30.624686\du)--(39.606748\du,30.624686\du)--(39.606748\du,26.070243\du)--cycle;
\pgfsetlinewidth{0.100000\du}
\pgfsetbuttcap
\pgfsetmiterjoin
\pgfsetdash{}{0pt}
\definecolor{dialinecolor}{rgb}{0.000000, 0.000000, 0.000000}
\pgfsetstrokecolor{dialinecolor}
\draw (38.240415\du,26.222058\du)--(38.240415\du,28.043835\du)--(39.454933\du,28.043835\du)--(39.454933\du,26.222058\du)--cycle;
\pgfsetlinewidth{0.010000\du}
\pgfsetbuttcap
\pgfsetmiterjoin
\pgfsetdash{}{0pt}
\definecolor{dialinecolor}{rgb}{0.000000, 0.000000, 0.000000}
\pgfsetstrokecolor{dialinecolor}
\draw (38.240415\du,26.222058\du)--(38.240415\du,28.043835\du)--(39.454933\du,28.043835\du)--(39.454933\du,26.222058\du)--cycle;
\pgfsetbuttcap
\pgfsetmiterjoin
\pgfsetdash{}{0pt}
\definecolor{dialinecolor}{rgb}{0.000000, 0.000000, 0.000000}
\pgfsetstrokecolor{dialinecolor}
\draw (38.240415\du,26.525688\du)--(39.454933\du,26.525688\du);
\pgfsetbuttcap
\pgfsetmiterjoin
\pgfsetdash{}{0pt}
\definecolor{dialinecolor}{rgb}{0.000000, 0.000000, 0.000000}
\pgfsetstrokecolor{dialinecolor}
\draw (39.454933\du,26.829317\du)--(38.240415\du,26.829317\du);
\pgfsetbuttcap
\pgfsetmiterjoin
\pgfsetdash{}{0pt}
\definecolor{dialinecolor}{rgb}{0.000000, 0.000000, 0.000000}
\pgfsetstrokecolor{dialinecolor}
\draw (38.240415\du,27.132947\du)--(39.454933\du,27.132947\du);
\pgfsetbuttcap
\pgfsetmiterjoin
\pgfsetdash{}{0pt}
\definecolor{dialinecolor}{rgb}{0.000000, 0.000000, 0.000000}
\pgfsetstrokecolor{dialinecolor}
\draw (38.240415\du,27.436576\du)--(39.454933\du,27.436576\du);
\pgfsetbuttcap
\pgfsetmiterjoin
\pgfsetdash{}{0pt}
\definecolor{dialinecolor}{rgb}{0.000000, 0.000000, 0.000000}
\pgfsetstrokecolor{dialinecolor}
\draw (39.454933\du,27.740206\du)--(38.240415\du,27.740206\du);
\pgfsetlinewidth{0.100000\du}
\pgfsetbuttcap
\pgfsetmiterjoin
\pgfsetdash{}{0pt}
\definecolor{dialinecolor}{rgb}{0.000000, 0.000000, 0.000000}
\pgfsetstrokecolor{dialinecolor}
\draw (38.240415\du,28.195650\du)--(38.240415\du,28.651094\du)--(39.075396\du,28.651094\du)--(39.075396\du,28.195650\du)--cycle;
\pgfsetlinewidth{0.010000\du}
\pgfsetbuttcap
\pgfsetmiterjoin
\pgfsetdash{}{0pt}
\definecolor{dialinecolor}{rgb}{0.000000, 0.000000, 0.000000}
\pgfsetstrokecolor{dialinecolor}
\draw (38.240415\du,28.195650\du)--(38.240415\du,28.651094\du)--(39.075396\du,28.651094\du)--(39.075396\du,28.195650\du)--cycle;
\pgfsetbuttcap
\pgfsetmiterjoin
\pgfsetdash{}{0pt}
\definecolor{dialinecolor}{rgb}{0.000000, 0.000000, 0.000000}
\pgfsetstrokecolor{dialinecolor}
\draw (38.088600\du,28.954724\du)--(39.606748\du,28.954724\du);
\pgfsetlinewidth{0.100000\du}
\pgfsetbuttcap
\pgfsetmiterjoin
\pgfsetdash{}{0pt}
\definecolor{dialinecolor}{rgb}{0.000000, 0.000000, 0.000000}
\pgfsetstrokecolor{dialinecolor}
\draw (38.771766\du,29.106539\du)--(38.771766\du,29.182446\du)--(38.847674\du,29.182446\du)--(38.847674\du,29.106539\du)--cycle;
\pgfsetlinewidth{0.010000\du}
\pgfsetbuttcap
\pgfsetmiterjoin
\pgfsetdash{}{0pt}
\definecolor{dialinecolor}{rgb}{0.000000, 0.000000, 0.000000}
\pgfsetstrokecolor{dialinecolor}
\draw (38.771766\du,29.106539\du)--(38.771766\du,29.182446\du)--(38.847674\du,29.182446\du)--(38.847674\du,29.106539\du)--cycle;
\pgfsetlinewidth{0.100000\du}
\pgfsetbuttcap
\pgfsetmiterjoin
\pgfsetdash{}{0pt}
\definecolor{dialinecolor}{rgb}{0.000000, 0.000000, 0.000000}
\pgfsetstrokecolor{dialinecolor}
\draw (39.075396\du,29.106539\du)--(39.075396\du,29.182446\du)--(39.151303\du,29.182446\du)--(39.151303\du,29.106539\du)--cycle;
\pgfsetlinewidth{0.010000\du}
\pgfsetbuttcap
\pgfsetmiterjoin
\pgfsetdash{}{0pt}
\definecolor{dialinecolor}{rgb}{0.000000, 0.000000, 0.000000}
\pgfsetstrokecolor{dialinecolor}
\draw (39.075396\du,29.106539\du)--(39.075396\du,29.182446\du)--(39.151303\du,29.182446\du)--(39.151303\du,29.106539\du)--cycle;
\pgfsetlinewidth{0.100000\du}
\pgfsetbuttcap
\pgfsetmiterjoin
\pgfsetdash{}{0pt}
\definecolor{dialinecolor}{rgb}{0.000000, 0.000000, 0.000000}
\pgfsetstrokecolor{dialinecolor}
\draw (39.379026\du,29.106539\du)--(39.379026\du,29.182446\du)--(39.454933\du,29.182446\du)--(39.454933\du,29.106539\du)--cycle;
\pgfsetlinewidth{0.010000\du}
\pgfsetbuttcap
\pgfsetmiterjoin
\pgfsetdash{}{0pt}
\definecolor{dialinecolor}{rgb}{0.000000, 0.000000, 0.000000}
\pgfsetstrokecolor{dialinecolor}
\draw (39.379026\du,29.106539\du)--(39.379026\du,29.182446\du)--(39.454933\du,29.182446\du)--(39.454933\du,29.106539\du)--cycle;
\pgfsetlinewidth{0.100000\du}
\pgfsetbuttcap
\pgfsetmiterjoin
\pgfsetdash{}{0pt}
\definecolor{dialinecolor}{rgb}{0.000000, 0.000000, 0.000000}
\pgfsetstrokecolor{dialinecolor}
\draw (39.303118\du,28.651094\du)--(39.303118\du,28.802909\du)--(39.454933\du,28.802909\du)--(39.454933\du,28.651094\du)--cycle;
\pgfsetlinewidth{0.010000\du}
\pgfsetbuttcap
\pgfsetmiterjoin
\pgfsetdash{}{0pt}
\definecolor{dialinecolor}{rgb}{0.000000, 0.000000, 0.000000}
\pgfsetstrokecolor{dialinecolor}
\draw (39.303118\du,28.651094\du)--(39.303118\du,28.802909\du)--(39.454933\du,28.802909\du)--(39.454933\du,28.651094\du)--cycle;
\pgfsetbuttcap
\pgfsetmiterjoin
\pgfsetdash{}{0pt}
\definecolor{dialinecolor}{rgb}{0.000000, 0.000000, 0.000000}
\pgfsetstrokecolor{dialinecolor}
\draw (38.240415\du,28.423372\du)--(39.075396\du,28.423372\du);
\pgfsetlinewidth{0.100000\du}
\pgfsetbuttcap
\pgfsetmiterjoin
\pgfsetdash{}{0pt}
\definecolor{dialinecolor}{rgb}{0.000000, 0.000000, 0.000000}
\pgfsetstrokecolor{dialinecolor}
\draw (38.240415\du,29.030631\du)--(38.240415\du,29.258353\du)--(38.468137\du,29.258353\du)--(38.468137\du,29.030631\du)--cycle;
\pgfsetlinewidth{0.010000\du}
\pgfsetbuttcap
\pgfsetmiterjoin
\pgfsetdash{}{0pt}
\definecolor{dialinecolor}{rgb}{0.000000, 0.000000, 0.000000}
\pgfsetstrokecolor{dialinecolor}
\draw (38.240415\du,29.030631\du)--(38.240415\du,29.258353\du)--(38.468137\du,29.258353\du)--(38.468137\du,29.030631\du)--cycle;
\pgfsetlinewidth{0.100000\du}
\pgfsetbuttcap
\pgfsetmiterjoin
\pgfsetdash{}{0pt}
\definecolor{dialinecolor}{rgb}{0.000000, 0.000000, 0.000000}
\pgfsetstrokecolor{dialinecolor}
\draw (38.316322\du,27.816113\du)--(38.316322\du,27.892021\du)--(39.379026\du,27.892021\du)--(39.379026\du,27.816113\du)--cycle;
\pgfsetlinewidth{0.010000\du}
\pgfsetbuttcap
\pgfsetmiterjoin
\pgfsetdash{}{0pt}
\definecolor{dialinecolor}{rgb}{0.000000, 0.000000, 0.000000}
\pgfsetstrokecolor{dialinecolor}
\draw (38.316322\du,27.816113\du)--(38.316322\du,27.892021\du)--(39.379026\du,27.892021\du)--(39.379026\du,27.816113\du)--cycle;
\pgfsetbuttcap
\pgfsetmiterjoin
\pgfsetdash{}{0pt}
\definecolor{dialinecolor}{rgb}{0.000000, 0.000000, 0.000000}
\pgfsetstrokecolor{dialinecolor}
\draw (38.316322\du,28.271557\du)--(38.999489\du,28.271557\du);
\pgfsetbuttcap
\pgfsetmiterjoin
\pgfsetdash{}{0pt}
\definecolor{dialinecolor}{rgb}{0.000000, 0.000000, 0.000000}
\pgfsetstrokecolor{dialinecolor}
\draw (38.999489\du,28.347465\du)--(38.923581\du,28.347465\du);
\pgfsetbuttcap
\pgfsetmiterjoin
\pgfsetdash{}{0pt}
\definecolor{dialinecolor}{rgb}{0.000000, 0.000000, 0.000000}
\pgfsetstrokecolor{dialinecolor}
\draw (38.316322\du,28.347465\du)--(38.392230\du,28.347465\du);
\pgfsetlinewidth{0.100000\du}
\pgfsetbuttcap
\pgfsetmiterjoin
\pgfsetdash{}{0pt}
\definecolor{dialinecolor}{rgb}{0.000000, 0.000000, 0.000000}
\pgfsetstrokecolor{dialinecolor}
\draw (38.468137\du,28.271557\du)--(38.468137\du,28.347465\du)--(38.847674\du,28.347465\du)--(38.847674\du,28.271557\du)--cycle;
\pgfsetlinewidth{0.010000\du}
\pgfsetbuttcap
\pgfsetmiterjoin
\pgfsetdash{}{0pt}
\definecolor{dialinecolor}{rgb}{0.000000, 0.000000, 0.000000}
\pgfsetstrokecolor{dialinecolor}
\draw (38.468137\du,28.271557\du)--(38.468137\du,28.347465\du)--(38.847674\du,28.347465\du)--(38.847674\du,28.271557\du)--cycle;
\pgfsetbuttcap
\pgfsetmiterjoin
\pgfsetdash{}{0pt}
\definecolor{dialinecolor}{rgb}{0.000000, 0.000000, 0.000000}
\pgfsetstrokecolor{dialinecolor}
\draw (38.316322\du,27.967928\du)--(38.392230\du,27.967928\du);
\pgfsetbuttcap
\pgfsetmiterjoin
\pgfsetdash{}{0pt}
\definecolor{dialinecolor}{rgb}{0.000000, 0.000000, 0.000000}
\pgfsetstrokecolor{dialinecolor}
\draw (38.468137\du,27.967928\du)--(38.544044\du,27.967928\du);
\pgfsetbuttcap
\pgfsetmiterjoin
\pgfsetdash{}{0pt}
\definecolor{dialinecolor}{rgb}{0.000000, 0.000000, 0.000000}
\pgfsetstrokecolor{dialinecolor}
\draw (39.227211\du,27.967928\du)--(39.379026\du,27.967928\du);
\pgfsetbuttcap
\pgfsetmiterjoin
\pgfsetdash{}{0pt}
\definecolor{dialinecolor}{rgb}{0.000000, 0.000000, 0.000000}
\pgfsetstrokecolor{dialinecolor}
\draw (38.164507\du,30.548779\du)--(39.530840\du,30.548779\du);
\pgfsetbuttcap
\pgfsetmiterjoin
\pgfsetdash{}{0pt}
\definecolor{dialinecolor}{rgb}{0.000000, 0.000000, 0.000000}
\pgfsetstrokecolor{dialinecolor}
\draw (39.530840\du,30.472872\du)--(38.164507\du,30.472872\du);
\pgfsetbuttcap
\pgfsetmiterjoin
\pgfsetdash{}{0pt}
\definecolor{dialinecolor}{rgb}{0.000000, 0.000000, 0.000000}
\pgfsetstrokecolor{dialinecolor}
\draw (38.164507\du,30.396964\du)--(39.530840\du,30.396964\du);
\pgfsetbuttcap
\pgfsetmiterjoin
\pgfsetdash{}{0pt}
\definecolor{dialinecolor}{rgb}{0.000000, 0.000000, 0.000000}
\pgfsetstrokecolor{dialinecolor}
\draw (39.530840\du,30.321057\du)--(38.164507\du,30.321057\du);
\pgfsetbuttcap
\pgfsetmiterjoin
\pgfsetdash{}{0pt}
\definecolor{dialinecolor}{rgb}{0.000000, 0.000000, 0.000000}
\pgfsetstrokecolor{dialinecolor}
\draw (38.164507\du,30.245150\du)--(39.530840\du,30.245150\du);
\pgfsetbuttcap
\pgfsetmiterjoin
\pgfsetdash{}{0pt}
\definecolor{dialinecolor}{rgb}{0.000000, 0.000000, 0.000000}
\pgfsetstrokecolor{dialinecolor}
\draw (39.530840\du,30.169242\du)--(38.164507\du,30.169242\du);
\pgfsetbuttcap
\pgfsetmiterjoin
\pgfsetdash{}{0pt}
\definecolor{dialinecolor}{rgb}{0.000000, 0.000000, 0.000000}
\pgfsetstrokecolor{dialinecolor}
\draw (38.164507\du,30.093335\du)--(39.530840\du,30.093335\du);
\pgfsetbuttcap
\pgfsetmiterjoin
\pgfsetdash{}{0pt}
\definecolor{dialinecolor}{rgb}{0.000000, 0.000000, 0.000000}
\pgfsetstrokecolor{dialinecolor}
\draw (39.530840\du,30.017427\du)--(38.164507\du,30.017427\du);
\pgfsetbuttcap
\pgfsetmiterjoin
\pgfsetdash{}{0pt}
\definecolor{dialinecolor}{rgb}{0.000000, 0.000000, 0.000000}
\pgfsetstrokecolor{dialinecolor}
\draw (38.164507\du,29.941520\du)--(39.530840\du,29.941520\du);
\pgfsetbuttcap
\pgfsetmiterjoin
\pgfsetdash{}{0pt}
\definecolor{dialinecolor}{rgb}{0.000000, 0.000000, 0.000000}
\pgfsetstrokecolor{dialinecolor}
\draw (39.530840\du,29.865613\du)--(38.164507\du,29.865613\du);
\pgfsetbuttcap
\pgfsetmiterjoin
\pgfsetdash{}{0pt}
\definecolor{dialinecolor}{rgb}{0.000000, 0.000000, 0.000000}
\pgfsetstrokecolor{dialinecolor}
\draw (38.164507\du,29.789705\du)--(39.530840\du,29.789705\du);
\pgfsetbuttcap
\pgfsetmiterjoin
\pgfsetdash{}{0pt}
\definecolor{dialinecolor}{rgb}{0.000000, 0.000000, 0.000000}
\pgfsetstrokecolor{dialinecolor}
\draw (39.530840\du,29.713798\du)--(38.164507\du,29.713798\du);
\pgfsetbuttcap
\pgfsetmiterjoin
\pgfsetdash{}{0pt}
\definecolor{dialinecolor}{rgb}{0.000000, 0.000000, 0.000000}
\pgfsetstrokecolor{dialinecolor}
\draw (38.164507\du,29.637890\du)--(39.530840\du,29.637890\du);
\pgfsetbuttcap
\pgfsetmiterjoin
\pgfsetdash{}{0pt}
\definecolor{dialinecolor}{rgb}{0.000000, 0.000000, 0.000000}
\pgfsetstrokecolor{dialinecolor}
\draw (39.530840\du,29.561983\du)--(38.164507\du,29.561983\du);
% setfont left to latex
\definecolor{dialinecolor}{rgb}{0.000000, 0.000000, 0.000000}
\pgfsetstrokecolor{dialinecolor}
\node at (38.636700\du,24.819100\du){VMware vCenter Server};
% setfont left to latex
\definecolor{dialinecolor}{rgb}{0.000000, 0.000000, 0.000000}
\pgfsetstrokecolor{dialinecolor}
\node at (38.636700\du,25.619100\du){VMware vSphere Client};
\pgfsetlinewidth{0.100000\du}
\pgfsetdash{}{0pt}
\pgfsetdash{}{0pt}
\pgfsetbuttcap
\pgfsetmiterjoin
\pgfsetlinewidth{0.100000\du}
\pgfsetbuttcap
\pgfsetmiterjoin
\pgfsetdash{}{0pt}
\definecolor{dialinecolor}{rgb}{0.000000, 0.000000, 0.000000}
\pgfsetstrokecolor{dialinecolor}
\draw (35.650000\du,34.105300\du)--(35.650000\du,35.713097\du)--(42.081186\du,35.713097\du)--(42.081186\du,34.105300\du)--cycle;
\pgfsetlinewidth{0.010000\du}
\pgfsetbuttcap
\pgfsetmiterjoin
\pgfsetdash{}{0pt}
\definecolor{dialinecolor}{rgb}{0.000000, 0.000000, 0.000000}
\pgfsetstrokecolor{dialinecolor}
\draw (35.650000\du,34.105300\du)--(35.650000\du,35.713097\du)--(42.081186\du,35.713097\du)--(42.081186\du,34.105300\du)--cycle;
\pgfsetbuttcap
\pgfsetmiterjoin
\pgfsetdash{}{0pt}
\definecolor{dialinecolor}{rgb}{0.000000, 0.000000, 0.000000}
\pgfsetstrokecolor{dialinecolor}
\draw (36.775458\du,34.748419\du)--(36.614678\du,34.909198\du);
\pgfsetbuttcap
\pgfsetmiterjoin
\pgfsetdash{}{0pt}
\definecolor{dialinecolor}{rgb}{0.000000, 0.000000, 0.000000}
\pgfsetstrokecolor{dialinecolor}
\draw (36.614678\du,34.909198\du)--(36.775458\du,35.069978\du);
\pgfsetbuttcap
\pgfsetmiterjoin
\pgfsetdash{}{0pt}
\definecolor{dialinecolor}{rgb}{0.000000, 0.000000, 0.000000}
\pgfsetstrokecolor{dialinecolor}
\draw (37.418576\du,34.748419\du)--(37.257797\du,34.909198\du);
\pgfsetbuttcap
\pgfsetmiterjoin
\pgfsetdash{}{0pt}
\definecolor{dialinecolor}{rgb}{0.000000, 0.000000, 0.000000}
\pgfsetstrokecolor{dialinecolor}
\draw (37.257797\du,34.909198\du)--(37.418576\du,35.069978\du);
\pgfsetbuttcap
\pgfsetmiterjoin
\pgfsetdash{}{0pt}
\definecolor{dialinecolor}{rgb}{0.000000, 0.000000, 0.000000}
\pgfsetstrokecolor{dialinecolor}
\draw (38.061695\du,34.748419\du)--(37.900915\du,34.909198\du);
\pgfsetbuttcap
\pgfsetmiterjoin
\pgfsetdash{}{0pt}
\definecolor{dialinecolor}{rgb}{0.000000, 0.000000, 0.000000}
\pgfsetstrokecolor{dialinecolor}
\draw (37.900915\du,34.909198\du)--(38.061695\du,35.069978\du);
\pgfsetbuttcap
\pgfsetmiterjoin
\pgfsetdash{}{0pt}
\definecolor{dialinecolor}{rgb}{0.000000, 0.000000, 0.000000}
\pgfsetstrokecolor{dialinecolor}
\draw (38.704814\du,34.748419\du)--(38.544034\du,34.909198\du);
\pgfsetbuttcap
\pgfsetmiterjoin
\pgfsetdash{}{0pt}
\definecolor{dialinecolor}{rgb}{0.000000, 0.000000, 0.000000}
\pgfsetstrokecolor{dialinecolor}
\draw (38.544034\du,34.909198\du)--(38.704814\du,35.069978\du);
\pgfsetbuttcap
\pgfsetmiterjoin
\pgfsetdash{}{0pt}
\definecolor{dialinecolor}{rgb}{0.000000, 0.000000, 0.000000}
\pgfsetstrokecolor{dialinecolor}
\draw (39.347932\du,34.748419\du)--(39.187153\du,34.909198\du);
\pgfsetbuttcap
\pgfsetmiterjoin
\pgfsetdash{}{0pt}
\definecolor{dialinecolor}{rgb}{0.000000, 0.000000, 0.000000}
\pgfsetstrokecolor{dialinecolor}
\draw (39.187153\du,34.909198\du)--(39.347932\du,35.069978\du);
\pgfsetbuttcap
\pgfsetmiterjoin
\pgfsetdash{}{0pt}
\definecolor{dialinecolor}{rgb}{0.000000, 0.000000, 0.000000}
\pgfsetstrokecolor{dialinecolor}
\draw (39.991051\du,34.748419\du)--(39.830271\du,34.909198\du);
\pgfsetbuttcap
\pgfsetmiterjoin
\pgfsetdash{}{0pt}
\definecolor{dialinecolor}{rgb}{0.000000, 0.000000, 0.000000}
\pgfsetstrokecolor{dialinecolor}
\draw (39.830271\du,34.909198\du)--(39.991051\du,35.069978\du);
\pgfsetbuttcap
\pgfsetmiterjoin
\pgfsetdash{}{0pt}
\definecolor{dialinecolor}{rgb}{0.000000, 0.000000, 0.000000}
\pgfsetstrokecolor{dialinecolor}
\draw (40.634170\du,34.748419\du)--(40.473390\du,34.909198\du);
\pgfsetbuttcap
\pgfsetmiterjoin
\pgfsetdash{}{0pt}
\definecolor{dialinecolor}{rgb}{0.000000, 0.000000, 0.000000}
\pgfsetstrokecolor{dialinecolor}
\draw (40.473390\du,34.909198\du)--(40.634170\du,35.069978\du);
\pgfsetbuttcap
\pgfsetmiterjoin
\pgfsetdash{}{0pt}
\definecolor{dialinecolor}{rgb}{0.000000, 0.000000, 0.000000}
\pgfsetstrokecolor{dialinecolor}
\draw (41.277288\du,34.748419\du)--(41.116509\du,34.909198\du);
\pgfsetbuttcap
\pgfsetmiterjoin
\pgfsetdash{}{0pt}
\definecolor{dialinecolor}{rgb}{0.000000, 0.000000, 0.000000}
\pgfsetstrokecolor{dialinecolor}
\draw (41.116509\du,34.909198\du)--(41.277288\du,35.069978\du);
\pgfsetbuttcap
\pgfsetmiterjoin
\pgfsetdash{}{0pt}
\definecolor{dialinecolor}{rgb}{0.000000, 0.000000, 0.000000}
\pgfsetstrokecolor{dialinecolor}
\draw (41.598847\du,35.230758\du)--(41.438068\du,35.391537\du);
\pgfsetbuttcap
\pgfsetmiterjoin
\pgfsetdash{}{0pt}
\definecolor{dialinecolor}{rgb}{0.000000, 0.000000, 0.000000}
\pgfsetstrokecolor{dialinecolor}
\draw (41.438068\du,35.391537\du)--(41.598847\du,35.552317\du);
\pgfsetbuttcap
\pgfsetmiterjoin
\pgfsetdash{}{0pt}
\definecolor{dialinecolor}{rgb}{0.000000, 0.000000, 0.000000}
\pgfsetstrokecolor{dialinecolor}
\draw (40.955729\du,35.230758\du)--(40.794949\du,35.391537\du);
\pgfsetbuttcap
\pgfsetmiterjoin
\pgfsetdash{}{0pt}
\definecolor{dialinecolor}{rgb}{0.000000, 0.000000, 0.000000}
\pgfsetstrokecolor{dialinecolor}
\draw (40.794949\du,35.391537\du)--(40.955729\du,35.552317\du);
\pgfsetbuttcap
\pgfsetmiterjoin
\pgfsetdash{}{0pt}
\definecolor{dialinecolor}{rgb}{0.000000, 0.000000, 0.000000}
\pgfsetstrokecolor{dialinecolor}
\draw (40.312610\du,35.230758\du)--(40.151831\du,35.391537\du);
\pgfsetbuttcap
\pgfsetmiterjoin
\pgfsetdash{}{0pt}
\definecolor{dialinecolor}{rgb}{0.000000, 0.000000, 0.000000}
\pgfsetstrokecolor{dialinecolor}
\draw (40.151831\du,35.391537\du)--(40.312610\du,35.552317\du);
\pgfsetbuttcap
\pgfsetmiterjoin
\pgfsetdash{}{0pt}
\definecolor{dialinecolor}{rgb}{0.000000, 0.000000, 0.000000}
\pgfsetstrokecolor{dialinecolor}
\draw (39.669492\du,35.230758\du)--(39.508712\du,35.391537\du);
\pgfsetbuttcap
\pgfsetmiterjoin
\pgfsetdash{}{0pt}
\definecolor{dialinecolor}{rgb}{0.000000, 0.000000, 0.000000}
\pgfsetstrokecolor{dialinecolor}
\draw (39.508712\du,35.391537\du)--(39.669492\du,35.552317\du);
\pgfsetbuttcap
\pgfsetmiterjoin
\pgfsetdash{}{0pt}
\definecolor{dialinecolor}{rgb}{0.000000, 0.000000, 0.000000}
\pgfsetstrokecolor{dialinecolor}
\draw (39.026373\du,35.230758\du)--(38.865593\du,35.391537\du);
\pgfsetbuttcap
\pgfsetmiterjoin
\pgfsetdash{}{0pt}
\definecolor{dialinecolor}{rgb}{0.000000, 0.000000, 0.000000}
\pgfsetstrokecolor{dialinecolor}
\draw (38.865593\du,35.391537\du)--(39.026373\du,35.552317\du);
\pgfsetbuttcap
\pgfsetmiterjoin
\pgfsetdash{}{0pt}
\definecolor{dialinecolor}{rgb}{0.000000, 0.000000, 0.000000}
\pgfsetstrokecolor{dialinecolor}
\draw (38.383254\du,35.230758\du)--(38.222475\du,35.391537\du);
\pgfsetbuttcap
\pgfsetmiterjoin
\pgfsetdash{}{0pt}
\definecolor{dialinecolor}{rgb}{0.000000, 0.000000, 0.000000}
\pgfsetstrokecolor{dialinecolor}
\draw (38.222475\du,35.391537\du)--(38.383254\du,35.552317\du);
\pgfsetbuttcap
\pgfsetmiterjoin
\pgfsetdash{}{0pt}
\definecolor{dialinecolor}{rgb}{0.000000, 0.000000, 0.000000}
\pgfsetstrokecolor{dialinecolor}
\draw (37.740136\du,35.230758\du)--(37.579356\du,35.391537\du);
\pgfsetbuttcap
\pgfsetmiterjoin
\pgfsetdash{}{0pt}
\definecolor{dialinecolor}{rgb}{0.000000, 0.000000, 0.000000}
\pgfsetstrokecolor{dialinecolor}
\draw (37.579356\du,35.391537\du)--(37.740136\du,35.552317\du);
\pgfsetlinewidth{0.100000\du}
\pgfsetbuttcap
\pgfsetmiterjoin
\pgfsetdash{}{0pt}
\definecolor{dialinecolor}{rgb}{0.000000, 0.000000, 0.000000}
\pgfsetstrokecolor{dialinecolor}
\draw (36.775458\du,35.391537\du)--(36.775458\du,35.552317\du)--(37.418576\du,35.552317\du)--(37.418576\du,35.391537\du)--cycle;
\pgfsetlinewidth{0.010000\du}
\pgfsetbuttcap
\pgfsetmiterjoin
\pgfsetdash{}{0pt}
\definecolor{dialinecolor}{rgb}{0.000000, 0.000000, 0.000000}
\pgfsetstrokecolor{dialinecolor}
\draw (36.775458\du,35.391537\du)--(36.775458\du,35.552317\du)--(37.418576\du,35.552317\du)--(37.418576\du,35.391537\du)--cycle;
\pgfsetlinewidth{0.100000\du}
\pgfsetbuttcap
\pgfsetmiterjoin
\pgfsetdash{}{0pt}
\definecolor{dialinecolor}{rgb}{0.000000, 0.000000, 0.000000}
\pgfsetstrokecolor{dialinecolor}
\draw (36.132339\du,34.909198\du)--(36.453898\du,35.230758\du)--(36.132339\du,35.552317\du)--(35.810780\du,35.230758\du)--cycle;
\pgfsetlinewidth{0.010000\du}
\pgfsetbuttcap
\pgfsetmiterjoin
\pgfsetdash{}{0pt}
\definecolor{dialinecolor}{rgb}{0.000000, 0.000000, 0.000000}
\pgfsetstrokecolor{dialinecolor}
\draw (36.132339\du,34.909198\du)--(36.453898\du,35.230758\du)--(36.132339\du,35.552317\du)--(35.810780\du,35.230758\du)--cycle;
\pgfsetlinewidth{0.100000\du}
\pgfsetbuttcap
\pgfsetmiterjoin
\pgfsetdash{}{0pt}
\definecolor{dialinecolor}{rgb}{0.000000, 0.000000, 0.000000}
\pgfsetstrokecolor{dialinecolor}
\draw (35.650000\du,34.426859\du)--(35.650000\du,34.587639\du)--(42.081186\du,34.587639\du)--(42.081186\du,34.426859\du)--cycle;
\pgfsetlinewidth{0.010000\du}
\pgfsetbuttcap
\pgfsetmiterjoin
\pgfsetdash{}{0pt}
\definecolor{dialinecolor}{rgb}{0.000000, 0.000000, 0.000000}
\pgfsetstrokecolor{dialinecolor}
\draw (35.650000\du,34.426859\du)--(35.650000\du,34.587639\du)--(42.081186\du,34.587639\du)--(42.081186\du,34.426859\du)--cycle;
\pgfsetlinewidth{0.100000\du}
\pgfsetbuttcap
\pgfsetmiterjoin
\pgfsetdash{}{0pt}
\definecolor{dialinecolor}{rgb}{0.000000, 0.000000, 0.000000}
\pgfsetstrokecolor{dialinecolor}
\draw (36.453898\du,34.426859\du)--(36.453898\du,34.587639\du)--(37.900915\du,34.587639\du)--(37.900915\du,34.426859\du)--cycle;
\pgfsetlinewidth{0.010000\du}
\pgfsetbuttcap
\pgfsetmiterjoin
\pgfsetdash{}{0pt}
\definecolor{dialinecolor}{rgb}{0.000000, 0.000000, 0.000000}
\pgfsetstrokecolor{dialinecolor}
\draw (36.453898\du,34.426859\du)--(36.453898\du,34.587639\du)--(37.900915\du,34.587639\du)--(37.900915\du,34.426859\du)--cycle;
% setfont left to latex
\definecolor{dialinecolor}{rgb}{0.000000, 0.000000, 0.000000}
\pgfsetstrokecolor{dialinecolor}
\node at (39.078800\du,36.378000\du){IBM RackServer};
% setfont left to latex
\definecolor{dialinecolor}{rgb}{0.000000, 0.000000, 0.000000}
\pgfsetstrokecolor{dialinecolor}
\node at (39.078800\du,37.178000\du){VMware ESXi};
\pgfsetlinewidth{0.100000\du}
\pgfsetdash{}{0pt}
\pgfsetdash{}{0pt}
\pgfsetbuttcap
{
\definecolor{dialinecolor}{rgb}{0.000000, 0.000000, 0.000000}
\pgfsetfillcolor{dialinecolor}
% was here!!!
\pgfsetarrowsstart{latex}
\pgfsetarrowsend{latex}
\definecolor{dialinecolor}{rgb}{0.000000, 0.000000, 0.000000}
\pgfsetstrokecolor{dialinecolor}
\draw (38.847674\du,30.624686\du)--(38.865593\du,34.105300\du);
}
% setfont left to latex
\definecolor{dialinecolor}{rgb}{0.000000, 0.000000, 0.000000}
\pgfsetstrokecolor{dialinecolor}
\node[anchor=west] at (38.959000\du,32.514900\du){Ethernet};
\end{tikzpicture}

\caption[Aufbau Geräte des Testsystems]{Aufbau Geräte des Testsystems}
\label{fig:physAufb}
\end{figure}
\begin{figure}[h!]
\centering
% Graphic for TeX using PGF
% Title: /home/tboonx/Dokumente/Studium/Masterarbeit/masterthesis/Abbildungen/Testsystem_VMsdia
% Creator: Dia v0.97.2
% CreationDate: Fri Mar 20 18:44:40 2015
% For: tboonx
% \usepackage{tikz}
% The following commands are not supported in PSTricks at present
% We define them conditionally, so when they are implemented,
% this pgf file will use them.
\ifx\du\undefined
  \newlength{\du}
\fi
\setlength{\du}{15\unitlength}
\begin{tikzpicture}
\pgftransformxscale{1.000000}
\pgftransformyscale{-1.000000}
\definecolor{dialinecolor}{rgb}{0.000000, 0.000000, 0.000000}
\pgfsetstrokecolor{dialinecolor}
\definecolor{dialinecolor}{rgb}{1.000000, 1.000000, 1.000000}
\pgfsetfillcolor{dialinecolor}
\pgfsetlinewidth{0.100000\du}
\pgfsetdash{}{0pt}
\pgfsetdash{}{0pt}
\pgfsetmiterjoin
\pgfsetbuttcap
\definecolor{dialinecolor}{rgb}{0.000000, 0.000000, 0.000000}
\pgfsetstrokecolor{dialinecolor}
\draw (38.047610\du,24.381985\du)--(45.825891\du,24.339940\du)--(45.867936\du,38.635129\du)--(30.900000\du,38.635129\du)--(30.900000\du,28.502369\du)--(38.047610\du,28.460329\du)--cycle;
% setfont left to latex
\definecolor{dialinecolor}{rgb}{0.000000, 0.000000, 0.000000}
\pgfsetstrokecolor{dialinecolor}
\node[anchor=west] at (31.037500\du,24.925000\du){je};
% setfont left to latex
\definecolor{dialinecolor}{rgb}{0.000000, 0.000000, 0.000000}
\pgfsetstrokecolor{dialinecolor}
\node[anchor=west] at (31.037500\du,25.986156\du){    2 Kerne};
% setfont left to latex
\definecolor{dialinecolor}{rgb}{0.000000, 0.000000, 0.000000}
\pgfsetstrokecolor{dialinecolor}
\node[anchor=west] at (31.037500\du,27.047311\du){    3GB RAM};
% setfont left to latex
\definecolor{dialinecolor}{rgb}{0.000000, 0.000000, 0.000000}
\pgfsetstrokecolor{dialinecolor}
\node[anchor=west] at (31.037500\du,28.108467\du){    100GB};
\pgfsetlinewidth{0.100000\du}
\pgfsetdash{}{0pt}
\pgfsetdash{}{0pt}
\pgfsetbuttcap
\pgfsetmiterjoin
\pgfsetlinewidth{0.100000\du}
\pgfsetbuttcap
\pgfsetmiterjoin
\pgfsetdash{}{0pt}
\definecolor{dialinecolor}{rgb}{0.850980, 0.850980, 0.803922}
\pgfsetfillcolor{dialinecolor}
\fill (39.874800\du,25.769189\du)--(39.874800\du,26.675004\du)--(43.498061\du,26.675004\du)--(43.498061\du,25.769189\du)--cycle;
\definecolor{dialinecolor}{rgb}{0.000000, 0.000000, 0.000000}
\pgfsetstrokecolor{dialinecolor}
\draw (39.874800\du,25.769189\du)--(39.874800\du,26.675004\du)--(43.498061\du,26.675004\du)--(43.498061\du,25.769189\du)--cycle;
\pgfsetlinewidth{0.010000\du}
\pgfsetbuttcap
\pgfsetmiterjoin
\pgfsetdash{}{0pt}
\definecolor{dialinecolor}{rgb}{0.000000, 0.000000, 0.000000}
\pgfsetstrokecolor{dialinecolor}
\draw (39.874800\du,25.769189\du)--(39.874800\du,26.675004\du)--(43.498061\du,26.675004\du)--(43.498061\du,25.769189\du)--cycle;
\pgfsetbuttcap
\pgfsetmiterjoin
\pgfsetdash{}{0pt}
\definecolor{dialinecolor}{rgb}{0.000000, 0.000000, 0.000000}
\pgfsetstrokecolor{dialinecolor}
\draw (40.508871\du,26.131515\du)--(40.418289\du,26.222097\du);
\pgfsetbuttcap
\pgfsetmiterjoin
\pgfsetdash{}{0pt}
\definecolor{dialinecolor}{rgb}{0.000000, 0.000000, 0.000000}
\pgfsetstrokecolor{dialinecolor}
\draw (40.418289\du,26.222097\du)--(40.508871\du,26.312678\du);
\pgfsetbuttcap
\pgfsetmiterjoin
\pgfsetdash{}{0pt}
\definecolor{dialinecolor}{rgb}{0.000000, 0.000000, 0.000000}
\pgfsetstrokecolor{dialinecolor}
\draw (40.871197\du,26.131515\du)--(40.780615\du,26.222097\du);
\pgfsetbuttcap
\pgfsetmiterjoin
\pgfsetdash{}{0pt}
\definecolor{dialinecolor}{rgb}{0.000000, 0.000000, 0.000000}
\pgfsetstrokecolor{dialinecolor}
\draw (40.780615\du,26.222097\du)--(40.871197\du,26.312678\du);
\pgfsetbuttcap
\pgfsetmiterjoin
\pgfsetdash{}{0pt}
\definecolor{dialinecolor}{rgb}{0.000000, 0.000000, 0.000000}
\pgfsetstrokecolor{dialinecolor}
\draw (41.233523\du,26.131515\du)--(41.142941\du,26.222097\du);
\pgfsetbuttcap
\pgfsetmiterjoin
\pgfsetdash{}{0pt}
\definecolor{dialinecolor}{rgb}{0.000000, 0.000000, 0.000000}
\pgfsetstrokecolor{dialinecolor}
\draw (41.142941\du,26.222097\du)--(41.233523\du,26.312678\du);
\pgfsetbuttcap
\pgfsetmiterjoin
\pgfsetdash{}{0pt}
\definecolor{dialinecolor}{rgb}{0.000000, 0.000000, 0.000000}
\pgfsetstrokecolor{dialinecolor}
\draw (41.595849\du,26.131515\du)--(41.505268\du,26.222097\du);
\pgfsetbuttcap
\pgfsetmiterjoin
\pgfsetdash{}{0pt}
\definecolor{dialinecolor}{rgb}{0.000000, 0.000000, 0.000000}
\pgfsetstrokecolor{dialinecolor}
\draw (41.505268\du,26.222097\du)--(41.595849\du,26.312678\du);
\pgfsetbuttcap
\pgfsetmiterjoin
\pgfsetdash{}{0pt}
\definecolor{dialinecolor}{rgb}{0.000000, 0.000000, 0.000000}
\pgfsetstrokecolor{dialinecolor}
\draw (41.958175\du,26.131515\du)--(41.867594\du,26.222097\du);
\pgfsetbuttcap
\pgfsetmiterjoin
\pgfsetdash{}{0pt}
\definecolor{dialinecolor}{rgb}{0.000000, 0.000000, 0.000000}
\pgfsetstrokecolor{dialinecolor}
\draw (41.867594\du,26.222097\du)--(41.958175\du,26.312678\du);
\pgfsetbuttcap
\pgfsetmiterjoin
\pgfsetdash{}{0pt}
\definecolor{dialinecolor}{rgb}{0.000000, 0.000000, 0.000000}
\pgfsetstrokecolor{dialinecolor}
\draw (42.320501\du,26.131515\du)--(42.229920\du,26.222097\du);
\pgfsetbuttcap
\pgfsetmiterjoin
\pgfsetdash{}{0pt}
\definecolor{dialinecolor}{rgb}{0.000000, 0.000000, 0.000000}
\pgfsetstrokecolor{dialinecolor}
\draw (42.229920\du,26.222097\du)--(42.320501\du,26.312678\du);
\pgfsetbuttcap
\pgfsetmiterjoin
\pgfsetdash{}{0pt}
\definecolor{dialinecolor}{rgb}{0.000000, 0.000000, 0.000000}
\pgfsetstrokecolor{dialinecolor}
\draw (42.682827\du,26.131515\du)--(42.592246\du,26.222097\du);
\pgfsetbuttcap
\pgfsetmiterjoin
\pgfsetdash{}{0pt}
\definecolor{dialinecolor}{rgb}{0.000000, 0.000000, 0.000000}
\pgfsetstrokecolor{dialinecolor}
\draw (42.592246\du,26.222097\du)--(42.682827\du,26.312678\du);
\pgfsetbuttcap
\pgfsetmiterjoin
\pgfsetdash{}{0pt}
\definecolor{dialinecolor}{rgb}{0.000000, 0.000000, 0.000000}
\pgfsetstrokecolor{dialinecolor}
\draw (43.045154\du,26.131515\du)--(42.954572\du,26.222097\du);
\pgfsetbuttcap
\pgfsetmiterjoin
\pgfsetdash{}{0pt}
\definecolor{dialinecolor}{rgb}{0.000000, 0.000000, 0.000000}
\pgfsetstrokecolor{dialinecolor}
\draw (42.954572\du,26.222097\du)--(43.045154\du,26.312678\du);
\pgfsetbuttcap
\pgfsetmiterjoin
\pgfsetdash{}{0pt}
\definecolor{dialinecolor}{rgb}{0.000000, 0.000000, 0.000000}
\pgfsetstrokecolor{dialinecolor}
\draw (43.226317\du,26.403260\du)--(43.135735\du,26.493841\du);
\pgfsetbuttcap
\pgfsetmiterjoin
\pgfsetdash{}{0pt}
\definecolor{dialinecolor}{rgb}{0.000000, 0.000000, 0.000000}
\pgfsetstrokecolor{dialinecolor}
\draw (43.135735\du,26.493841\du)--(43.226317\du,26.584423\du);
\pgfsetbuttcap
\pgfsetmiterjoin
\pgfsetdash{}{0pt}
\definecolor{dialinecolor}{rgb}{0.000000, 0.000000, 0.000000}
\pgfsetstrokecolor{dialinecolor}
\draw (42.863990\du,26.403260\du)--(42.773409\du,26.493841\du);
\pgfsetbuttcap
\pgfsetmiterjoin
\pgfsetdash{}{0pt}
\definecolor{dialinecolor}{rgb}{0.000000, 0.000000, 0.000000}
\pgfsetstrokecolor{dialinecolor}
\draw (42.773409\du,26.493841\du)--(42.863990\du,26.584423\du);
\pgfsetbuttcap
\pgfsetmiterjoin
\pgfsetdash{}{0pt}
\definecolor{dialinecolor}{rgb}{0.000000, 0.000000, 0.000000}
\pgfsetstrokecolor{dialinecolor}
\draw (42.501664\du,26.403260\du)--(42.411083\du,26.493841\du);
\pgfsetbuttcap
\pgfsetmiterjoin
\pgfsetdash{}{0pt}
\definecolor{dialinecolor}{rgb}{0.000000, 0.000000, 0.000000}
\pgfsetstrokecolor{dialinecolor}
\draw (42.411083\du,26.493841\du)--(42.501664\du,26.584423\du);
\pgfsetbuttcap
\pgfsetmiterjoin
\pgfsetdash{}{0pt}
\definecolor{dialinecolor}{rgb}{0.000000, 0.000000, 0.000000}
\pgfsetstrokecolor{dialinecolor}
\draw (42.139338\du,26.403260\du)--(42.048757\du,26.493841\du);
\pgfsetbuttcap
\pgfsetmiterjoin
\pgfsetdash{}{0pt}
\definecolor{dialinecolor}{rgb}{0.000000, 0.000000, 0.000000}
\pgfsetstrokecolor{dialinecolor}
\draw (42.048757\du,26.493841\du)--(42.139338\du,26.584423\du);
\pgfsetbuttcap
\pgfsetmiterjoin
\pgfsetdash{}{0pt}
\definecolor{dialinecolor}{rgb}{0.000000, 0.000000, 0.000000}
\pgfsetstrokecolor{dialinecolor}
\draw (41.777012\du,26.403260\du)--(41.686431\du,26.493841\du);
\pgfsetbuttcap
\pgfsetmiterjoin
\pgfsetdash{}{0pt}
\definecolor{dialinecolor}{rgb}{0.000000, 0.000000, 0.000000}
\pgfsetstrokecolor{dialinecolor}
\draw (41.686431\du,26.493841\du)--(41.777012\du,26.584423\du);
\pgfsetbuttcap
\pgfsetmiterjoin
\pgfsetdash{}{0pt}
\definecolor{dialinecolor}{rgb}{0.000000, 0.000000, 0.000000}
\pgfsetstrokecolor{dialinecolor}
\draw (41.414686\du,26.403260\du)--(41.324104\du,26.493841\du);
\pgfsetbuttcap
\pgfsetmiterjoin
\pgfsetdash{}{0pt}
\definecolor{dialinecolor}{rgb}{0.000000, 0.000000, 0.000000}
\pgfsetstrokecolor{dialinecolor}
\draw (41.324104\du,26.493841\du)--(41.414686\du,26.584423\du);
\pgfsetbuttcap
\pgfsetmiterjoin
\pgfsetdash{}{0pt}
\definecolor{dialinecolor}{rgb}{0.000000, 0.000000, 0.000000}
\pgfsetstrokecolor{dialinecolor}
\draw (41.052360\du,26.403260\du)--(40.961778\du,26.493841\du);
\pgfsetbuttcap
\pgfsetmiterjoin
\pgfsetdash{}{0pt}
\definecolor{dialinecolor}{rgb}{0.000000, 0.000000, 0.000000}
\pgfsetstrokecolor{dialinecolor}
\draw (40.961778\du,26.493841\du)--(41.052360\du,26.584423\du);
\pgfsetlinewidth{0.100000\du}
\pgfsetbuttcap
\pgfsetmiterjoin
\pgfsetdash{}{0pt}
\definecolor{dialinecolor}{rgb}{0.850980, 0.850980, 0.803922}
\pgfsetfillcolor{dialinecolor}
\fill (40.508871\du,26.493841\du)--(40.508871\du,26.584423\du)--(40.871197\du,26.584423\du)--(40.871197\du,26.493841\du)--cycle;
\definecolor{dialinecolor}{rgb}{0.000000, 0.000000, 0.000000}
\pgfsetstrokecolor{dialinecolor}
\draw (40.508871\du,26.493841\du)--(40.508871\du,26.584423\du)--(40.871197\du,26.584423\du)--(40.871197\du,26.493841\du)--cycle;
\pgfsetlinewidth{0.010000\du}
\pgfsetbuttcap
\pgfsetmiterjoin
\pgfsetdash{}{0pt}
\definecolor{dialinecolor}{rgb}{0.000000, 0.000000, 0.000000}
\pgfsetstrokecolor{dialinecolor}
\draw (40.508871\du,26.493841\du)--(40.508871\du,26.584423\du)--(40.871197\du,26.584423\du)--(40.871197\du,26.493841\du)--cycle;
\pgfsetlinewidth{0.100000\du}
\pgfsetbuttcap
\pgfsetmiterjoin
\pgfsetdash{}{0pt}
\definecolor{dialinecolor}{rgb}{0.803922, 0.803922, 0.803922}
\pgfsetfillcolor{dialinecolor}
\fill (40.146545\du,26.222097\du)--(40.327708\du,26.403260\du)--(40.146545\du,26.584423\du)--(39.965382\du,26.403260\du)--cycle;
\definecolor{dialinecolor}{rgb}{0.000000, 0.000000, 0.000000}
\pgfsetstrokecolor{dialinecolor}
\draw (40.146545\du,26.222097\du)--(40.327708\du,26.403260\du)--(40.146545\du,26.584423\du)--(39.965382\du,26.403260\du)--cycle;
\pgfsetlinewidth{0.010000\du}
\pgfsetbuttcap
\pgfsetmiterjoin
\pgfsetdash{}{0pt}
\definecolor{dialinecolor}{rgb}{0.000000, 0.000000, 0.000000}
\pgfsetstrokecolor{dialinecolor}
\draw (40.146545\du,26.222097\du)--(40.327708\du,26.403260\du)--(40.146545\du,26.584423\du)--(39.965382\du,26.403260\du)--cycle;
\pgfsetlinewidth{0.100000\du}
\pgfsetbuttcap
\pgfsetmiterjoin
\pgfsetdash{}{0pt}
\definecolor{dialinecolor}{rgb}{0.850980, 0.850980, 0.803922}
\pgfsetfillcolor{dialinecolor}
\fill (39.874800\du,25.950352\du)--(39.874800\du,26.040934\du)--(43.498061\du,26.040934\du)--(43.498061\du,25.950352\du)--cycle;
\definecolor{dialinecolor}{rgb}{0.000000, 0.000000, 0.000000}
\pgfsetstrokecolor{dialinecolor}
\draw (39.874800\du,25.950352\du)--(39.874800\du,26.040934\du)--(43.498061\du,26.040934\du)--(43.498061\du,25.950352\du)--cycle;
\pgfsetlinewidth{0.010000\du}
\pgfsetbuttcap
\pgfsetmiterjoin
\pgfsetdash{}{0pt}
\definecolor{dialinecolor}{rgb}{0.000000, 0.000000, 0.000000}
\pgfsetstrokecolor{dialinecolor}
\draw (39.874800\du,25.950352\du)--(39.874800\du,26.040934\du)--(43.498061\du,26.040934\du)--(43.498061\du,25.950352\du)--cycle;
\pgfsetlinewidth{0.100000\du}
\pgfsetbuttcap
\pgfsetmiterjoin
\pgfsetdash{}{0pt}
\definecolor{dialinecolor}{rgb}{0.803922, 0.803922, 0.803922}
\pgfsetfillcolor{dialinecolor}
\fill (40.327708\du,25.950352\du)--(40.327708\du,26.040934\du)--(41.142941\du,26.040934\du)--(41.142941\du,25.950352\du)--cycle;
\definecolor{dialinecolor}{rgb}{0.000000, 0.000000, 0.000000}
\pgfsetstrokecolor{dialinecolor}
\draw (40.327708\du,25.950352\du)--(40.327708\du,26.040934\du)--(41.142941\du,26.040934\du)--(41.142941\du,25.950352\du)--cycle;
\pgfsetlinewidth{0.010000\du}
\pgfsetbuttcap
\pgfsetmiterjoin
\pgfsetdash{}{0pt}
\definecolor{dialinecolor}{rgb}{0.000000, 0.000000, 0.000000}
\pgfsetstrokecolor{dialinecolor}
\draw (40.327708\du,25.950352\du)--(40.327708\du,26.040934\du)--(41.142941\du,26.040934\du)--(41.142941\du,25.950352\du)--cycle;
% setfont left to latex
\definecolor{dialinecolor}{rgb}{0.000000, 0.000000, 0.000000}
\pgfsetstrokecolor{dialinecolor}
\node[anchor=west] at (39.826560\du,27.445729\du){GTM};
\pgfsetlinewidth{0.100000\du}
\pgfsetdash{}{0pt}
\pgfsetdash{}{0pt}
\pgfsetbuttcap
\pgfsetmiterjoin
\pgfsetlinewidth{0.100000\du}
\pgfsetbuttcap
\pgfsetmiterjoin
\pgfsetdash{}{0pt}
\definecolor{dialinecolor}{rgb}{0.850980, 0.850980, 0.803922}
\pgfsetfillcolor{dialinecolor}
\fill (31.614800\du,29.464479\du)--(31.614800\du,30.370294\du)--(35.238061\du,30.370294\du)--(35.238061\du,29.464479\du)--cycle;
\definecolor{dialinecolor}{rgb}{0.000000, 0.000000, 0.000000}
\pgfsetstrokecolor{dialinecolor}
\draw (31.614800\du,29.464479\du)--(31.614800\du,30.370294\du)--(35.238061\du,30.370294\du)--(35.238061\du,29.464479\du)--cycle;
\pgfsetlinewidth{0.010000\du}
\pgfsetbuttcap
\pgfsetmiterjoin
\pgfsetdash{}{0pt}
\definecolor{dialinecolor}{rgb}{0.000000, 0.000000, 0.000000}
\pgfsetstrokecolor{dialinecolor}
\draw (31.614800\du,29.464479\du)--(31.614800\du,30.370294\du)--(35.238061\du,30.370294\du)--(35.238061\du,29.464479\du)--cycle;
\pgfsetbuttcap
\pgfsetmiterjoin
\pgfsetdash{}{0pt}
\definecolor{dialinecolor}{rgb}{0.000000, 0.000000, 0.000000}
\pgfsetstrokecolor{dialinecolor}
\draw (32.248871\du,29.826805\du)--(32.158289\du,29.917387\du);
\pgfsetbuttcap
\pgfsetmiterjoin
\pgfsetdash{}{0pt}
\definecolor{dialinecolor}{rgb}{0.000000, 0.000000, 0.000000}
\pgfsetstrokecolor{dialinecolor}
\draw (32.158289\du,29.917387\du)--(32.248871\du,30.007968\du);
\pgfsetbuttcap
\pgfsetmiterjoin
\pgfsetdash{}{0pt}
\definecolor{dialinecolor}{rgb}{0.000000, 0.000000, 0.000000}
\pgfsetstrokecolor{dialinecolor}
\draw (32.611197\du,29.826805\du)--(32.520615\du,29.917387\du);
\pgfsetbuttcap
\pgfsetmiterjoin
\pgfsetdash{}{0pt}
\definecolor{dialinecolor}{rgb}{0.000000, 0.000000, 0.000000}
\pgfsetstrokecolor{dialinecolor}
\draw (32.520615\du,29.917387\du)--(32.611197\du,30.007968\du);
\pgfsetbuttcap
\pgfsetmiterjoin
\pgfsetdash{}{0pt}
\definecolor{dialinecolor}{rgb}{0.000000, 0.000000, 0.000000}
\pgfsetstrokecolor{dialinecolor}
\draw (32.973523\du,29.826805\du)--(32.882941\du,29.917387\du);
\pgfsetbuttcap
\pgfsetmiterjoin
\pgfsetdash{}{0pt}
\definecolor{dialinecolor}{rgb}{0.000000, 0.000000, 0.000000}
\pgfsetstrokecolor{dialinecolor}
\draw (32.882941\du,29.917387\du)--(32.973523\du,30.007968\du);
\pgfsetbuttcap
\pgfsetmiterjoin
\pgfsetdash{}{0pt}
\definecolor{dialinecolor}{rgb}{0.000000, 0.000000, 0.000000}
\pgfsetstrokecolor{dialinecolor}
\draw (33.335849\du,29.826805\du)--(33.245268\du,29.917387\du);
\pgfsetbuttcap
\pgfsetmiterjoin
\pgfsetdash{}{0pt}
\definecolor{dialinecolor}{rgb}{0.000000, 0.000000, 0.000000}
\pgfsetstrokecolor{dialinecolor}
\draw (33.245268\du,29.917387\du)--(33.335849\du,30.007968\du);
\pgfsetbuttcap
\pgfsetmiterjoin
\pgfsetdash{}{0pt}
\definecolor{dialinecolor}{rgb}{0.000000, 0.000000, 0.000000}
\pgfsetstrokecolor{dialinecolor}
\draw (33.698175\du,29.826805\du)--(33.607594\du,29.917387\du);
\pgfsetbuttcap
\pgfsetmiterjoin
\pgfsetdash{}{0pt}
\definecolor{dialinecolor}{rgb}{0.000000, 0.000000, 0.000000}
\pgfsetstrokecolor{dialinecolor}
\draw (33.607594\du,29.917387\du)--(33.698175\du,30.007968\du);
\pgfsetbuttcap
\pgfsetmiterjoin
\pgfsetdash{}{0pt}
\definecolor{dialinecolor}{rgb}{0.000000, 0.000000, 0.000000}
\pgfsetstrokecolor{dialinecolor}
\draw (34.060501\du,29.826805\du)--(33.969920\du,29.917387\du);
\pgfsetbuttcap
\pgfsetmiterjoin
\pgfsetdash{}{0pt}
\definecolor{dialinecolor}{rgb}{0.000000, 0.000000, 0.000000}
\pgfsetstrokecolor{dialinecolor}
\draw (33.969920\du,29.917387\du)--(34.060501\du,30.007968\du);
\pgfsetbuttcap
\pgfsetmiterjoin
\pgfsetdash{}{0pt}
\definecolor{dialinecolor}{rgb}{0.000000, 0.000000, 0.000000}
\pgfsetstrokecolor{dialinecolor}
\draw (34.422827\du,29.826805\du)--(34.332246\du,29.917387\du);
\pgfsetbuttcap
\pgfsetmiterjoin
\pgfsetdash{}{0pt}
\definecolor{dialinecolor}{rgb}{0.000000, 0.000000, 0.000000}
\pgfsetstrokecolor{dialinecolor}
\draw (34.332246\du,29.917387\du)--(34.422827\du,30.007968\du);
\pgfsetbuttcap
\pgfsetmiterjoin
\pgfsetdash{}{0pt}
\definecolor{dialinecolor}{rgb}{0.000000, 0.000000, 0.000000}
\pgfsetstrokecolor{dialinecolor}
\draw (34.785154\du,29.826805\du)--(34.694572\du,29.917387\du);
\pgfsetbuttcap
\pgfsetmiterjoin
\pgfsetdash{}{0pt}
\definecolor{dialinecolor}{rgb}{0.000000, 0.000000, 0.000000}
\pgfsetstrokecolor{dialinecolor}
\draw (34.694572\du,29.917387\du)--(34.785154\du,30.007968\du);
\pgfsetbuttcap
\pgfsetmiterjoin
\pgfsetdash{}{0pt}
\definecolor{dialinecolor}{rgb}{0.000000, 0.000000, 0.000000}
\pgfsetstrokecolor{dialinecolor}
\draw (34.966317\du,30.098550\du)--(34.875735\du,30.189131\du);
\pgfsetbuttcap
\pgfsetmiterjoin
\pgfsetdash{}{0pt}
\definecolor{dialinecolor}{rgb}{0.000000, 0.000000, 0.000000}
\pgfsetstrokecolor{dialinecolor}
\draw (34.875735\du,30.189131\du)--(34.966317\du,30.279713\du);
\pgfsetbuttcap
\pgfsetmiterjoin
\pgfsetdash{}{0pt}
\definecolor{dialinecolor}{rgb}{0.000000, 0.000000, 0.000000}
\pgfsetstrokecolor{dialinecolor}
\draw (34.603990\du,30.098550\du)--(34.513409\du,30.189131\du);
\pgfsetbuttcap
\pgfsetmiterjoin
\pgfsetdash{}{0pt}
\definecolor{dialinecolor}{rgb}{0.000000, 0.000000, 0.000000}
\pgfsetstrokecolor{dialinecolor}
\draw (34.513409\du,30.189131\du)--(34.603990\du,30.279713\du);
\pgfsetbuttcap
\pgfsetmiterjoin
\pgfsetdash{}{0pt}
\definecolor{dialinecolor}{rgb}{0.000000, 0.000000, 0.000000}
\pgfsetstrokecolor{dialinecolor}
\draw (34.241664\du,30.098550\du)--(34.151083\du,30.189131\du);
\pgfsetbuttcap
\pgfsetmiterjoin
\pgfsetdash{}{0pt}
\definecolor{dialinecolor}{rgb}{0.000000, 0.000000, 0.000000}
\pgfsetstrokecolor{dialinecolor}
\draw (34.151083\du,30.189131\du)--(34.241664\du,30.279713\du);
\pgfsetbuttcap
\pgfsetmiterjoin
\pgfsetdash{}{0pt}
\definecolor{dialinecolor}{rgb}{0.000000, 0.000000, 0.000000}
\pgfsetstrokecolor{dialinecolor}
\draw (33.879338\du,30.098550\du)--(33.788757\du,30.189131\du);
\pgfsetbuttcap
\pgfsetmiterjoin
\pgfsetdash{}{0pt}
\definecolor{dialinecolor}{rgb}{0.000000, 0.000000, 0.000000}
\pgfsetstrokecolor{dialinecolor}
\draw (33.788757\du,30.189131\du)--(33.879338\du,30.279713\du);
\pgfsetbuttcap
\pgfsetmiterjoin
\pgfsetdash{}{0pt}
\definecolor{dialinecolor}{rgb}{0.000000, 0.000000, 0.000000}
\pgfsetstrokecolor{dialinecolor}
\draw (33.517012\du,30.098550\du)--(33.426431\du,30.189131\du);
\pgfsetbuttcap
\pgfsetmiterjoin
\pgfsetdash{}{0pt}
\definecolor{dialinecolor}{rgb}{0.000000, 0.000000, 0.000000}
\pgfsetstrokecolor{dialinecolor}
\draw (33.426431\du,30.189131\du)--(33.517012\du,30.279713\du);
\pgfsetbuttcap
\pgfsetmiterjoin
\pgfsetdash{}{0pt}
\definecolor{dialinecolor}{rgb}{0.000000, 0.000000, 0.000000}
\pgfsetstrokecolor{dialinecolor}
\draw (33.154686\du,30.098550\du)--(33.064104\du,30.189131\du);
\pgfsetbuttcap
\pgfsetmiterjoin
\pgfsetdash{}{0pt}
\definecolor{dialinecolor}{rgb}{0.000000, 0.000000, 0.000000}
\pgfsetstrokecolor{dialinecolor}
\draw (33.064104\du,30.189131\du)--(33.154686\du,30.279713\du);
\pgfsetbuttcap
\pgfsetmiterjoin
\pgfsetdash{}{0pt}
\definecolor{dialinecolor}{rgb}{0.000000, 0.000000, 0.000000}
\pgfsetstrokecolor{dialinecolor}
\draw (32.792360\du,30.098550\du)--(32.701778\du,30.189131\du);
\pgfsetbuttcap
\pgfsetmiterjoin
\pgfsetdash{}{0pt}
\definecolor{dialinecolor}{rgb}{0.000000, 0.000000, 0.000000}
\pgfsetstrokecolor{dialinecolor}
\draw (32.701778\du,30.189131\du)--(32.792360\du,30.279713\du);
\pgfsetlinewidth{0.100000\du}
\pgfsetbuttcap
\pgfsetmiterjoin
\pgfsetdash{}{0pt}
\definecolor{dialinecolor}{rgb}{0.850980, 0.850980, 0.803922}
\pgfsetfillcolor{dialinecolor}
\fill (32.248871\du,30.189131\du)--(32.248871\du,30.279713\du)--(32.611197\du,30.279713\du)--(32.611197\du,30.189131\du)--cycle;
\definecolor{dialinecolor}{rgb}{0.000000, 0.000000, 0.000000}
\pgfsetstrokecolor{dialinecolor}
\draw (32.248871\du,30.189131\du)--(32.248871\du,30.279713\du)--(32.611197\du,30.279713\du)--(32.611197\du,30.189131\du)--cycle;
\pgfsetlinewidth{0.010000\du}
\pgfsetbuttcap
\pgfsetmiterjoin
\pgfsetdash{}{0pt}
\definecolor{dialinecolor}{rgb}{0.000000, 0.000000, 0.000000}
\pgfsetstrokecolor{dialinecolor}
\draw (32.248871\du,30.189131\du)--(32.248871\du,30.279713\du)--(32.611197\du,30.279713\du)--(32.611197\du,30.189131\du)--cycle;
\pgfsetlinewidth{0.100000\du}
\pgfsetbuttcap
\pgfsetmiterjoin
\pgfsetdash{}{0pt}
\definecolor{dialinecolor}{rgb}{0.803922, 0.803922, 0.803922}
\pgfsetfillcolor{dialinecolor}
\fill (31.886545\du,29.917387\du)--(32.067708\du,30.098550\du)--(31.886545\du,30.279713\du)--(31.705382\du,30.098550\du)--cycle;
\definecolor{dialinecolor}{rgb}{0.000000, 0.000000, 0.000000}
\pgfsetstrokecolor{dialinecolor}
\draw (31.886545\du,29.917387\du)--(32.067708\du,30.098550\du)--(31.886545\du,30.279713\du)--(31.705382\du,30.098550\du)--cycle;
\pgfsetlinewidth{0.010000\du}
\pgfsetbuttcap
\pgfsetmiterjoin
\pgfsetdash{}{0pt}
\definecolor{dialinecolor}{rgb}{0.000000, 0.000000, 0.000000}
\pgfsetstrokecolor{dialinecolor}
\draw (31.886545\du,29.917387\du)--(32.067708\du,30.098550\du)--(31.886545\du,30.279713\du)--(31.705382\du,30.098550\du)--cycle;
\pgfsetlinewidth{0.100000\du}
\pgfsetbuttcap
\pgfsetmiterjoin
\pgfsetdash{}{0pt}
\definecolor{dialinecolor}{rgb}{0.850980, 0.850980, 0.803922}
\pgfsetfillcolor{dialinecolor}
\fill (31.614800\du,29.645642\du)--(31.614800\du,29.736224\du)--(35.238061\du,29.736224\du)--(35.238061\du,29.645642\du)--cycle;
\definecolor{dialinecolor}{rgb}{0.000000, 0.000000, 0.000000}
\pgfsetstrokecolor{dialinecolor}
\draw (31.614800\du,29.645642\du)--(31.614800\du,29.736224\du)--(35.238061\du,29.736224\du)--(35.238061\du,29.645642\du)--cycle;
\pgfsetlinewidth{0.010000\du}
\pgfsetbuttcap
\pgfsetmiterjoin
\pgfsetdash{}{0pt}
\definecolor{dialinecolor}{rgb}{0.000000, 0.000000, 0.000000}
\pgfsetstrokecolor{dialinecolor}
\draw (31.614800\du,29.645642\du)--(31.614800\du,29.736224\du)--(35.238061\du,29.736224\du)--(35.238061\du,29.645642\du)--cycle;
\pgfsetlinewidth{0.100000\du}
\pgfsetbuttcap
\pgfsetmiterjoin
\pgfsetdash{}{0pt}
\definecolor{dialinecolor}{rgb}{0.803922, 0.803922, 0.803922}
\pgfsetfillcolor{dialinecolor}
\fill (32.067708\du,29.645642\du)--(32.067708\du,29.736224\du)--(32.882941\du,29.736224\du)--(32.882941\du,29.645642\du)--cycle;
\definecolor{dialinecolor}{rgb}{0.000000, 0.000000, 0.000000}
\pgfsetstrokecolor{dialinecolor}
\draw (32.067708\du,29.645642\du)--(32.067708\du,29.736224\du)--(32.882941\du,29.736224\du)--(32.882941\du,29.645642\du)--cycle;
\pgfsetlinewidth{0.010000\du}
\pgfsetbuttcap
\pgfsetmiterjoin
\pgfsetdash{}{0pt}
\definecolor{dialinecolor}{rgb}{0.000000, 0.000000, 0.000000}
\pgfsetstrokecolor{dialinecolor}
\draw (32.067708\du,29.645642\du)--(32.067708\du,29.736224\du)--(32.882941\du,29.736224\du)--(32.882941\du,29.645642\du)--cycle;
\pgfsetlinewidth{0.100000\du}
\pgfsetdash{}{0pt}
\pgfsetdash{}{0pt}
\pgfsetbuttcap
\pgfsetmiterjoin
\pgfsetlinewidth{0.100000\du}
\pgfsetbuttcap
\pgfsetmiterjoin
\pgfsetdash{}{0pt}
\definecolor{dialinecolor}{rgb}{0.850980, 0.850980, 0.803922}
\pgfsetfillcolor{dialinecolor}
\fill (36.449910\du,29.451859\du)--(36.449910\du,30.357674\du)--(40.073171\du,30.357674\du)--(40.073171\du,29.451859\du)--cycle;
\definecolor{dialinecolor}{rgb}{0.000000, 0.000000, 0.000000}
\pgfsetstrokecolor{dialinecolor}
\draw (36.449910\du,29.451859\du)--(36.449910\du,30.357674\du)--(40.073171\du,30.357674\du)--(40.073171\du,29.451859\du)--cycle;
\pgfsetlinewidth{0.010000\du}
\pgfsetbuttcap
\pgfsetmiterjoin
\pgfsetdash{}{0pt}
\definecolor{dialinecolor}{rgb}{0.000000, 0.000000, 0.000000}
\pgfsetstrokecolor{dialinecolor}
\draw (36.449910\du,29.451859\du)--(36.449910\du,30.357674\du)--(40.073171\du,30.357674\du)--(40.073171\du,29.451859\du)--cycle;
\pgfsetbuttcap
\pgfsetmiterjoin
\pgfsetdash{}{0pt}
\definecolor{dialinecolor}{rgb}{0.000000, 0.000000, 0.000000}
\pgfsetstrokecolor{dialinecolor}
\draw (37.083981\du,29.814185\du)--(36.993399\du,29.904767\du);
\pgfsetbuttcap
\pgfsetmiterjoin
\pgfsetdash{}{0pt}
\definecolor{dialinecolor}{rgb}{0.000000, 0.000000, 0.000000}
\pgfsetstrokecolor{dialinecolor}
\draw (36.993399\du,29.904767\du)--(37.083981\du,29.995348\du);
\pgfsetbuttcap
\pgfsetmiterjoin
\pgfsetdash{}{0pt}
\definecolor{dialinecolor}{rgb}{0.000000, 0.000000, 0.000000}
\pgfsetstrokecolor{dialinecolor}
\draw (37.446307\du,29.814185\du)--(37.355725\du,29.904767\du);
\pgfsetbuttcap
\pgfsetmiterjoin
\pgfsetdash{}{0pt}
\definecolor{dialinecolor}{rgb}{0.000000, 0.000000, 0.000000}
\pgfsetstrokecolor{dialinecolor}
\draw (37.355725\du,29.904767\du)--(37.446307\du,29.995348\du);
\pgfsetbuttcap
\pgfsetmiterjoin
\pgfsetdash{}{0pt}
\definecolor{dialinecolor}{rgb}{0.000000, 0.000000, 0.000000}
\pgfsetstrokecolor{dialinecolor}
\draw (37.808633\du,29.814185\du)--(37.718051\du,29.904767\du);
\pgfsetbuttcap
\pgfsetmiterjoin
\pgfsetdash{}{0pt}
\definecolor{dialinecolor}{rgb}{0.000000, 0.000000, 0.000000}
\pgfsetstrokecolor{dialinecolor}
\draw (37.718051\du,29.904767\du)--(37.808633\du,29.995348\du);
\pgfsetbuttcap
\pgfsetmiterjoin
\pgfsetdash{}{0pt}
\definecolor{dialinecolor}{rgb}{0.000000, 0.000000, 0.000000}
\pgfsetstrokecolor{dialinecolor}
\draw (38.170959\du,29.814185\du)--(38.080378\du,29.904767\du);
\pgfsetbuttcap
\pgfsetmiterjoin
\pgfsetdash{}{0pt}
\definecolor{dialinecolor}{rgb}{0.000000, 0.000000, 0.000000}
\pgfsetstrokecolor{dialinecolor}
\draw (38.080378\du,29.904767\du)--(38.170959\du,29.995348\du);
\pgfsetbuttcap
\pgfsetmiterjoin
\pgfsetdash{}{0pt}
\definecolor{dialinecolor}{rgb}{0.000000, 0.000000, 0.000000}
\pgfsetstrokecolor{dialinecolor}
\draw (38.533285\du,29.814185\du)--(38.442704\du,29.904767\du);
\pgfsetbuttcap
\pgfsetmiterjoin
\pgfsetdash{}{0pt}
\definecolor{dialinecolor}{rgb}{0.000000, 0.000000, 0.000000}
\pgfsetstrokecolor{dialinecolor}
\draw (38.442704\du,29.904767\du)--(38.533285\du,29.995348\du);
\pgfsetbuttcap
\pgfsetmiterjoin
\pgfsetdash{}{0pt}
\definecolor{dialinecolor}{rgb}{0.000000, 0.000000, 0.000000}
\pgfsetstrokecolor{dialinecolor}
\draw (38.895611\du,29.814185\du)--(38.805030\du,29.904767\du);
\pgfsetbuttcap
\pgfsetmiterjoin
\pgfsetdash{}{0pt}
\definecolor{dialinecolor}{rgb}{0.000000, 0.000000, 0.000000}
\pgfsetstrokecolor{dialinecolor}
\draw (38.805030\du,29.904767\du)--(38.895611\du,29.995348\du);
\pgfsetbuttcap
\pgfsetmiterjoin
\pgfsetdash{}{0pt}
\definecolor{dialinecolor}{rgb}{0.000000, 0.000000, 0.000000}
\pgfsetstrokecolor{dialinecolor}
\draw (39.257937\du,29.814185\du)--(39.167356\du,29.904767\du);
\pgfsetbuttcap
\pgfsetmiterjoin
\pgfsetdash{}{0pt}
\definecolor{dialinecolor}{rgb}{0.000000, 0.000000, 0.000000}
\pgfsetstrokecolor{dialinecolor}
\draw (39.167356\du,29.904767\du)--(39.257937\du,29.995348\du);
\pgfsetbuttcap
\pgfsetmiterjoin
\pgfsetdash{}{0pt}
\definecolor{dialinecolor}{rgb}{0.000000, 0.000000, 0.000000}
\pgfsetstrokecolor{dialinecolor}
\draw (39.620264\du,29.814185\du)--(39.529682\du,29.904767\du);
\pgfsetbuttcap
\pgfsetmiterjoin
\pgfsetdash{}{0pt}
\definecolor{dialinecolor}{rgb}{0.000000, 0.000000, 0.000000}
\pgfsetstrokecolor{dialinecolor}
\draw (39.529682\du,29.904767\du)--(39.620264\du,29.995348\du);
\pgfsetbuttcap
\pgfsetmiterjoin
\pgfsetdash{}{0pt}
\definecolor{dialinecolor}{rgb}{0.000000, 0.000000, 0.000000}
\pgfsetstrokecolor{dialinecolor}
\draw (39.801427\du,30.085930\du)--(39.710845\du,30.176511\du);
\pgfsetbuttcap
\pgfsetmiterjoin
\pgfsetdash{}{0pt}
\definecolor{dialinecolor}{rgb}{0.000000, 0.000000, 0.000000}
\pgfsetstrokecolor{dialinecolor}
\draw (39.710845\du,30.176511\du)--(39.801427\du,30.267093\du);
\pgfsetbuttcap
\pgfsetmiterjoin
\pgfsetdash{}{0pt}
\definecolor{dialinecolor}{rgb}{0.000000, 0.000000, 0.000000}
\pgfsetstrokecolor{dialinecolor}
\draw (39.439100\du,30.085930\du)--(39.348519\du,30.176511\du);
\pgfsetbuttcap
\pgfsetmiterjoin
\pgfsetdash{}{0pt}
\definecolor{dialinecolor}{rgb}{0.000000, 0.000000, 0.000000}
\pgfsetstrokecolor{dialinecolor}
\draw (39.348519\du,30.176511\du)--(39.439100\du,30.267093\du);
\pgfsetbuttcap
\pgfsetmiterjoin
\pgfsetdash{}{0pt}
\definecolor{dialinecolor}{rgb}{0.000000, 0.000000, 0.000000}
\pgfsetstrokecolor{dialinecolor}
\draw (39.076774\du,30.085930\du)--(38.986193\du,30.176511\du);
\pgfsetbuttcap
\pgfsetmiterjoin
\pgfsetdash{}{0pt}
\definecolor{dialinecolor}{rgb}{0.000000, 0.000000, 0.000000}
\pgfsetstrokecolor{dialinecolor}
\draw (38.986193\du,30.176511\du)--(39.076774\du,30.267093\du);
\pgfsetbuttcap
\pgfsetmiterjoin
\pgfsetdash{}{0pt}
\definecolor{dialinecolor}{rgb}{0.000000, 0.000000, 0.000000}
\pgfsetstrokecolor{dialinecolor}
\draw (38.714448\du,30.085930\du)--(38.623867\du,30.176511\du);
\pgfsetbuttcap
\pgfsetmiterjoin
\pgfsetdash{}{0pt}
\definecolor{dialinecolor}{rgb}{0.000000, 0.000000, 0.000000}
\pgfsetstrokecolor{dialinecolor}
\draw (38.623867\du,30.176511\du)--(38.714448\du,30.267093\du);
\pgfsetbuttcap
\pgfsetmiterjoin
\pgfsetdash{}{0pt}
\definecolor{dialinecolor}{rgb}{0.000000, 0.000000, 0.000000}
\pgfsetstrokecolor{dialinecolor}
\draw (38.352122\du,30.085930\du)--(38.261541\du,30.176511\du);
\pgfsetbuttcap
\pgfsetmiterjoin
\pgfsetdash{}{0pt}
\definecolor{dialinecolor}{rgb}{0.000000, 0.000000, 0.000000}
\pgfsetstrokecolor{dialinecolor}
\draw (38.261541\du,30.176511\du)--(38.352122\du,30.267093\du);
\pgfsetbuttcap
\pgfsetmiterjoin
\pgfsetdash{}{0pt}
\definecolor{dialinecolor}{rgb}{0.000000, 0.000000, 0.000000}
\pgfsetstrokecolor{dialinecolor}
\draw (37.989796\du,30.085930\du)--(37.899214\du,30.176511\du);
\pgfsetbuttcap
\pgfsetmiterjoin
\pgfsetdash{}{0pt}
\definecolor{dialinecolor}{rgb}{0.000000, 0.000000, 0.000000}
\pgfsetstrokecolor{dialinecolor}
\draw (37.899214\du,30.176511\du)--(37.989796\du,30.267093\du);
\pgfsetbuttcap
\pgfsetmiterjoin
\pgfsetdash{}{0pt}
\definecolor{dialinecolor}{rgb}{0.000000, 0.000000, 0.000000}
\pgfsetstrokecolor{dialinecolor}
\draw (37.627470\du,30.085930\du)--(37.536888\du,30.176511\du);
\pgfsetbuttcap
\pgfsetmiterjoin
\pgfsetdash{}{0pt}
\definecolor{dialinecolor}{rgb}{0.000000, 0.000000, 0.000000}
\pgfsetstrokecolor{dialinecolor}
\draw (37.536888\du,30.176511\du)--(37.627470\du,30.267093\du);
\pgfsetlinewidth{0.100000\du}
\pgfsetbuttcap
\pgfsetmiterjoin
\pgfsetdash{}{0pt}
\definecolor{dialinecolor}{rgb}{0.850980, 0.850980, 0.803922}
\pgfsetfillcolor{dialinecolor}
\fill (37.083981\du,30.176511\du)--(37.083981\du,30.267093\du)--(37.446307\du,30.267093\du)--(37.446307\du,30.176511\du)--cycle;
\definecolor{dialinecolor}{rgb}{0.000000, 0.000000, 0.000000}
\pgfsetstrokecolor{dialinecolor}
\draw (37.083981\du,30.176511\du)--(37.083981\du,30.267093\du)--(37.446307\du,30.267093\du)--(37.446307\du,30.176511\du)--cycle;
\pgfsetlinewidth{0.010000\du}
\pgfsetbuttcap
\pgfsetmiterjoin
\pgfsetdash{}{0pt}
\definecolor{dialinecolor}{rgb}{0.000000, 0.000000, 0.000000}
\pgfsetstrokecolor{dialinecolor}
\draw (37.083981\du,30.176511\du)--(37.083981\du,30.267093\du)--(37.446307\du,30.267093\du)--(37.446307\du,30.176511\du)--cycle;
\pgfsetlinewidth{0.100000\du}
\pgfsetbuttcap
\pgfsetmiterjoin
\pgfsetdash{}{0pt}
\definecolor{dialinecolor}{rgb}{0.803922, 0.803922, 0.803922}
\pgfsetfillcolor{dialinecolor}
\fill (36.721655\du,29.904767\du)--(36.902818\du,30.085930\du)--(36.721655\du,30.267093\du)--(36.540492\du,30.085930\du)--cycle;
\definecolor{dialinecolor}{rgb}{0.000000, 0.000000, 0.000000}
\pgfsetstrokecolor{dialinecolor}
\draw (36.721655\du,29.904767\du)--(36.902818\du,30.085930\du)--(36.721655\du,30.267093\du)--(36.540492\du,30.085930\du)--cycle;
\pgfsetlinewidth{0.010000\du}
\pgfsetbuttcap
\pgfsetmiterjoin
\pgfsetdash{}{0pt}
\definecolor{dialinecolor}{rgb}{0.000000, 0.000000, 0.000000}
\pgfsetstrokecolor{dialinecolor}
\draw (36.721655\du,29.904767\du)--(36.902818\du,30.085930\du)--(36.721655\du,30.267093\du)--(36.540492\du,30.085930\du)--cycle;
\pgfsetlinewidth{0.100000\du}
\pgfsetbuttcap
\pgfsetmiterjoin
\pgfsetdash{}{0pt}
\definecolor{dialinecolor}{rgb}{0.850980, 0.850980, 0.803922}
\pgfsetfillcolor{dialinecolor}
\fill (36.449910\du,29.633022\du)--(36.449910\du,29.723604\du)--(40.073171\du,29.723604\du)--(40.073171\du,29.633022\du)--cycle;
\definecolor{dialinecolor}{rgb}{0.000000, 0.000000, 0.000000}
\pgfsetstrokecolor{dialinecolor}
\draw (36.449910\du,29.633022\du)--(36.449910\du,29.723604\du)--(40.073171\du,29.723604\du)--(40.073171\du,29.633022\du)--cycle;
\pgfsetlinewidth{0.010000\du}
\pgfsetbuttcap
\pgfsetmiterjoin
\pgfsetdash{}{0pt}
\definecolor{dialinecolor}{rgb}{0.000000, 0.000000, 0.000000}
\pgfsetstrokecolor{dialinecolor}
\draw (36.449910\du,29.633022\du)--(36.449910\du,29.723604\du)--(40.073171\du,29.723604\du)--(40.073171\du,29.633022\du)--cycle;
\pgfsetlinewidth{0.100000\du}
\pgfsetbuttcap
\pgfsetmiterjoin
\pgfsetdash{}{0pt}
\definecolor{dialinecolor}{rgb}{0.803922, 0.803922, 0.803922}
\pgfsetfillcolor{dialinecolor}
\fill (36.902818\du,29.633022\du)--(36.902818\du,29.723604\du)--(37.718051\du,29.723604\du)--(37.718051\du,29.633022\du)--cycle;
\definecolor{dialinecolor}{rgb}{0.000000, 0.000000, 0.000000}
\pgfsetstrokecolor{dialinecolor}
\draw (36.902818\du,29.633022\du)--(36.902818\du,29.723604\du)--(37.718051\du,29.723604\du)--(37.718051\du,29.633022\du)--cycle;
\pgfsetlinewidth{0.010000\du}
\pgfsetbuttcap
\pgfsetmiterjoin
\pgfsetdash{}{0pt}
\definecolor{dialinecolor}{rgb}{0.000000, 0.000000, 0.000000}
\pgfsetstrokecolor{dialinecolor}
\draw (36.902818\du,29.633022\du)--(36.902818\du,29.723604\du)--(37.718051\du,29.723604\du)--(37.718051\du,29.633022\du)--cycle;
\pgfsetlinewidth{0.100000\du}
\pgfsetdash{}{0pt}
\pgfsetdash{}{0pt}
\pgfsetbuttcap
\pgfsetmiterjoin
\pgfsetlinewidth{0.100000\du}
\pgfsetbuttcap
\pgfsetmiterjoin
\pgfsetdash{}{0pt}
\definecolor{dialinecolor}{rgb}{0.850980, 0.850980, 0.803922}
\pgfsetfillcolor{dialinecolor}
\fill (41.243010\du,29.481289\du)--(41.243010\du,30.387104\du)--(44.866271\du,30.387104\du)--(44.866271\du,29.481289\du)--cycle;
\definecolor{dialinecolor}{rgb}{0.000000, 0.000000, 0.000000}
\pgfsetstrokecolor{dialinecolor}
\draw (41.243010\du,29.481289\du)--(41.243010\du,30.387104\du)--(44.866271\du,30.387104\du)--(44.866271\du,29.481289\du)--cycle;
\pgfsetlinewidth{0.010000\du}
\pgfsetbuttcap
\pgfsetmiterjoin
\pgfsetdash{}{0pt}
\definecolor{dialinecolor}{rgb}{0.000000, 0.000000, 0.000000}
\pgfsetstrokecolor{dialinecolor}
\draw (41.243010\du,29.481289\du)--(41.243010\du,30.387104\du)--(44.866271\du,30.387104\du)--(44.866271\du,29.481289\du)--cycle;
\pgfsetbuttcap
\pgfsetmiterjoin
\pgfsetdash{}{0pt}
\definecolor{dialinecolor}{rgb}{0.000000, 0.000000, 0.000000}
\pgfsetstrokecolor{dialinecolor}
\draw (41.877081\du,29.843615\du)--(41.786499\du,29.934197\du);
\pgfsetbuttcap
\pgfsetmiterjoin
\pgfsetdash{}{0pt}
\definecolor{dialinecolor}{rgb}{0.000000, 0.000000, 0.000000}
\pgfsetstrokecolor{dialinecolor}
\draw (41.786499\du,29.934197\du)--(41.877081\du,30.024778\du);
\pgfsetbuttcap
\pgfsetmiterjoin
\pgfsetdash{}{0pt}
\definecolor{dialinecolor}{rgb}{0.000000, 0.000000, 0.000000}
\pgfsetstrokecolor{dialinecolor}
\draw (42.239407\du,29.843615\du)--(42.148825\du,29.934197\du);
\pgfsetbuttcap
\pgfsetmiterjoin
\pgfsetdash{}{0pt}
\definecolor{dialinecolor}{rgb}{0.000000, 0.000000, 0.000000}
\pgfsetstrokecolor{dialinecolor}
\draw (42.148825\du,29.934197\du)--(42.239407\du,30.024778\du);
\pgfsetbuttcap
\pgfsetmiterjoin
\pgfsetdash{}{0pt}
\definecolor{dialinecolor}{rgb}{0.000000, 0.000000, 0.000000}
\pgfsetstrokecolor{dialinecolor}
\draw (42.601733\du,29.843615\du)--(42.511151\du,29.934197\du);
\pgfsetbuttcap
\pgfsetmiterjoin
\pgfsetdash{}{0pt}
\definecolor{dialinecolor}{rgb}{0.000000, 0.000000, 0.000000}
\pgfsetstrokecolor{dialinecolor}
\draw (42.511151\du,29.934197\du)--(42.601733\du,30.024778\du);
\pgfsetbuttcap
\pgfsetmiterjoin
\pgfsetdash{}{0pt}
\definecolor{dialinecolor}{rgb}{0.000000, 0.000000, 0.000000}
\pgfsetstrokecolor{dialinecolor}
\draw (42.964059\du,29.843615\du)--(42.873478\du,29.934197\du);
\pgfsetbuttcap
\pgfsetmiterjoin
\pgfsetdash{}{0pt}
\definecolor{dialinecolor}{rgb}{0.000000, 0.000000, 0.000000}
\pgfsetstrokecolor{dialinecolor}
\draw (42.873478\du,29.934197\du)--(42.964059\du,30.024778\du);
\pgfsetbuttcap
\pgfsetmiterjoin
\pgfsetdash{}{0pt}
\definecolor{dialinecolor}{rgb}{0.000000, 0.000000, 0.000000}
\pgfsetstrokecolor{dialinecolor}
\draw (43.326385\du,29.843615\du)--(43.235804\du,29.934197\du);
\pgfsetbuttcap
\pgfsetmiterjoin
\pgfsetdash{}{0pt}
\definecolor{dialinecolor}{rgb}{0.000000, 0.000000, 0.000000}
\pgfsetstrokecolor{dialinecolor}
\draw (43.235804\du,29.934197\du)--(43.326385\du,30.024778\du);
\pgfsetbuttcap
\pgfsetmiterjoin
\pgfsetdash{}{0pt}
\definecolor{dialinecolor}{rgb}{0.000000, 0.000000, 0.000000}
\pgfsetstrokecolor{dialinecolor}
\draw (43.688711\du,29.843615\du)--(43.598130\du,29.934197\du);
\pgfsetbuttcap
\pgfsetmiterjoin
\pgfsetdash{}{0pt}
\definecolor{dialinecolor}{rgb}{0.000000, 0.000000, 0.000000}
\pgfsetstrokecolor{dialinecolor}
\draw (43.598130\du,29.934197\du)--(43.688711\du,30.024778\du);
\pgfsetbuttcap
\pgfsetmiterjoin
\pgfsetdash{}{0pt}
\definecolor{dialinecolor}{rgb}{0.000000, 0.000000, 0.000000}
\pgfsetstrokecolor{dialinecolor}
\draw (44.051037\du,29.843615\du)--(43.960456\du,29.934197\du);
\pgfsetbuttcap
\pgfsetmiterjoin
\pgfsetdash{}{0pt}
\definecolor{dialinecolor}{rgb}{0.000000, 0.000000, 0.000000}
\pgfsetstrokecolor{dialinecolor}
\draw (43.960456\du,29.934197\du)--(44.051037\du,30.024778\du);
\pgfsetbuttcap
\pgfsetmiterjoin
\pgfsetdash{}{0pt}
\definecolor{dialinecolor}{rgb}{0.000000, 0.000000, 0.000000}
\pgfsetstrokecolor{dialinecolor}
\draw (44.413364\du,29.843615\du)--(44.322782\du,29.934197\du);
\pgfsetbuttcap
\pgfsetmiterjoin
\pgfsetdash{}{0pt}
\definecolor{dialinecolor}{rgb}{0.000000, 0.000000, 0.000000}
\pgfsetstrokecolor{dialinecolor}
\draw (44.322782\du,29.934197\du)--(44.413364\du,30.024778\du);
\pgfsetbuttcap
\pgfsetmiterjoin
\pgfsetdash{}{0pt}
\definecolor{dialinecolor}{rgb}{0.000000, 0.000000, 0.000000}
\pgfsetstrokecolor{dialinecolor}
\draw (44.594527\du,30.115360\du)--(44.503945\du,30.205941\du);
\pgfsetbuttcap
\pgfsetmiterjoin
\pgfsetdash{}{0pt}
\definecolor{dialinecolor}{rgb}{0.000000, 0.000000, 0.000000}
\pgfsetstrokecolor{dialinecolor}
\draw (44.503945\du,30.205941\du)--(44.594527\du,30.296523\du);
\pgfsetbuttcap
\pgfsetmiterjoin
\pgfsetdash{}{0pt}
\definecolor{dialinecolor}{rgb}{0.000000, 0.000000, 0.000000}
\pgfsetstrokecolor{dialinecolor}
\draw (44.232200\du,30.115360\du)--(44.141619\du,30.205941\du);
\pgfsetbuttcap
\pgfsetmiterjoin
\pgfsetdash{}{0pt}
\definecolor{dialinecolor}{rgb}{0.000000, 0.000000, 0.000000}
\pgfsetstrokecolor{dialinecolor}
\draw (44.141619\du,30.205941\du)--(44.232200\du,30.296523\du);
\pgfsetbuttcap
\pgfsetmiterjoin
\pgfsetdash{}{0pt}
\definecolor{dialinecolor}{rgb}{0.000000, 0.000000, 0.000000}
\pgfsetstrokecolor{dialinecolor}
\draw (43.869874\du,30.115360\du)--(43.779293\du,30.205941\du);
\pgfsetbuttcap
\pgfsetmiterjoin
\pgfsetdash{}{0pt}
\definecolor{dialinecolor}{rgb}{0.000000, 0.000000, 0.000000}
\pgfsetstrokecolor{dialinecolor}
\draw (43.779293\du,30.205941\du)--(43.869874\du,30.296523\du);
\pgfsetbuttcap
\pgfsetmiterjoin
\pgfsetdash{}{0pt}
\definecolor{dialinecolor}{rgb}{0.000000, 0.000000, 0.000000}
\pgfsetstrokecolor{dialinecolor}
\draw (43.507548\du,30.115360\du)--(43.416967\du,30.205941\du);
\pgfsetbuttcap
\pgfsetmiterjoin
\pgfsetdash{}{0pt}
\definecolor{dialinecolor}{rgb}{0.000000, 0.000000, 0.000000}
\pgfsetstrokecolor{dialinecolor}
\draw (43.416967\du,30.205941\du)--(43.507548\du,30.296523\du);
\pgfsetbuttcap
\pgfsetmiterjoin
\pgfsetdash{}{0pt}
\definecolor{dialinecolor}{rgb}{0.000000, 0.000000, 0.000000}
\pgfsetstrokecolor{dialinecolor}
\draw (43.145222\du,30.115360\du)--(43.054641\du,30.205941\du);
\pgfsetbuttcap
\pgfsetmiterjoin
\pgfsetdash{}{0pt}
\definecolor{dialinecolor}{rgb}{0.000000, 0.000000, 0.000000}
\pgfsetstrokecolor{dialinecolor}
\draw (43.054641\du,30.205941\du)--(43.145222\du,30.296523\du);
\pgfsetbuttcap
\pgfsetmiterjoin
\pgfsetdash{}{0pt}
\definecolor{dialinecolor}{rgb}{0.000000, 0.000000, 0.000000}
\pgfsetstrokecolor{dialinecolor}
\draw (42.782896\du,30.115360\du)--(42.692314\du,30.205941\du);
\pgfsetbuttcap
\pgfsetmiterjoin
\pgfsetdash{}{0pt}
\definecolor{dialinecolor}{rgb}{0.000000, 0.000000, 0.000000}
\pgfsetstrokecolor{dialinecolor}
\draw (42.692314\du,30.205941\du)--(42.782896\du,30.296523\du);
\pgfsetbuttcap
\pgfsetmiterjoin
\pgfsetdash{}{0pt}
\definecolor{dialinecolor}{rgb}{0.000000, 0.000000, 0.000000}
\pgfsetstrokecolor{dialinecolor}
\draw (42.420570\du,30.115360\du)--(42.329988\du,30.205941\du);
\pgfsetbuttcap
\pgfsetmiterjoin
\pgfsetdash{}{0pt}
\definecolor{dialinecolor}{rgb}{0.000000, 0.000000, 0.000000}
\pgfsetstrokecolor{dialinecolor}
\draw (42.329988\du,30.205941\du)--(42.420570\du,30.296523\du);
\pgfsetlinewidth{0.100000\du}
\pgfsetbuttcap
\pgfsetmiterjoin
\pgfsetdash{}{0pt}
\definecolor{dialinecolor}{rgb}{0.850980, 0.850980, 0.803922}
\pgfsetfillcolor{dialinecolor}
\fill (41.877081\du,30.205941\du)--(41.877081\du,30.296523\du)--(42.239407\du,30.296523\du)--(42.239407\du,30.205941\du)--cycle;
\definecolor{dialinecolor}{rgb}{0.000000, 0.000000, 0.000000}
\pgfsetstrokecolor{dialinecolor}
\draw (41.877081\du,30.205941\du)--(41.877081\du,30.296523\du)--(42.239407\du,30.296523\du)--(42.239407\du,30.205941\du)--cycle;
\pgfsetlinewidth{0.010000\du}
\pgfsetbuttcap
\pgfsetmiterjoin
\pgfsetdash{}{0pt}
\definecolor{dialinecolor}{rgb}{0.000000, 0.000000, 0.000000}
\pgfsetstrokecolor{dialinecolor}
\draw (41.877081\du,30.205941\du)--(41.877081\du,30.296523\du)--(42.239407\du,30.296523\du)--(42.239407\du,30.205941\du)--cycle;
\pgfsetlinewidth{0.100000\du}
\pgfsetbuttcap
\pgfsetmiterjoin
\pgfsetdash{}{0pt}
\definecolor{dialinecolor}{rgb}{0.803922, 0.803922, 0.803922}
\pgfsetfillcolor{dialinecolor}
\fill (41.514755\du,29.934197\du)--(41.695918\du,30.115360\du)--(41.514755\du,30.296523\du)--(41.333592\du,30.115360\du)--cycle;
\definecolor{dialinecolor}{rgb}{0.000000, 0.000000, 0.000000}
\pgfsetstrokecolor{dialinecolor}
\draw (41.514755\du,29.934197\du)--(41.695918\du,30.115360\du)--(41.514755\du,30.296523\du)--(41.333592\du,30.115360\du)--cycle;
\pgfsetlinewidth{0.010000\du}
\pgfsetbuttcap
\pgfsetmiterjoin
\pgfsetdash{}{0pt}
\definecolor{dialinecolor}{rgb}{0.000000, 0.000000, 0.000000}
\pgfsetstrokecolor{dialinecolor}
\draw (41.514755\du,29.934197\du)--(41.695918\du,30.115360\du)--(41.514755\du,30.296523\du)--(41.333592\du,30.115360\du)--cycle;
\pgfsetlinewidth{0.100000\du}
\pgfsetbuttcap
\pgfsetmiterjoin
\pgfsetdash{}{0pt}
\definecolor{dialinecolor}{rgb}{0.850980, 0.850980, 0.803922}
\pgfsetfillcolor{dialinecolor}
\fill (41.243010\du,29.662452\du)--(41.243010\du,29.753034\du)--(44.866271\du,29.753034\du)--(44.866271\du,29.662452\du)--cycle;
\definecolor{dialinecolor}{rgb}{0.000000, 0.000000, 0.000000}
\pgfsetstrokecolor{dialinecolor}
\draw (41.243010\du,29.662452\du)--(41.243010\du,29.753034\du)--(44.866271\du,29.753034\du)--(44.866271\du,29.662452\du)--cycle;
\pgfsetlinewidth{0.010000\du}
\pgfsetbuttcap
\pgfsetmiterjoin
\pgfsetdash{}{0pt}
\definecolor{dialinecolor}{rgb}{0.000000, 0.000000, 0.000000}
\pgfsetstrokecolor{dialinecolor}
\draw (41.243010\du,29.662452\du)--(41.243010\du,29.753034\du)--(44.866271\du,29.753034\du)--(44.866271\du,29.662452\du)--cycle;
\pgfsetlinewidth{0.100000\du}
\pgfsetbuttcap
\pgfsetmiterjoin
\pgfsetdash{}{0pt}
\definecolor{dialinecolor}{rgb}{0.803922, 0.803922, 0.803922}
\pgfsetfillcolor{dialinecolor}
\fill (41.695918\du,29.662452\du)--(41.695918\du,29.753034\du)--(42.511151\du,29.753034\du)--(42.511151\du,29.662452\du)--cycle;
\definecolor{dialinecolor}{rgb}{0.000000, 0.000000, 0.000000}
\pgfsetstrokecolor{dialinecolor}
\draw (41.695918\du,29.662452\du)--(41.695918\du,29.753034\du)--(42.511151\du,29.753034\du)--(42.511151\du,29.662452\du)--cycle;
\pgfsetlinewidth{0.010000\du}
\pgfsetbuttcap
\pgfsetmiterjoin
\pgfsetdash{}{0pt}
\definecolor{dialinecolor}{rgb}{0.000000, 0.000000, 0.000000}
\pgfsetstrokecolor{dialinecolor}
\draw (41.695918\du,29.662452\du)--(41.695918\du,29.753034\du)--(42.511151\du,29.753034\du)--(42.511151\du,29.662452\du)--cycle;
\pgfsetlinewidth{0.100000\du}
\pgfsetdash{}{0pt}
\pgfsetdash{}{0pt}
\pgfsetbuttcap
\pgfsetmiterjoin
\pgfsetlinewidth{0.100000\du}
\pgfsetbuttcap
\pgfsetmiterjoin
\pgfsetdash{}{0pt}
\definecolor{dialinecolor}{rgb}{0.850980, 0.850980, 0.803922}
\pgfsetfillcolor{dialinecolor}
\fill (31.614800\du,34.606529\du)--(31.614800\du,35.512344\du)--(35.238061\du,35.512344\du)--(35.238061\du,34.606529\du)--cycle;
\definecolor{dialinecolor}{rgb}{0.000000, 0.000000, 0.000000}
\pgfsetstrokecolor{dialinecolor}
\draw (31.614800\du,34.606529\du)--(31.614800\du,35.512344\du)--(35.238061\du,35.512344\du)--(35.238061\du,34.606529\du)--cycle;
\pgfsetlinewidth{0.010000\du}
\pgfsetbuttcap
\pgfsetmiterjoin
\pgfsetdash{}{0pt}
\definecolor{dialinecolor}{rgb}{0.000000, 0.000000, 0.000000}
\pgfsetstrokecolor{dialinecolor}
\draw (31.614800\du,34.606529\du)--(31.614800\du,35.512344\du)--(35.238061\du,35.512344\du)--(35.238061\du,34.606529\du)--cycle;
\pgfsetbuttcap
\pgfsetmiterjoin
\pgfsetdash{}{0pt}
\definecolor{dialinecolor}{rgb}{0.000000, 0.000000, 0.000000}
\pgfsetstrokecolor{dialinecolor}
\draw (32.248871\du,34.968855\du)--(32.158289\du,35.059437\du);
\pgfsetbuttcap
\pgfsetmiterjoin
\pgfsetdash{}{0pt}
\definecolor{dialinecolor}{rgb}{0.000000, 0.000000, 0.000000}
\pgfsetstrokecolor{dialinecolor}
\draw (32.158289\du,35.059437\du)--(32.248871\du,35.150018\du);
\pgfsetbuttcap
\pgfsetmiterjoin
\pgfsetdash{}{0pt}
\definecolor{dialinecolor}{rgb}{0.000000, 0.000000, 0.000000}
\pgfsetstrokecolor{dialinecolor}
\draw (32.611197\du,34.968855\du)--(32.520615\du,35.059437\du);
\pgfsetbuttcap
\pgfsetmiterjoin
\pgfsetdash{}{0pt}
\definecolor{dialinecolor}{rgb}{0.000000, 0.000000, 0.000000}
\pgfsetstrokecolor{dialinecolor}
\draw (32.520615\du,35.059437\du)--(32.611197\du,35.150018\du);
\pgfsetbuttcap
\pgfsetmiterjoin
\pgfsetdash{}{0pt}
\definecolor{dialinecolor}{rgb}{0.000000, 0.000000, 0.000000}
\pgfsetstrokecolor{dialinecolor}
\draw (32.973523\du,34.968855\du)--(32.882941\du,35.059437\du);
\pgfsetbuttcap
\pgfsetmiterjoin
\pgfsetdash{}{0pt}
\definecolor{dialinecolor}{rgb}{0.000000, 0.000000, 0.000000}
\pgfsetstrokecolor{dialinecolor}
\draw (32.882941\du,35.059437\du)--(32.973523\du,35.150018\du);
\pgfsetbuttcap
\pgfsetmiterjoin
\pgfsetdash{}{0pt}
\definecolor{dialinecolor}{rgb}{0.000000, 0.000000, 0.000000}
\pgfsetstrokecolor{dialinecolor}
\draw (33.335849\du,34.968855\du)--(33.245268\du,35.059437\du);
\pgfsetbuttcap
\pgfsetmiterjoin
\pgfsetdash{}{0pt}
\definecolor{dialinecolor}{rgb}{0.000000, 0.000000, 0.000000}
\pgfsetstrokecolor{dialinecolor}
\draw (33.245268\du,35.059437\du)--(33.335849\du,35.150018\du);
\pgfsetbuttcap
\pgfsetmiterjoin
\pgfsetdash{}{0pt}
\definecolor{dialinecolor}{rgb}{0.000000, 0.000000, 0.000000}
\pgfsetstrokecolor{dialinecolor}
\draw (33.698175\du,34.968855\du)--(33.607594\du,35.059437\du);
\pgfsetbuttcap
\pgfsetmiterjoin
\pgfsetdash{}{0pt}
\definecolor{dialinecolor}{rgb}{0.000000, 0.000000, 0.000000}
\pgfsetstrokecolor{dialinecolor}
\draw (33.607594\du,35.059437\du)--(33.698175\du,35.150018\du);
\pgfsetbuttcap
\pgfsetmiterjoin
\pgfsetdash{}{0pt}
\definecolor{dialinecolor}{rgb}{0.000000, 0.000000, 0.000000}
\pgfsetstrokecolor{dialinecolor}
\draw (34.060501\du,34.968855\du)--(33.969920\du,35.059437\du);
\pgfsetbuttcap
\pgfsetmiterjoin
\pgfsetdash{}{0pt}
\definecolor{dialinecolor}{rgb}{0.000000, 0.000000, 0.000000}
\pgfsetstrokecolor{dialinecolor}
\draw (33.969920\du,35.059437\du)--(34.060501\du,35.150018\du);
\pgfsetbuttcap
\pgfsetmiterjoin
\pgfsetdash{}{0pt}
\definecolor{dialinecolor}{rgb}{0.000000, 0.000000, 0.000000}
\pgfsetstrokecolor{dialinecolor}
\draw (34.422827\du,34.968855\du)--(34.332246\du,35.059437\du);
\pgfsetbuttcap
\pgfsetmiterjoin
\pgfsetdash{}{0pt}
\definecolor{dialinecolor}{rgb}{0.000000, 0.000000, 0.000000}
\pgfsetstrokecolor{dialinecolor}
\draw (34.332246\du,35.059437\du)--(34.422827\du,35.150018\du);
\pgfsetbuttcap
\pgfsetmiterjoin
\pgfsetdash{}{0pt}
\definecolor{dialinecolor}{rgb}{0.000000, 0.000000, 0.000000}
\pgfsetstrokecolor{dialinecolor}
\draw (34.785154\du,34.968855\du)--(34.694572\du,35.059437\du);
\pgfsetbuttcap
\pgfsetmiterjoin
\pgfsetdash{}{0pt}
\definecolor{dialinecolor}{rgb}{0.000000, 0.000000, 0.000000}
\pgfsetstrokecolor{dialinecolor}
\draw (34.694572\du,35.059437\du)--(34.785154\du,35.150018\du);
\pgfsetbuttcap
\pgfsetmiterjoin
\pgfsetdash{}{0pt}
\definecolor{dialinecolor}{rgb}{0.000000, 0.000000, 0.000000}
\pgfsetstrokecolor{dialinecolor}
\draw (34.966317\du,35.240600\du)--(34.875735\du,35.331181\du);
\pgfsetbuttcap
\pgfsetmiterjoin
\pgfsetdash{}{0pt}
\definecolor{dialinecolor}{rgb}{0.000000, 0.000000, 0.000000}
\pgfsetstrokecolor{dialinecolor}
\draw (34.875735\du,35.331181\du)--(34.966317\du,35.421763\du);
\pgfsetbuttcap
\pgfsetmiterjoin
\pgfsetdash{}{0pt}
\definecolor{dialinecolor}{rgb}{0.000000, 0.000000, 0.000000}
\pgfsetstrokecolor{dialinecolor}
\draw (34.603990\du,35.240600\du)--(34.513409\du,35.331181\du);
\pgfsetbuttcap
\pgfsetmiterjoin
\pgfsetdash{}{0pt}
\definecolor{dialinecolor}{rgb}{0.000000, 0.000000, 0.000000}
\pgfsetstrokecolor{dialinecolor}
\draw (34.513409\du,35.331181\du)--(34.603990\du,35.421763\du);
\pgfsetbuttcap
\pgfsetmiterjoin
\pgfsetdash{}{0pt}
\definecolor{dialinecolor}{rgb}{0.000000, 0.000000, 0.000000}
\pgfsetstrokecolor{dialinecolor}
\draw (34.241664\du,35.240600\du)--(34.151083\du,35.331181\du);
\pgfsetbuttcap
\pgfsetmiterjoin
\pgfsetdash{}{0pt}
\definecolor{dialinecolor}{rgb}{0.000000, 0.000000, 0.000000}
\pgfsetstrokecolor{dialinecolor}
\draw (34.151083\du,35.331181\du)--(34.241664\du,35.421763\du);
\pgfsetbuttcap
\pgfsetmiterjoin
\pgfsetdash{}{0pt}
\definecolor{dialinecolor}{rgb}{0.000000, 0.000000, 0.000000}
\pgfsetstrokecolor{dialinecolor}
\draw (33.879338\du,35.240600\du)--(33.788757\du,35.331181\du);
\pgfsetbuttcap
\pgfsetmiterjoin
\pgfsetdash{}{0pt}
\definecolor{dialinecolor}{rgb}{0.000000, 0.000000, 0.000000}
\pgfsetstrokecolor{dialinecolor}
\draw (33.788757\du,35.331181\du)--(33.879338\du,35.421763\du);
\pgfsetbuttcap
\pgfsetmiterjoin
\pgfsetdash{}{0pt}
\definecolor{dialinecolor}{rgb}{0.000000, 0.000000, 0.000000}
\pgfsetstrokecolor{dialinecolor}
\draw (33.517012\du,35.240600\du)--(33.426431\du,35.331181\du);
\pgfsetbuttcap
\pgfsetmiterjoin
\pgfsetdash{}{0pt}
\definecolor{dialinecolor}{rgb}{0.000000, 0.000000, 0.000000}
\pgfsetstrokecolor{dialinecolor}
\draw (33.426431\du,35.331181\du)--(33.517012\du,35.421763\du);
\pgfsetbuttcap
\pgfsetmiterjoin
\pgfsetdash{}{0pt}
\definecolor{dialinecolor}{rgb}{0.000000, 0.000000, 0.000000}
\pgfsetstrokecolor{dialinecolor}
\draw (33.154686\du,35.240600\du)--(33.064104\du,35.331181\du);
\pgfsetbuttcap
\pgfsetmiterjoin
\pgfsetdash{}{0pt}
\definecolor{dialinecolor}{rgb}{0.000000, 0.000000, 0.000000}
\pgfsetstrokecolor{dialinecolor}
\draw (33.064104\du,35.331181\du)--(33.154686\du,35.421763\du);
\pgfsetbuttcap
\pgfsetmiterjoin
\pgfsetdash{}{0pt}
\definecolor{dialinecolor}{rgb}{0.000000, 0.000000, 0.000000}
\pgfsetstrokecolor{dialinecolor}
\draw (32.792360\du,35.240600\du)--(32.701778\du,35.331181\du);
\pgfsetbuttcap
\pgfsetmiterjoin
\pgfsetdash{}{0pt}
\definecolor{dialinecolor}{rgb}{0.000000, 0.000000, 0.000000}
\pgfsetstrokecolor{dialinecolor}
\draw (32.701778\du,35.331181\du)--(32.792360\du,35.421763\du);
\pgfsetlinewidth{0.100000\du}
\pgfsetbuttcap
\pgfsetmiterjoin
\pgfsetdash{}{0pt}
\definecolor{dialinecolor}{rgb}{0.850980, 0.850980, 0.803922}
\pgfsetfillcolor{dialinecolor}
\fill (32.248871\du,35.331181\du)--(32.248871\du,35.421763\du)--(32.611197\du,35.421763\du)--(32.611197\du,35.331181\du)--cycle;
\definecolor{dialinecolor}{rgb}{0.000000, 0.000000, 0.000000}
\pgfsetstrokecolor{dialinecolor}
\draw (32.248871\du,35.331181\du)--(32.248871\du,35.421763\du)--(32.611197\du,35.421763\du)--(32.611197\du,35.331181\du)--cycle;
\pgfsetlinewidth{0.010000\du}
\pgfsetbuttcap
\pgfsetmiterjoin
\pgfsetdash{}{0pt}
\definecolor{dialinecolor}{rgb}{0.000000, 0.000000, 0.000000}
\pgfsetstrokecolor{dialinecolor}
\draw (32.248871\du,35.331181\du)--(32.248871\du,35.421763\du)--(32.611197\du,35.421763\du)--(32.611197\du,35.331181\du)--cycle;
\pgfsetlinewidth{0.100000\du}
\pgfsetbuttcap
\pgfsetmiterjoin
\pgfsetdash{}{0pt}
\definecolor{dialinecolor}{rgb}{0.803922, 0.803922, 0.803922}
\pgfsetfillcolor{dialinecolor}
\fill (31.886545\du,35.059437\du)--(32.067708\du,35.240600\du)--(31.886545\du,35.421763\du)--(31.705382\du,35.240600\du)--cycle;
\definecolor{dialinecolor}{rgb}{0.000000, 0.000000, 0.000000}
\pgfsetstrokecolor{dialinecolor}
\draw (31.886545\du,35.059437\du)--(32.067708\du,35.240600\du)--(31.886545\du,35.421763\du)--(31.705382\du,35.240600\du)--cycle;
\pgfsetlinewidth{0.010000\du}
\pgfsetbuttcap
\pgfsetmiterjoin
\pgfsetdash{}{0pt}
\definecolor{dialinecolor}{rgb}{0.000000, 0.000000, 0.000000}
\pgfsetstrokecolor{dialinecolor}
\draw (31.886545\du,35.059437\du)--(32.067708\du,35.240600\du)--(31.886545\du,35.421763\du)--(31.705382\du,35.240600\du)--cycle;
\pgfsetlinewidth{0.100000\du}
\pgfsetbuttcap
\pgfsetmiterjoin
\pgfsetdash{}{0pt}
\definecolor{dialinecolor}{rgb}{0.850980, 0.850980, 0.803922}
\pgfsetfillcolor{dialinecolor}
\fill (31.614800\du,34.787692\du)--(31.614800\du,34.878274\du)--(35.238061\du,34.878274\du)--(35.238061\du,34.787692\du)--cycle;
\definecolor{dialinecolor}{rgb}{0.000000, 0.000000, 0.000000}
\pgfsetstrokecolor{dialinecolor}
\draw (31.614800\du,34.787692\du)--(31.614800\du,34.878274\du)--(35.238061\du,34.878274\du)--(35.238061\du,34.787692\du)--cycle;
\pgfsetlinewidth{0.010000\du}
\pgfsetbuttcap
\pgfsetmiterjoin
\pgfsetdash{}{0pt}
\definecolor{dialinecolor}{rgb}{0.000000, 0.000000, 0.000000}
\pgfsetstrokecolor{dialinecolor}
\draw (31.614800\du,34.787692\du)--(31.614800\du,34.878274\du)--(35.238061\du,34.878274\du)--(35.238061\du,34.787692\du)--cycle;
\pgfsetlinewidth{0.100000\du}
\pgfsetbuttcap
\pgfsetmiterjoin
\pgfsetdash{}{0pt}
\definecolor{dialinecolor}{rgb}{0.803922, 0.803922, 0.803922}
\pgfsetfillcolor{dialinecolor}
\fill (32.067708\du,34.787692\du)--(32.067708\du,34.878274\du)--(32.882941\du,34.878274\du)--(32.882941\du,34.787692\du)--cycle;
\definecolor{dialinecolor}{rgb}{0.000000, 0.000000, 0.000000}
\pgfsetstrokecolor{dialinecolor}
\draw (32.067708\du,34.787692\du)--(32.067708\du,34.878274\du)--(32.882941\du,34.878274\du)--(32.882941\du,34.787692\du)--cycle;
\pgfsetlinewidth{0.010000\du}
\pgfsetbuttcap
\pgfsetmiterjoin
\pgfsetdash{}{0pt}
\definecolor{dialinecolor}{rgb}{0.000000, 0.000000, 0.000000}
\pgfsetstrokecolor{dialinecolor}
\draw (32.067708\du,34.787692\du)--(32.067708\du,34.878274\du)--(32.882941\du,34.878274\du)--(32.882941\du,34.787692\du)--cycle;
\pgfsetlinewidth{0.100000\du}
\pgfsetdash{}{0pt}
\pgfsetdash{}{0pt}
\pgfsetbuttcap
\pgfsetmiterjoin
\pgfsetlinewidth{0.100000\du}
\pgfsetbuttcap
\pgfsetmiterjoin
\pgfsetdash{}{0pt}
\definecolor{dialinecolor}{rgb}{0.850980, 0.850980, 0.803922}
\pgfsetfillcolor{dialinecolor}
\fill (36.449910\du,34.593929\du)--(36.449910\du,35.499744\du)--(40.073171\du,35.499744\du)--(40.073171\du,34.593929\du)--cycle;
\definecolor{dialinecolor}{rgb}{0.000000, 0.000000, 0.000000}
\pgfsetstrokecolor{dialinecolor}
\draw (36.449910\du,34.593929\du)--(36.449910\du,35.499744\du)--(40.073171\du,35.499744\du)--(40.073171\du,34.593929\du)--cycle;
\pgfsetlinewidth{0.010000\du}
\pgfsetbuttcap
\pgfsetmiterjoin
\pgfsetdash{}{0pt}
\definecolor{dialinecolor}{rgb}{0.000000, 0.000000, 0.000000}
\pgfsetstrokecolor{dialinecolor}
\draw (36.449910\du,34.593929\du)--(36.449910\du,35.499744\du)--(40.073171\du,35.499744\du)--(40.073171\du,34.593929\du)--cycle;
\pgfsetbuttcap
\pgfsetmiterjoin
\pgfsetdash{}{0pt}
\definecolor{dialinecolor}{rgb}{0.000000, 0.000000, 0.000000}
\pgfsetstrokecolor{dialinecolor}
\draw (37.083981\du,34.956255\du)--(36.993399\du,35.046837\du);
\pgfsetbuttcap
\pgfsetmiterjoin
\pgfsetdash{}{0pt}
\definecolor{dialinecolor}{rgb}{0.000000, 0.000000, 0.000000}
\pgfsetstrokecolor{dialinecolor}
\draw (36.993399\du,35.046837\du)--(37.083981\du,35.137418\du);
\pgfsetbuttcap
\pgfsetmiterjoin
\pgfsetdash{}{0pt}
\definecolor{dialinecolor}{rgb}{0.000000, 0.000000, 0.000000}
\pgfsetstrokecolor{dialinecolor}
\draw (37.446307\du,34.956255\du)--(37.355725\du,35.046837\du);
\pgfsetbuttcap
\pgfsetmiterjoin
\pgfsetdash{}{0pt}
\definecolor{dialinecolor}{rgb}{0.000000, 0.000000, 0.000000}
\pgfsetstrokecolor{dialinecolor}
\draw (37.355725\du,35.046837\du)--(37.446307\du,35.137418\du);
\pgfsetbuttcap
\pgfsetmiterjoin
\pgfsetdash{}{0pt}
\definecolor{dialinecolor}{rgb}{0.000000, 0.000000, 0.000000}
\pgfsetstrokecolor{dialinecolor}
\draw (37.808633\du,34.956255\du)--(37.718051\du,35.046837\du);
\pgfsetbuttcap
\pgfsetmiterjoin
\pgfsetdash{}{0pt}
\definecolor{dialinecolor}{rgb}{0.000000, 0.000000, 0.000000}
\pgfsetstrokecolor{dialinecolor}
\draw (37.718051\du,35.046837\du)--(37.808633\du,35.137418\du);
\pgfsetbuttcap
\pgfsetmiterjoin
\pgfsetdash{}{0pt}
\definecolor{dialinecolor}{rgb}{0.000000, 0.000000, 0.000000}
\pgfsetstrokecolor{dialinecolor}
\draw (38.170959\du,34.956255\du)--(38.080378\du,35.046837\du);
\pgfsetbuttcap
\pgfsetmiterjoin
\pgfsetdash{}{0pt}
\definecolor{dialinecolor}{rgb}{0.000000, 0.000000, 0.000000}
\pgfsetstrokecolor{dialinecolor}
\draw (38.080378\du,35.046837\du)--(38.170959\du,35.137418\du);
\pgfsetbuttcap
\pgfsetmiterjoin
\pgfsetdash{}{0pt}
\definecolor{dialinecolor}{rgb}{0.000000, 0.000000, 0.000000}
\pgfsetstrokecolor{dialinecolor}
\draw (38.533285\du,34.956255\du)--(38.442704\du,35.046837\du);
\pgfsetbuttcap
\pgfsetmiterjoin
\pgfsetdash{}{0pt}
\definecolor{dialinecolor}{rgb}{0.000000, 0.000000, 0.000000}
\pgfsetstrokecolor{dialinecolor}
\draw (38.442704\du,35.046837\du)--(38.533285\du,35.137418\du);
\pgfsetbuttcap
\pgfsetmiterjoin
\pgfsetdash{}{0pt}
\definecolor{dialinecolor}{rgb}{0.000000, 0.000000, 0.000000}
\pgfsetstrokecolor{dialinecolor}
\draw (38.895611\du,34.956255\du)--(38.805030\du,35.046837\du);
\pgfsetbuttcap
\pgfsetmiterjoin
\pgfsetdash{}{0pt}
\definecolor{dialinecolor}{rgb}{0.000000, 0.000000, 0.000000}
\pgfsetstrokecolor{dialinecolor}
\draw (38.805030\du,35.046837\du)--(38.895611\du,35.137418\du);
\pgfsetbuttcap
\pgfsetmiterjoin
\pgfsetdash{}{0pt}
\definecolor{dialinecolor}{rgb}{0.000000, 0.000000, 0.000000}
\pgfsetstrokecolor{dialinecolor}
\draw (39.257937\du,34.956255\du)--(39.167356\du,35.046837\du);
\pgfsetbuttcap
\pgfsetmiterjoin
\pgfsetdash{}{0pt}
\definecolor{dialinecolor}{rgb}{0.000000, 0.000000, 0.000000}
\pgfsetstrokecolor{dialinecolor}
\draw (39.167356\du,35.046837\du)--(39.257937\du,35.137418\du);
\pgfsetbuttcap
\pgfsetmiterjoin
\pgfsetdash{}{0pt}
\definecolor{dialinecolor}{rgb}{0.000000, 0.000000, 0.000000}
\pgfsetstrokecolor{dialinecolor}
\draw (39.620264\du,34.956255\du)--(39.529682\du,35.046837\du);
\pgfsetbuttcap
\pgfsetmiterjoin
\pgfsetdash{}{0pt}
\definecolor{dialinecolor}{rgb}{0.000000, 0.000000, 0.000000}
\pgfsetstrokecolor{dialinecolor}
\draw (39.529682\du,35.046837\du)--(39.620264\du,35.137418\du);
\pgfsetbuttcap
\pgfsetmiterjoin
\pgfsetdash{}{0pt}
\definecolor{dialinecolor}{rgb}{0.000000, 0.000000, 0.000000}
\pgfsetstrokecolor{dialinecolor}
\draw (39.801427\du,35.228000\du)--(39.710845\du,35.318581\du);
\pgfsetbuttcap
\pgfsetmiterjoin
\pgfsetdash{}{0pt}
\definecolor{dialinecolor}{rgb}{0.000000, 0.000000, 0.000000}
\pgfsetstrokecolor{dialinecolor}
\draw (39.710845\du,35.318581\du)--(39.801427\du,35.409163\du);
\pgfsetbuttcap
\pgfsetmiterjoin
\pgfsetdash{}{0pt}
\definecolor{dialinecolor}{rgb}{0.000000, 0.000000, 0.000000}
\pgfsetstrokecolor{dialinecolor}
\draw (39.439100\du,35.228000\du)--(39.348519\du,35.318581\du);
\pgfsetbuttcap
\pgfsetmiterjoin
\pgfsetdash{}{0pt}
\definecolor{dialinecolor}{rgb}{0.000000, 0.000000, 0.000000}
\pgfsetstrokecolor{dialinecolor}
\draw (39.348519\du,35.318581\du)--(39.439100\du,35.409163\du);
\pgfsetbuttcap
\pgfsetmiterjoin
\pgfsetdash{}{0pt}
\definecolor{dialinecolor}{rgb}{0.000000, 0.000000, 0.000000}
\pgfsetstrokecolor{dialinecolor}
\draw (39.076774\du,35.228000\du)--(38.986193\du,35.318581\du);
\pgfsetbuttcap
\pgfsetmiterjoin
\pgfsetdash{}{0pt}
\definecolor{dialinecolor}{rgb}{0.000000, 0.000000, 0.000000}
\pgfsetstrokecolor{dialinecolor}
\draw (38.986193\du,35.318581\du)--(39.076774\du,35.409163\du);
\pgfsetbuttcap
\pgfsetmiterjoin
\pgfsetdash{}{0pt}
\definecolor{dialinecolor}{rgb}{0.000000, 0.000000, 0.000000}
\pgfsetstrokecolor{dialinecolor}
\draw (38.714448\du,35.228000\du)--(38.623867\du,35.318581\du);
\pgfsetbuttcap
\pgfsetmiterjoin
\pgfsetdash{}{0pt}
\definecolor{dialinecolor}{rgb}{0.000000, 0.000000, 0.000000}
\pgfsetstrokecolor{dialinecolor}
\draw (38.623867\du,35.318581\du)--(38.714448\du,35.409163\du);
\pgfsetbuttcap
\pgfsetmiterjoin
\pgfsetdash{}{0pt}
\definecolor{dialinecolor}{rgb}{0.000000, 0.000000, 0.000000}
\pgfsetstrokecolor{dialinecolor}
\draw (38.352122\du,35.228000\du)--(38.261541\du,35.318581\du);
\pgfsetbuttcap
\pgfsetmiterjoin
\pgfsetdash{}{0pt}
\definecolor{dialinecolor}{rgb}{0.000000, 0.000000, 0.000000}
\pgfsetstrokecolor{dialinecolor}
\draw (38.261541\du,35.318581\du)--(38.352122\du,35.409163\du);
\pgfsetbuttcap
\pgfsetmiterjoin
\pgfsetdash{}{0pt}
\definecolor{dialinecolor}{rgb}{0.000000, 0.000000, 0.000000}
\pgfsetstrokecolor{dialinecolor}
\draw (37.989796\du,35.228000\du)--(37.899214\du,35.318581\du);
\pgfsetbuttcap
\pgfsetmiterjoin
\pgfsetdash{}{0pt}
\definecolor{dialinecolor}{rgb}{0.000000, 0.000000, 0.000000}
\pgfsetstrokecolor{dialinecolor}
\draw (37.899214\du,35.318581\du)--(37.989796\du,35.409163\du);
\pgfsetbuttcap
\pgfsetmiterjoin
\pgfsetdash{}{0pt}
\definecolor{dialinecolor}{rgb}{0.000000, 0.000000, 0.000000}
\pgfsetstrokecolor{dialinecolor}
\draw (37.627470\du,35.228000\du)--(37.536888\du,35.318581\du);
\pgfsetbuttcap
\pgfsetmiterjoin
\pgfsetdash{}{0pt}
\definecolor{dialinecolor}{rgb}{0.000000, 0.000000, 0.000000}
\pgfsetstrokecolor{dialinecolor}
\draw (37.536888\du,35.318581\du)--(37.627470\du,35.409163\du);
\pgfsetlinewidth{0.100000\du}
\pgfsetbuttcap
\pgfsetmiterjoin
\pgfsetdash{}{0pt}
\definecolor{dialinecolor}{rgb}{0.850980, 0.850980, 0.803922}
\pgfsetfillcolor{dialinecolor}
\fill (37.083981\du,35.318581\du)--(37.083981\du,35.409163\du)--(37.446307\du,35.409163\du)--(37.446307\du,35.318581\du)--cycle;
\definecolor{dialinecolor}{rgb}{0.000000, 0.000000, 0.000000}
\pgfsetstrokecolor{dialinecolor}
\draw (37.083981\du,35.318581\du)--(37.083981\du,35.409163\du)--(37.446307\du,35.409163\du)--(37.446307\du,35.318581\du)--cycle;
\pgfsetlinewidth{0.010000\du}
\pgfsetbuttcap
\pgfsetmiterjoin
\pgfsetdash{}{0pt}
\definecolor{dialinecolor}{rgb}{0.000000, 0.000000, 0.000000}
\pgfsetstrokecolor{dialinecolor}
\draw (37.083981\du,35.318581\du)--(37.083981\du,35.409163\du)--(37.446307\du,35.409163\du)--(37.446307\du,35.318581\du)--cycle;
\pgfsetlinewidth{0.100000\du}
\pgfsetbuttcap
\pgfsetmiterjoin
\pgfsetdash{}{0pt}
\definecolor{dialinecolor}{rgb}{0.803922, 0.803922, 0.803922}
\pgfsetfillcolor{dialinecolor}
\fill (36.721655\du,35.046837\du)--(36.902818\du,35.228000\du)--(36.721655\du,35.409163\du)--(36.540492\du,35.228000\du)--cycle;
\definecolor{dialinecolor}{rgb}{0.000000, 0.000000, 0.000000}
\pgfsetstrokecolor{dialinecolor}
\draw (36.721655\du,35.046837\du)--(36.902818\du,35.228000\du)--(36.721655\du,35.409163\du)--(36.540492\du,35.228000\du)--cycle;
\pgfsetlinewidth{0.010000\du}
\pgfsetbuttcap
\pgfsetmiterjoin
\pgfsetdash{}{0pt}
\definecolor{dialinecolor}{rgb}{0.000000, 0.000000, 0.000000}
\pgfsetstrokecolor{dialinecolor}
\draw (36.721655\du,35.046837\du)--(36.902818\du,35.228000\du)--(36.721655\du,35.409163\du)--(36.540492\du,35.228000\du)--cycle;
\pgfsetlinewidth{0.100000\du}
\pgfsetbuttcap
\pgfsetmiterjoin
\pgfsetdash{}{0pt}
\definecolor{dialinecolor}{rgb}{0.850980, 0.850980, 0.803922}
\pgfsetfillcolor{dialinecolor}
\fill (36.449910\du,34.775092\du)--(36.449910\du,34.865674\du)--(40.073171\du,34.865674\du)--(40.073171\du,34.775092\du)--cycle;
\definecolor{dialinecolor}{rgb}{0.000000, 0.000000, 0.000000}
\pgfsetstrokecolor{dialinecolor}
\draw (36.449910\du,34.775092\du)--(36.449910\du,34.865674\du)--(40.073171\du,34.865674\du)--(40.073171\du,34.775092\du)--cycle;
\pgfsetlinewidth{0.010000\du}
\pgfsetbuttcap
\pgfsetmiterjoin
\pgfsetdash{}{0pt}
\definecolor{dialinecolor}{rgb}{0.000000, 0.000000, 0.000000}
\pgfsetstrokecolor{dialinecolor}
\draw (36.449910\du,34.775092\du)--(36.449910\du,34.865674\du)--(40.073171\du,34.865674\du)--(40.073171\du,34.775092\du)--cycle;
\pgfsetlinewidth{0.100000\du}
\pgfsetbuttcap
\pgfsetmiterjoin
\pgfsetdash{}{0pt}
\definecolor{dialinecolor}{rgb}{0.803922, 0.803922, 0.803922}
\pgfsetfillcolor{dialinecolor}
\fill (36.902818\du,34.775092\du)--(36.902818\du,34.865674\du)--(37.718051\du,34.865674\du)--(37.718051\du,34.775092\du)--cycle;
\definecolor{dialinecolor}{rgb}{0.000000, 0.000000, 0.000000}
\pgfsetstrokecolor{dialinecolor}
\draw (36.902818\du,34.775092\du)--(36.902818\du,34.865674\du)--(37.718051\du,34.865674\du)--(37.718051\du,34.775092\du)--cycle;
\pgfsetlinewidth{0.010000\du}
\pgfsetbuttcap
\pgfsetmiterjoin
\pgfsetdash{}{0pt}
\definecolor{dialinecolor}{rgb}{0.000000, 0.000000, 0.000000}
\pgfsetstrokecolor{dialinecolor}
\draw (36.902818\du,34.775092\du)--(36.902818\du,34.865674\du)--(37.718051\du,34.865674\du)--(37.718051\du,34.775092\du)--cycle;
\pgfsetlinewidth{0.100000\du}
\pgfsetdash{}{0pt}
\pgfsetdash{}{0pt}
\pgfsetbuttcap
\pgfsetmiterjoin
\pgfsetlinewidth{0.100000\du}
\pgfsetbuttcap
\pgfsetmiterjoin
\pgfsetdash{}{0pt}
\definecolor{dialinecolor}{rgb}{0.850980, 0.850980, 0.803922}
\pgfsetfillcolor{dialinecolor}
\fill (41.243010\du,34.623329\du)--(41.243010\du,35.529144\du)--(44.866271\du,35.529144\du)--(44.866271\du,34.623329\du)--cycle;
\definecolor{dialinecolor}{rgb}{0.000000, 0.000000, 0.000000}
\pgfsetstrokecolor{dialinecolor}
\draw (41.243010\du,34.623329\du)--(41.243010\du,35.529144\du)--(44.866271\du,35.529144\du)--(44.866271\du,34.623329\du)--cycle;
\pgfsetlinewidth{0.010000\du}
\pgfsetbuttcap
\pgfsetmiterjoin
\pgfsetdash{}{0pt}
\definecolor{dialinecolor}{rgb}{0.000000, 0.000000, 0.000000}
\pgfsetstrokecolor{dialinecolor}
\draw (41.243010\du,34.623329\du)--(41.243010\du,35.529144\du)--(44.866271\du,35.529144\du)--(44.866271\du,34.623329\du)--cycle;
\pgfsetbuttcap
\pgfsetmiterjoin
\pgfsetdash{}{0pt}
\definecolor{dialinecolor}{rgb}{0.000000, 0.000000, 0.000000}
\pgfsetstrokecolor{dialinecolor}
\draw (41.877081\du,34.985655\du)--(41.786499\du,35.076237\du);
\pgfsetbuttcap
\pgfsetmiterjoin
\pgfsetdash{}{0pt}
\definecolor{dialinecolor}{rgb}{0.000000, 0.000000, 0.000000}
\pgfsetstrokecolor{dialinecolor}
\draw (41.786499\du,35.076237\du)--(41.877081\du,35.166818\du);
\pgfsetbuttcap
\pgfsetmiterjoin
\pgfsetdash{}{0pt}
\definecolor{dialinecolor}{rgb}{0.000000, 0.000000, 0.000000}
\pgfsetstrokecolor{dialinecolor}
\draw (42.239407\du,34.985655\du)--(42.148825\du,35.076237\du);
\pgfsetbuttcap
\pgfsetmiterjoin
\pgfsetdash{}{0pt}
\definecolor{dialinecolor}{rgb}{0.000000, 0.000000, 0.000000}
\pgfsetstrokecolor{dialinecolor}
\draw (42.148825\du,35.076237\du)--(42.239407\du,35.166818\du);
\pgfsetbuttcap
\pgfsetmiterjoin
\pgfsetdash{}{0pt}
\definecolor{dialinecolor}{rgb}{0.000000, 0.000000, 0.000000}
\pgfsetstrokecolor{dialinecolor}
\draw (42.601733\du,34.985655\du)--(42.511151\du,35.076237\du);
\pgfsetbuttcap
\pgfsetmiterjoin
\pgfsetdash{}{0pt}
\definecolor{dialinecolor}{rgb}{0.000000, 0.000000, 0.000000}
\pgfsetstrokecolor{dialinecolor}
\draw (42.511151\du,35.076237\du)--(42.601733\du,35.166818\du);
\pgfsetbuttcap
\pgfsetmiterjoin
\pgfsetdash{}{0pt}
\definecolor{dialinecolor}{rgb}{0.000000, 0.000000, 0.000000}
\pgfsetstrokecolor{dialinecolor}
\draw (42.964059\du,34.985655\du)--(42.873478\du,35.076237\du);
\pgfsetbuttcap
\pgfsetmiterjoin
\pgfsetdash{}{0pt}
\definecolor{dialinecolor}{rgb}{0.000000, 0.000000, 0.000000}
\pgfsetstrokecolor{dialinecolor}
\draw (42.873478\du,35.076237\du)--(42.964059\du,35.166818\du);
\pgfsetbuttcap
\pgfsetmiterjoin
\pgfsetdash{}{0pt}
\definecolor{dialinecolor}{rgb}{0.000000, 0.000000, 0.000000}
\pgfsetstrokecolor{dialinecolor}
\draw (43.326385\du,34.985655\du)--(43.235804\du,35.076237\du);
\pgfsetbuttcap
\pgfsetmiterjoin
\pgfsetdash{}{0pt}
\definecolor{dialinecolor}{rgb}{0.000000, 0.000000, 0.000000}
\pgfsetstrokecolor{dialinecolor}
\draw (43.235804\du,35.076237\du)--(43.326385\du,35.166818\du);
\pgfsetbuttcap
\pgfsetmiterjoin
\pgfsetdash{}{0pt}
\definecolor{dialinecolor}{rgb}{0.000000, 0.000000, 0.000000}
\pgfsetstrokecolor{dialinecolor}
\draw (43.688711\du,34.985655\du)--(43.598130\du,35.076237\du);
\pgfsetbuttcap
\pgfsetmiterjoin
\pgfsetdash{}{0pt}
\definecolor{dialinecolor}{rgb}{0.000000, 0.000000, 0.000000}
\pgfsetstrokecolor{dialinecolor}
\draw (43.598130\du,35.076237\du)--(43.688711\du,35.166818\du);
\pgfsetbuttcap
\pgfsetmiterjoin
\pgfsetdash{}{0pt}
\definecolor{dialinecolor}{rgb}{0.000000, 0.000000, 0.000000}
\pgfsetstrokecolor{dialinecolor}
\draw (44.051037\du,34.985655\du)--(43.960456\du,35.076237\du);
\pgfsetbuttcap
\pgfsetmiterjoin
\pgfsetdash{}{0pt}
\definecolor{dialinecolor}{rgb}{0.000000, 0.000000, 0.000000}
\pgfsetstrokecolor{dialinecolor}
\draw (43.960456\du,35.076237\du)--(44.051037\du,35.166818\du);
\pgfsetbuttcap
\pgfsetmiterjoin
\pgfsetdash{}{0pt}
\definecolor{dialinecolor}{rgb}{0.000000, 0.000000, 0.000000}
\pgfsetstrokecolor{dialinecolor}
\draw (44.413364\du,34.985655\du)--(44.322782\du,35.076237\du);
\pgfsetbuttcap
\pgfsetmiterjoin
\pgfsetdash{}{0pt}
\definecolor{dialinecolor}{rgb}{0.000000, 0.000000, 0.000000}
\pgfsetstrokecolor{dialinecolor}
\draw (44.322782\du,35.076237\du)--(44.413364\du,35.166818\du);
\pgfsetbuttcap
\pgfsetmiterjoin
\pgfsetdash{}{0pt}
\definecolor{dialinecolor}{rgb}{0.000000, 0.000000, 0.000000}
\pgfsetstrokecolor{dialinecolor}
\draw (44.594527\du,35.257400\du)--(44.503945\du,35.347981\du);
\pgfsetbuttcap
\pgfsetmiterjoin
\pgfsetdash{}{0pt}
\definecolor{dialinecolor}{rgb}{0.000000, 0.000000, 0.000000}
\pgfsetstrokecolor{dialinecolor}
\draw (44.503945\du,35.347981\du)--(44.594527\du,35.438563\du);
\pgfsetbuttcap
\pgfsetmiterjoin
\pgfsetdash{}{0pt}
\definecolor{dialinecolor}{rgb}{0.000000, 0.000000, 0.000000}
\pgfsetstrokecolor{dialinecolor}
\draw (44.232200\du,35.257400\du)--(44.141619\du,35.347981\du);
\pgfsetbuttcap
\pgfsetmiterjoin
\pgfsetdash{}{0pt}
\definecolor{dialinecolor}{rgb}{0.000000, 0.000000, 0.000000}
\pgfsetstrokecolor{dialinecolor}
\draw (44.141619\du,35.347981\du)--(44.232200\du,35.438563\du);
\pgfsetbuttcap
\pgfsetmiterjoin
\pgfsetdash{}{0pt}
\definecolor{dialinecolor}{rgb}{0.000000, 0.000000, 0.000000}
\pgfsetstrokecolor{dialinecolor}
\draw (43.869874\du,35.257400\du)--(43.779293\du,35.347981\du);
\pgfsetbuttcap
\pgfsetmiterjoin
\pgfsetdash{}{0pt}
\definecolor{dialinecolor}{rgb}{0.000000, 0.000000, 0.000000}
\pgfsetstrokecolor{dialinecolor}
\draw (43.779293\du,35.347981\du)--(43.869874\du,35.438563\du);
\pgfsetbuttcap
\pgfsetmiterjoin
\pgfsetdash{}{0pt}
\definecolor{dialinecolor}{rgb}{0.000000, 0.000000, 0.000000}
\pgfsetstrokecolor{dialinecolor}
\draw (43.507548\du,35.257400\du)--(43.416967\du,35.347981\du);
\pgfsetbuttcap
\pgfsetmiterjoin
\pgfsetdash{}{0pt}
\definecolor{dialinecolor}{rgb}{0.000000, 0.000000, 0.000000}
\pgfsetstrokecolor{dialinecolor}
\draw (43.416967\du,35.347981\du)--(43.507548\du,35.438563\du);
\pgfsetbuttcap
\pgfsetmiterjoin
\pgfsetdash{}{0pt}
\definecolor{dialinecolor}{rgb}{0.000000, 0.000000, 0.000000}
\pgfsetstrokecolor{dialinecolor}
\draw (43.145222\du,35.257400\du)--(43.054641\du,35.347981\du);
\pgfsetbuttcap
\pgfsetmiterjoin
\pgfsetdash{}{0pt}
\definecolor{dialinecolor}{rgb}{0.000000, 0.000000, 0.000000}
\pgfsetstrokecolor{dialinecolor}
\draw (43.054641\du,35.347981\du)--(43.145222\du,35.438563\du);
\pgfsetbuttcap
\pgfsetmiterjoin
\pgfsetdash{}{0pt}
\definecolor{dialinecolor}{rgb}{0.000000, 0.000000, 0.000000}
\pgfsetstrokecolor{dialinecolor}
\draw (42.782896\du,35.257400\du)--(42.692314\du,35.347981\du);
\pgfsetbuttcap
\pgfsetmiterjoin
\pgfsetdash{}{0pt}
\definecolor{dialinecolor}{rgb}{0.000000, 0.000000, 0.000000}
\pgfsetstrokecolor{dialinecolor}
\draw (42.692314\du,35.347981\du)--(42.782896\du,35.438563\du);
\pgfsetbuttcap
\pgfsetmiterjoin
\pgfsetdash{}{0pt}
\definecolor{dialinecolor}{rgb}{0.000000, 0.000000, 0.000000}
\pgfsetstrokecolor{dialinecolor}
\draw (42.420570\du,35.257400\du)--(42.329988\du,35.347981\du);
\pgfsetbuttcap
\pgfsetmiterjoin
\pgfsetdash{}{0pt}
\definecolor{dialinecolor}{rgb}{0.000000, 0.000000, 0.000000}
\pgfsetstrokecolor{dialinecolor}
\draw (42.329988\du,35.347981\du)--(42.420570\du,35.438563\du);
\pgfsetlinewidth{0.100000\du}
\pgfsetbuttcap
\pgfsetmiterjoin
\pgfsetdash{}{0pt}
\definecolor{dialinecolor}{rgb}{0.850980, 0.850980, 0.803922}
\pgfsetfillcolor{dialinecolor}
\fill (41.877081\du,35.347981\du)--(41.877081\du,35.438563\du)--(42.239407\du,35.438563\du)--(42.239407\du,35.347981\du)--cycle;
\definecolor{dialinecolor}{rgb}{0.000000, 0.000000, 0.000000}
\pgfsetstrokecolor{dialinecolor}
\draw (41.877081\du,35.347981\du)--(41.877081\du,35.438563\du)--(42.239407\du,35.438563\du)--(42.239407\du,35.347981\du)--cycle;
\pgfsetlinewidth{0.010000\du}
\pgfsetbuttcap
\pgfsetmiterjoin
\pgfsetdash{}{0pt}
\definecolor{dialinecolor}{rgb}{0.000000, 0.000000, 0.000000}
\pgfsetstrokecolor{dialinecolor}
\draw (41.877081\du,35.347981\du)--(41.877081\du,35.438563\du)--(42.239407\du,35.438563\du)--(42.239407\du,35.347981\du)--cycle;
\pgfsetlinewidth{0.100000\du}
\pgfsetbuttcap
\pgfsetmiterjoin
\pgfsetdash{}{0pt}
\definecolor{dialinecolor}{rgb}{0.803922, 0.803922, 0.803922}
\pgfsetfillcolor{dialinecolor}
\fill (41.514755\du,35.076237\du)--(41.695918\du,35.257400\du)--(41.514755\du,35.438563\du)--(41.333592\du,35.257400\du)--cycle;
\definecolor{dialinecolor}{rgb}{0.000000, 0.000000, 0.000000}
\pgfsetstrokecolor{dialinecolor}
\draw (41.514755\du,35.076237\du)--(41.695918\du,35.257400\du)--(41.514755\du,35.438563\du)--(41.333592\du,35.257400\du)--cycle;
\pgfsetlinewidth{0.010000\du}
\pgfsetbuttcap
\pgfsetmiterjoin
\pgfsetdash{}{0pt}
\definecolor{dialinecolor}{rgb}{0.000000, 0.000000, 0.000000}
\pgfsetstrokecolor{dialinecolor}
\draw (41.514755\du,35.076237\du)--(41.695918\du,35.257400\du)--(41.514755\du,35.438563\du)--(41.333592\du,35.257400\du)--cycle;
\pgfsetlinewidth{0.100000\du}
\pgfsetbuttcap
\pgfsetmiterjoin
\pgfsetdash{}{0pt}
\definecolor{dialinecolor}{rgb}{0.850980, 0.850980, 0.803922}
\pgfsetfillcolor{dialinecolor}
\fill (41.243010\du,34.804492\du)--(41.243010\du,34.895074\du)--(44.866271\du,34.895074\du)--(44.866271\du,34.804492\du)--cycle;
\definecolor{dialinecolor}{rgb}{0.000000, 0.000000, 0.000000}
\pgfsetstrokecolor{dialinecolor}
\draw (41.243010\du,34.804492\du)--(41.243010\du,34.895074\du)--(44.866271\du,34.895074\du)--(44.866271\du,34.804492\du)--cycle;
\pgfsetlinewidth{0.010000\du}
\pgfsetbuttcap
\pgfsetmiterjoin
\pgfsetdash{}{0pt}
\definecolor{dialinecolor}{rgb}{0.000000, 0.000000, 0.000000}
\pgfsetstrokecolor{dialinecolor}
\draw (41.243010\du,34.804492\du)--(41.243010\du,34.895074\du)--(44.866271\du,34.895074\du)--(44.866271\du,34.804492\du)--cycle;
\pgfsetlinewidth{0.100000\du}
\pgfsetbuttcap
\pgfsetmiterjoin
\pgfsetdash{}{0pt}
\definecolor{dialinecolor}{rgb}{0.803922, 0.803922, 0.803922}
\pgfsetfillcolor{dialinecolor}
\fill (41.695918\du,34.804492\du)--(41.695918\du,34.895074\du)--(42.511151\du,34.895074\du)--(42.511151\du,34.804492\du)--cycle;
\definecolor{dialinecolor}{rgb}{0.000000, 0.000000, 0.000000}
\pgfsetstrokecolor{dialinecolor}
\draw (41.695918\du,34.804492\du)--(41.695918\du,34.895074\du)--(42.511151\du,34.895074\du)--(42.511151\du,34.804492\du)--cycle;
\pgfsetlinewidth{0.010000\du}
\pgfsetbuttcap
\pgfsetmiterjoin
\pgfsetdash{}{0pt}
\definecolor{dialinecolor}{rgb}{0.000000, 0.000000, 0.000000}
\pgfsetstrokecolor{dialinecolor}
\draw (41.695918\du,34.804492\du)--(41.695918\du,34.895074\du)--(42.511151\du,34.895074\du)--(42.511151\du,34.804492\du)--cycle;
% setfont left to latex
\definecolor{dialinecolor}{rgb}{0.000000, 0.000000, 0.000000}
\pgfsetstrokecolor{dialinecolor}
\node[anchor=west] at (31.522700\du,31.235009\du){GTM-Proxy};
% setfont left to latex
\definecolor{dialinecolor}{rgb}{0.000000, 0.000000, 0.000000}
\pgfsetstrokecolor{dialinecolor}
\node[anchor=west] at (31.522700\du,32.035009\du){Coordinator};
% setfont left to latex
\definecolor{dialinecolor}{rgb}{0.000000, 0.000000, 0.000000}
\pgfsetstrokecolor{dialinecolor}
\node[anchor=west] at (31.522700\du,32.835009\du){2x DataNode};
% setfont left to latex
\definecolor{dialinecolor}{rgb}{0.000000, 0.000000, 0.000000}
\pgfsetstrokecolor{dialinecolor}
\node[anchor=west] at (36.315820\du,31.186729\du){GTM-Proxy};
% setfont left to latex
\definecolor{dialinecolor}{rgb}{0.000000, 0.000000, 0.000000}
\pgfsetstrokecolor{dialinecolor}
\node[anchor=west] at (36.315820\du,31.986729\du){Coordinator};
% setfont left to latex
\definecolor{dialinecolor}{rgb}{0.000000, 0.000000, 0.000000}
\pgfsetstrokecolor{dialinecolor}
\node[anchor=west] at (36.315820\du,32.786729\du){2x DataNode};
% setfont left to latex
\definecolor{dialinecolor}{rgb}{0.000000, 0.000000, 0.000000}
\pgfsetstrokecolor{dialinecolor}
\node[anchor=west] at (41.193010\du,31.174119\du){GTM-Proxy};
% setfont left to latex
\definecolor{dialinecolor}{rgb}{0.000000, 0.000000, 0.000000}
\pgfsetstrokecolor{dialinecolor}
\node[anchor=west] at (41.193010\du,31.974119\du){Coordinator};
% setfont left to latex
\definecolor{dialinecolor}{rgb}{0.000000, 0.000000, 0.000000}
\pgfsetstrokecolor{dialinecolor}
\node[anchor=west] at (41.193010\du,32.774119\du){2x DataNode};
% setfont left to latex
\definecolor{dialinecolor}{rgb}{0.000000, 0.000000, 0.000000}
\pgfsetstrokecolor{dialinecolor}
\node[anchor=west] at (31.564800\du,36.375029\du){GTM-Proxy};
% setfont left to latex
\definecolor{dialinecolor}{rgb}{0.000000, 0.000000, 0.000000}
\pgfsetstrokecolor{dialinecolor}
\node[anchor=west] at (31.564800\du,37.175029\du){Coordinator};
% setfont left to latex
\definecolor{dialinecolor}{rgb}{0.000000, 0.000000, 0.000000}
\pgfsetstrokecolor{dialinecolor}
\node[anchor=west] at (31.564800\du,37.975029\du){2x DataNode};
% setfont left to latex
\definecolor{dialinecolor}{rgb}{0.000000, 0.000000, 0.000000}
\pgfsetstrokecolor{dialinecolor}
\node[anchor=west] at (36.357860\du,36.320429\du){GTM-Proxy};
% setfont left to latex
\definecolor{dialinecolor}{rgb}{0.000000, 0.000000, 0.000000}
\pgfsetstrokecolor{dialinecolor}
\node[anchor=west] at (36.357860\du,37.120429\du){Coordinator};
% setfont left to latex
\definecolor{dialinecolor}{rgb}{0.000000, 0.000000, 0.000000}
\pgfsetstrokecolor{dialinecolor}
\node[anchor=west] at (36.357860\du,37.920429\du){2x DataNode};
% setfont left to latex
\definecolor{dialinecolor}{rgb}{0.000000, 0.000000, 0.000000}
\pgfsetstrokecolor{dialinecolor}
\node[anchor=west] at (41.235060\du,36.433929\du){GTM-Proxy};
% setfont left to latex
\definecolor{dialinecolor}{rgb}{0.000000, 0.000000, 0.000000}
\pgfsetstrokecolor{dialinecolor}
\node[anchor=west] at (41.235060\du,37.233929\du){Coordinator};
% setfont left to latex
\definecolor{dialinecolor}{rgb}{0.000000, 0.000000, 0.000000}
\pgfsetstrokecolor{dialinecolor}
\node[anchor=west] at (41.235060\du,38.033929\du){2x DataNode};
\end{tikzpicture}

\caption[Aufbau VMs des Testsystems]{Aufbau VMs des Testsystems}
\label{fig:VMAufb}
\end{figure}

\subsection{Funktionstests}
%Schnittstellen: PostgreSQL mit dblink demonstrieren; PostGIS: Tabelle mit geometry erzeugen, füllen und Funktionen darauf anwenden; UMN: simples Mapfile für nsensorlogs erstellen
%Austauschformate: Übertragung der vorhandenen Daten ist beleg
%EPSG Codes 4326 und 3857 testen (Zwischen ihnen Umwandeln)
%weitere Funktionen: Schläge miteinander verschneiden (typische Fälle verwenden, mit UMN darstellen?); Geostatistik mit R Bibo zeigen; topologische Filterung?; räumliche Filterung anhand von Fields demonstrieren   <- Immer auf Funktionsdefinition mit Parametern verweisen

\subsection{Leistungstests}
%Ziel ist Vergleich und Aussage über Skalierbarkeit(T(1)/T(p))/Effizienz(S(p)/p)

%historische Daten?
%Wie schnell werden nsensorlogs gespeichert?
%Wie schnell werden nsensorlogs abgerufen?
%Laufzeit Kriging: abhängig nach Distribution - entweder nach geom oder farmid oder fileid

%mit SQL Bordmitteln - pg_bench eventuell zusätzlich um ein paar Aussagen zum cluster overhead zu treffen