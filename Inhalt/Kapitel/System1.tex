\chapter{Realisierung mit Postgres-XL}
\label{chapter:postgresxl}

\section{Verwendung}
%Indexstrukturen nicht vergessen!
%Mehrrechner-Datenbanksystem?
%replikationsverfahren wichtig?

\subsection{Installation}
%mit Systemvoraussetzungen
\subsubsection{Systemvoraussetzungen}
Die Dokumentation verwendet hierbei unter\\
\url{http://files.postgres-xl.org/documentation/install-requirements.html} deckungsgleich die offizielle Dokumentation zu den Systemanforderungen von PostgreSQL in \url{http://www.postgresql.org/docs/9.2/static/install-requirements.html}.
Dabei wird ein Linux Betriebssystem, 155MB freien Festplattenspeicher für die Übersetzung, Installation und Erstellung eines leeren Datenbankclusters sowie eine Menge von benötigten und optionalen Paketen genannt.
Neben Linux Derivaten wird auch FreeBSD und Max OS X unterstützt.
Die Prozessorarchitektur von Intel wird unterstützt, andere sind laut Dokumentation ebenso verwendbar.
Im Rahmen dieser Arbeit konnte Postgres-XL auch auf einem Raspberry Pi 1 Model B, basierend auf einem ARMv6 Prozessor und dem Linux Derivat Raspbian, installiert und verwendet werden.


\subsection{Schnittstelle}
%Datenimport gehört dazu
Der Zugriff auf Daten eines Postgres-XL Clusters erfolgt über die Coordinators.
Dazu sind unveränderte Programme und Tools aus dem PostgreSQL Umfeld zu verwenden.
Dazu zählen \Gls{jdbc}, das Kommandozeilentool psql und das grafische Programm zur Datenbankverwaltung pgAdminIII.
Ebenso sind die SQL Befehle bis auf ein paar Ausnahmen deckungsgleich.
Diese Ausnahmen beziehen sich auf die Verteilung der Daten und Verwaltung des Clusters.
Beispielsweise das SQL Statement \textit{Create Table tblname (serial id, text data) Distributed by Hash(id);} weicht durch die Ergänzung \textit{Distributed by} vom PostgreSQL SQL Syntax ab.
Jede Tabelle wird in jeder PostgreSQL Instanz erzeugt.
Dabei werden die Daten entweder nach einem Attribut verteilt oder zwischen den Datenbankinstanzen gespiegelt.
Das Schlüsselwort \textit{Distributed} veranlasst eine Verteilung der Daten, \textit{Replicate} dagegen eine Replikation der Tabelle über alle Nodes.


\subsection{Verarbeitung}


\section{Prototyp}

\subsection{Entwurf}

\subsection{Implementierung}


\section{Tests}
%pgbench http://www.postgresql.org/docs/devel/static/pgbench.html

\subsection{Funktionstests}

\subsection{Leistungstests}
%historische Daten?