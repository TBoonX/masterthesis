\chapter{Ausgangsszenario}

\section{Anforderungen}

% kartografisches Produkt
Aktuelle Möglichkeiten der Datenerfassung über Sensoren und moderne Probenahmegeräte führen zu mehr und mehr Datensätzen, die für einen Landwirtschaftsbetrieb ausgewertet werden müssen. Darüber hinaus besteht die Notwendigkeit, Daten Jahresübergreifend und betriebsübergreifend auszuwerten, um pflanzenbauliche Zusammenhänge über statistische Methoden untersuchen zu können.
In den letzten 3 Jahren wurde beispielsweise nur zum Thema N-Versorgung\footnote{Stickstoffdüngung und -aufnahme} für einen Betrieb etwa 800 Datensätze mit 1,9 Mio Einträgen erfasst. Alle diese Daten haben einen räumlichen Bezug, sie müssen weiterverarbeitet, kartographisch aufbereitet und dargestellt werden.\\
Daraus ergeben sich verschiedenen Anforderungen an die Technologie, die für die Verarbeitung, Analyse und Darstellung verwendet wird:
\begin{itemize}
\item PostgreSQL mit PostGIS zum Datenimport und -export nutzbar
\item Gruppieren und Filtern mit geringer Laufzeit
\item parallele Berechnung\footnote{hier Statistik bzw. Geostatistik sowie Interpolation} über große Datenmengen mit geringer Laufzeit
\item Räumliche Berechnungen wie Verschneiden, Berechnen von Overlays
\item  Unterschiedliche Prinzipien der Kartengenerierung, hier dynamisches rendern aus dem Datenbestand zur Laufzeit oder dynamisches rendern bei Dateneingang wodurch vorgerenderte Karten bereitstehen
\item nutzbare Schnittstelle zur Darstellung mit dem \Gls{umn}
\end{itemize}

% Eventuell technische SIcht extra darstellen, um Eignung der Systeme besser herausarbeiten zu können

Konkret handelt es sich bei den Eingangsdaten um folgende:
\begin{description}
\item[Pflanzenbauliche Daten]\footnote{Sensoren, Bodenuntersuchung, \Gls{bonitur}, Logger} Punktdaten
\item[Basisdaten wie Feldgrenzen] Vektordaten
\item[Externe Satelliteninformationen und Multispektralanalysen] Rasterdaten
\end{description}



\section{Ist-Stand}

