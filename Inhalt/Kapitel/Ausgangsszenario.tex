\chapter{Ausgangsszenario}

\section{Anforderungen}
\label{Anforderungen}

% kartografisches Produkt
Aktuelle Möglichkeiten der Datenerfassung über Sensoren und moderne Probenahmegeräte führen zu mehr und mehr Datensätzen, die für einen Landwirtschaftsbetrieb ausgewertet werden müssen. Darüber hinaus besteht die Notwendigkeit, Daten Jahresübergreifend und betriebsübergreifend auszuwerten, um pflanzenbauliche Zusammenhänge über statistische Methoden untersuchen zu können.
In den letzten 3 Jahren wurde beispielsweise nur zum Thema N-Versorgung\footnote{Stickstoffdüngung und -aufnahme} für einen Betrieb etwa 800 Datensätze mit 1,9 Mio Einträgen erfasst. Alle diese Daten haben einen räumlichen Bezug, sie müssen weiterverarbeitet, kartographisch aufbereitet und dargestellt werden.\\
Daraus ergeben sich verschiedenen Anforderungen an die Technologie, die für die Verarbeitung, Analyse und Darstellung verwendet wird:
\begin{itemize}
\item PostgreSQL mit PostGIS zum Datenimport und -export nutzbar
\item Gruppieren und Filtern mit geringer Laufzeit
\item parallele Berechnung\footnote{hier Statistik bzw. Geostatistik sowie Interpolation} über große Datenmengen mit geringer Laufzeit
\item Räumliche Berechnungen wie Verschneiden, Berechnen von Overlays
\item  Unterschiedliche Prinzipien der Kartengenerierung, hier dynamisches rendern aus dem Datenbestand zur Laufzeit oder dynamisches rendern bei Dateneingang wodurch vorgerenderte Karten bereitstehen % caches wirklich mit untersuchen? wenn ja in Schnittstellen aufnehmen
\item nutzbare Schnittstelle zur Darstellung mit dem \Gls{umn}
\end{itemize}

% Eventuell technische SIcht extra darstellen, um Eignung der Systeme besser herausarbeiten zu können

Konkret handelt es sich bei den Eingangsdaten um folgende:
\begin{description}
\item[Pflanzenbauliche Daten]\footnote{Sensoren, Bodenuntersuchung, \Gls{bonitur}, Logger} Punktdaten
\item[Basisdaten wie Feldgrenzen] Vektordaten
\item[Externe Satelliteninformationen und Multispektralanalysen] Rasterdaten
\end{description}

\subsection{Softwarequalität}
\label{softwarequalität}

Qualitätsmerkmale sind nach DIN 9126\footnote{DIN 9126 wurde durch ISO/IEC 25000 ersetzt, jedoch sind beide nur proprietär verfügbar} in \cite{book:lehrbuchsoftware} S. 258 f. Funktionalität, Zuverlässigkeit, Benutzbarkeit, Effizienz, Änderbarkeit und Übertragbarkeit.
Diese Merkmale werden durch Qualitätskriterien für jeden Anwendungsfall konkretisiert.


Nachfolgend werden die Qualitätsmerkmale für diesen Anwendungsfall konkretisiert und darauf die zu untersuchenden aufgelistet.


Da die zu analysierenden Systeme eine Datenbank beinhaltet, welche mit räumlichen Datentypen arbeitet, wurde die im Anhang C von \cite{book:objdbs} enthaltene Checkliste zur Auswahl eines \Gls{odbms} berücksichtigt.

\textbf{Funktionalität}\\
Das System stellt alle geforderten Funktionen mit den definierten Eigenschaften zur Verfügung.
\begin{description}
\item[Richtigkeit] Ergebnisse sind korrekt oder ausreichend genau. Die Ergebnisse sollen zu 99\% mit denen des Ist-Standes übereinstimmen.
\item[Interoperabilität] Es sind Schnittstellen zur Ein- und Ausgabe vorhanden. Dabei soll es sich um PostgreSQL Import sowie PostgreSQL und \Gls{umn} Export handeln.
\item[Funktionsumfang] Mindestens die benannte und essentielle Menge an Funktionalitäten wird bereitgestellt. Dazu zählt:
parallele Verarbeitung, Gruppierungs-, Filter-, Verschneidungs- sowie Overlayfunktionen, Geostatistik und Umrechnung zwischen Koordinatensystemen und -formaten. Außerdem sind vorhandene Datentypen und Schemaversionierung von Interesse.
\item[Ordnungsmäßigkeit] Die Implementation des Systems und dessen Funktionen erfüllt Normen, Vereinbarungen, gesetzliche Bestimmungen und andere Vorschriften. Hierzu ist zu nennen, dass besonders Berechnungsfunktionen nach mathematischen Gesetzen implementiert sein müssen. Konkret sind Berechnungen der räumlichen Verarbeitung nach anerkannten definierten Algorithmen durchzuführen.
\end{description}


\textbf{Zuverlässigkeit}\\
\begin{quote}
Fähigkeit einer Software, ihr Leistungsniveau unter festgelegten Bedingungen über einen festgelegten Zeitraum bewahren.\footnote{\cite{book:lehrbuchsoftware} S. 259}
\end{quote}
Nutzung von Tools zur Überwachung und Konfiguration immanent.
\begin{description}
\item[Fehlertoleranz] Das System sollte auftretende Fehler des Tagesgeschäftes abfangen und weiterarbeiten. Besonders Fehler in den Quelldaten können zu Fehlern während der Ausführung von Berechnungen führen, was per s\'{e} abgefangen werden muss.
\item[Wiederherstellbarkeit] Auch die Möglichkeit bei einem schwerwiegendem Fehler Daten und Stände der abgebrochenen Operationen wiederherzustellen ist ein zu betrachtendes Qualitätskriterium.
\item[\Gls{mttf}] Diese statische Kenngröße der erfahrungsgemäßen mittleren Lebensdauer ist für kritische Systeme relevant.
\end{description}


\textbf{Benutzbarkeit}\\
Qualität des Zugangs für Benutzer sowie Eignung für eine oder mehrere Benutzergruppen.
\begin{description}
\item[Verständlichkeit] 
\item[Bedienbarkeit] 
\item[Dokumentation] Eine ausführliche, aktuelle und korrekte Dokumentation ist Voraussetzung zur produktiven Verwendung.
\item[Eignung] Die angestrebte Benutzergruppe muss mit der aktuellen Benutzergruppe übereinstimmen. Die aktuelle Benutzergruppe ist Programmierer bzw. Administrator.
\end{description}


\textbf{Effizienz}\\
Das Verhältniss zwischen Auslastung der Hardware und erfolgreich bearbeiteten Aufgaben. Nach \cite{book:Leistungsanalyse} S. 21 ist Leistung paralleler Programme das Verhältnis des Speedups zur Anzahl der verwendeten Prozessoren. Wobei Speedup als Verhältnis der Ausführungszeiten zwischen der auf N Prozessoren ausgeführten parallelen Version eines Programms und der sequentiellen Version des Programmes definiert ist. Diese Definitionen treffen für die zu untersuchenden Systeme zu, da es sich um parallelisierende \Gls{gis} handelt.
\begin{description}
\item[Zeitverhalten] Oder auch Laufzeitverhalten genannt, dient allgemein zur Darstellung des Durchsatzes. Die Skalierung des Systems zählt hier dazu. Dies wird speziell durch zusätzliche Leistungstests beurteilt.
\item[Verbrauchsverhalten] Das Verhältnis aus erbrachter Leistung und dem dafür notwendig gewesenen Aufwand in Form von Hardwarenutzung.
\end{description}


\textbf{Änderbarkeit}\\
Aufwand zur Verbesserung oder Anpassung der Umgebung und der Spezifikationen, auch Wartungsaufwand genannt.
\begin{description}
\item[Analysierbarkeit] \glqq Aufwand, um Mängel oder Ursachen von Versagen zu diagnostizieren oder um änderungsbedüftige Teile zu bestimmen.\grqq\footnote{\cite{book:lehrbuchsoftware} S. 260}
\item[Modifizierbarkeit] Notwendiger Aufwand für Änderungen zum Ziele der Verbesserung und Fehlerbehebung.
\item[Stabilität] Wahrscheinlichkeit vom ungewollten Auswirkungen von Änderungen.
\item[Prüfbarkeit] Oder Testbarkeit als Merkmal, welches die Möglichkeiten und den Aufwand zum testen der originalen und geänderten Systeme.
\end{description}


\textbf{Übertragbarkeit}\\
Die Fähigkeit das System auf andere Hard- und Software und andere Vorgehensweisen zu migrieren.
\begin{description}
\item[Anpassbarkeit] Möglichkeiten des unveränderten Systems Änderungen vorzunehmen.
\item[Installierbarkeit] Systemvoraussetzung und Aufwand zur Installation des Systems.
\end{description}

\textbf{nichttechnische Kriterien}\\
Erweiterte Qualitätskriterien, welche nicht nach der DIN 9126 zugeordnet werden können.
\begin{description}
\item[Herstellerfirma und Produkt] Dazu zählt die Marktposition, der Preis, die Produktplanung und Service.
\item[Installierbarkeit] Systemvoraussetzung und Aufwand zur Installation des Systems.
\end{description}

Die zu untersuchenden Qualitätskriterien sind Funktionsumfang, Fehlertoleranz, Dokumentation, Zeitverhalten, Analysier- und Modifizierbarkeit.


\subsection{Qualitätsmetriken}


\subsection{Testfälle}

%Qualitätskriterien auswählen und Metriken definieren - Usecases bzw, Funktionstests bzw. Testfälle definieren
%

\section{Ist-Stand}
%groben Ablauf textuell und grafisch darstellen
