\chapter{Grundlagen}
\Gls{computer}

\section{Datenbank}

\subsection{ACID}
\Gls{acid}

\subsection{MVCC}
\Gls{mvcc}

\subsection{BASE}
\Gls{base}

\subsection{weitere Begriffsdefinitionen}

\subsection{Indexstrukturen}

\subsubsection{R-Baum}

\subsubsection{B-Baum}

\subsubsection{LSM-Baum}

\subsubsection{Geohash}

\subsection{Mehrrechner-Datenbanksystem}

\subsection{Verteiltes Datenbanksystem}

\subsection{Replikationsverfahren}

\subsubsection{Synchron}

\subsubsection{Asynchron}

\subsubsection{Kaskadiert}

\newpage

\section{geografische Datenverarbeitung}

\subsection{Bezugssysteme}

\subsection{Datenformate}

\subsubsection{Punkte}

\subsubsection{Vektoren}

\subsubsection{Raster}

\subsubsection{Shapefile}

\subsection{GIS}

\subsection{PostGIS}

\subsection{GeoTools}



\section{NoSQL}

\subsection{Definition}

\subsection{Kategorisierung}

\subsection{Hadoop}
% http://blog.samibadawi.com/2012/03/hive-pig-scalding-scoobi-scrunch-and.html

\subsection{Accumulo}
\url{https://en.wikipedia.org/wiki/Apache_Accumulo}

\subsection{NoSQL GIS}

\subsection{MongoDB}

\subsection{CouchDB}

\subsection{Neo4J}

\newpage

\subsection{Rasdaman}

\url{http://live.osgeo.org/de/overview/rasdaman_overview.html} :\\
- Array-Datenbanksystem
- PostgreSQL Aufsatz
- Multi-Dimensionalität
- eigene Anfragesprache
- skalierend
- unterstützt WCS Core und WCPS
- Implementierte Standards: OGC WMS 1.3, WCS 2.0, WCS-T 1.4, WCPS 1.0, WPS 1.0
- Lizenz: Clients und APIs: GNU Lesser General Public License (LGPL) version 3; Server-Engine: GNU General Public License (GPL) version 3
- Unterstützte Plattformen: Linux, MacOS, Solaris
- APIs: rasql, C++, Java


\url{http://www.rasdaman.org/} :\\
- open-source
- "extends standard relational database systems with the ability to store and retrieve multi-dimensional raster data"


\url{http://www.rasdaman.de/} :\\
- "erlaubt die Ablage von unbeschränkt grossen multi-dimensionalen Arrays ("Rasterdaten") in einer konventionellen Datenbank"


\subsection{Spacebase}

\url{http://docs.paralleluniverse.co/spacebase/} :\\
- serverseitig
- in-memory
- spatial data store
- ausgelegt für viele rechner und hohen Durchsatz (real-time)
- 2D und 3D Objekte mit 3D bbox
- load balancing enthalten
- spatial querys möglich
- benötigt JVM
- API für Java, Ruby, Python, Node.js, C++, Erlang
- API stellt nur elementare spatial querys zur verfügung: intersect oder contains
- eigene spatial querys können definiert werden

\subsection{Geomesa}

- Ingest = Import über Kommandozeile (geomesa-tools)
- Ingest von shp, csv und tsv Dateien
- Anderer Dateiimport mit GeoTools
- Verarbeitung nur über externe Tools (Spark, geotools)
- Export: csv, tsv, shp, geojson, gml

\url{http://www.eclipse.org/community/eclipse_newsletter/2014/march/article3.php} :\\
- open-source
- build on Accumulo and Hadoop
- Supporting the GeoTools API
- GeoServer Plugin
- geohash for indexing


\url{https://www.locationtech.org/proposals/geomesa} :\\
- outperforming postgis with geoserver


\url{http://de.slideshare.net/CCRinc/location-techdc-talk2-28465214}
- Verwendung fraktaler Kurven
- mit Spark und Scalding wesentlich schneller als PostGIS


\url{https://docs.google.com/presentation/d/1NO0ppk8MfDs8Q-QcUidZCSZK7YYwd9RjJoHV1V4Yq_w/edit?pli=1#slide=id.p} :\\
- 

%storm vs spark: http://xinhstechblog.blogspot.de/2014/06/storm-vs-spark-streaming-side-by-side.html https://stackoverflow.com/questions/24119897/apache-spark-vs-apache-storm http://www.zdatainc.com/2014/09/apache-storm-apache-spark/

\section{Leistungstests}

- siehe BA
- in Absprache mit Prof. Riechert