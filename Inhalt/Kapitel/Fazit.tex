\chapter{Fazit}
%4-5 Seiten
In diesem letzten Kapitel wird das ausgewählte Framework Postgres-XL erneut einer Nutzwertanalyse unterzogen, da die vorhergehenden Kapitel neue Erkenntnisse hervorbrachten.
Im Anschluss wird diese Arbeit zusammengefasst und die Ergebnisse dieser mit Schwerpunkt der Nutzwertanalyse für Agri~Con gewertet.
Diese Arbeit und dieses Kapitel enden mit einem Abschnitt zum Ausblick, in welchem die zukünftige Nutzung der gewonnenen Erkenntnisse allgemein und bei Agri~Con erläutert wird.

\section{Nutzwertanalyse}
Aufbauend auf die Definition der Nutzwertanalyse in Abschnitt \ref{section:definitionnutzwertanalyse} wird diese erweitert, um die Ergebnisse aus den Tests in Kapitel \ref{chapter:tests} zu berücksichtigen.

%Wertungsmaßstab ergänzen und anpassen
\begin{table}[h!]
\centering
\begin{tabular}{l|l}
\textbf{Metrik} & \textbf{Gewichtung in \%} \\ \hline
Interoperabilität & 20 \\ \hline
Funktionsumfang & 20 \\ \hline
Dokumentation & 15 \\ \hline
Zeitverhalten & 40 \\ \hline
Modifizierbarkeit & 5
\end{tabular}
\caption{Neuer Wertungsmaßstab der einzelnen Metriken}
\label{table:Wertungsmassstab2}
\end{table}
Nach Erfassung der Testergebnisse der Leistung, wird in diesem Schritt das Zeitverhalten zusätzlich in einer erneuten Nutzwertanalyse bewertet.

Die Interoperabilität wird weiterhin mit zwölf Punkten bzw. 100\%{} Erfüllung bewertet, da die Ergebnisse der Funktionstests beweisen, dass beide Schnittstellen verwendet werden können.

Die Erkenntnisse, welche durch die Durchführung der Funktions- und Leistungstests entstanden sind, bedingen eine Änderung der Punkte für den Funktionsumfang.
Verschneidungsfunktionen sind wie in FT05 gezeigt vorhanden und einsetzbar, weshalb der Wert dafür auf vier erhöht wird.
Vermindert wird dagegen der Wert der Parallelität auf eins, da Funktionen des Lasttestes der Verarbeitung nicht gleichzeitig genutzt werden konnten.
Dies ergibt einen Wert von 53.
%Funktionsumfang muss schlechter bewertet werden, da wichtige Funktionen eines DBMS nicht vorhanden sind. Dies ist abe rnicht in der Nutzwertanalyse berücksichtigt.

Der Wert der Dokumentation wird um eins auf acht vermindert, da fehlender Funktionsumfang schlecht oder gar nicht dokumentiert ist.
Dies ist aber eine wesentliche Information für die Auswahl und Verwendung eines Frameworks.
Funktionsumfang in Dokumentation wird so mit 1 bewertet.

Hinsichtlich der Modifizierbarkeit gab es keine Erkenntnisse, somit ändert sich dessen Bewertung nicht.

Das Zeitverhalten wurde mit den Lasttests in Kapitel \ref{section:leistungstests} ermittelt.
Die Laufzeit der Aggregation mit Postgres-XL liegt unter der mit PostgreSQL.
Dagegen ist die Verarbeitungsleistung gleich.
Entsprechend der Bewertungsfunktion auf Seite \pageref{bf:zeitverhalten} ergibt das die Bewertung drei und eins.
Das Zeitverhalten wird mit zwei gewertet.

\begin{table}[h!]
\centering
\small
\begin{tabular}{l|p{1.8cm}|c|p{3.1cm}|p{1.8cm}}
\textbf{Metrik} & \textbf{erreichter Wert} & \textbf{Erfüllung in \%} & \textbf{Kommentar} & \textbf{gewichteter Teilnutzen} \\ \hline
Interoperabilität & 12 & 100 & Analog des Ist-Standes. & 20 \\ \hline
Funktionsumfang & 53 & 87 & Mindestabdeckung erfüllt, jedoch sind Geostatistik und Versionierung nicht vorhanden. Außerdem keine parallele Nutzung einer Funktion. & 17 \\ \hline
Dokumentation & 8 & 69 & Dokumentation zu PostGIS ist sehr gut, zu Postgres-XL grob mit Mängeln bei fehlenden Funktionalitäten. Mindestabdeckung ist erfüllt. & 10 \\ \hline
Zeitverhalten & 2 & 67 & Bei Nutzung aller Coordinator besseres Zeitverhalten als PostgreSQL, jedoch bestehen hohe Kosten in der Hardwareanschaffung. & 27 \\ \hline
Modifizierbarkeit & 5 & 100 & Vollständige Abdeckung vorhanden. Möglichkeiten sind in SQL gegeben. & 5 \\
\end{tabular}
\caption{Neue Nutzwertanalyse von Postgres-XL}
\label{table:nutzwertanalyse2-postgresxl}
\end{table}
Entsprechend Tabelle \ref{table:nutzwertanalyse2-postgresxl} ergibt sich ein Nutzwert von 79.

\section{Zusammenfassung}


\section{Wertung}
%Ist Stand sollte nicht ersetzt werden
%Postgres-XL ist nicht ausgereift - für standardfälle geeignet
%Für spezielle Teilausfgaben könnte ein Framework eingeschätz und eigesetzt werden
%Unter realen Bedingungen (wie echte Netzwerk) höhere Kosten für Anfragen
%Docu schema für cluster geeignet

\section{Ausblick}

- auch Bezug auf Verarbeitung von ganzen Länderdaten mit dem System(en)
- Darstellung als wichtige Komponente: Möglichkeiten und Performanz
- verteilung der daten und query planning optimierung erneut nennen
- distribute by wichtig
- wenn trigger und subtransactions eingebaut sind und die Zuverlässigkeit erhöht wurde, kann der produktive EInsatz erneut validiert werden
- skalierung kaum vorhanden, neuere versionen validieren
- Kosten/Aufwand/Nutzen für Erstellung eines eigenen Clusters mit PostgreSQL Mitteln sind zu berücksichtigen
- 



\label{LastPage}