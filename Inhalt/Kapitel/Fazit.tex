\chapter{Fazit}
%4-5 Seiten
In diesem letzten Kapitel wird das ausgewählte Framework Postgres-XL erneut einer Nutzwertanalyse unterzogen, da die vorhergehenden Kapitel neue Erkenntnisse hervorbrachten.
Im Anschluss wird diese Arbeit zusammengefasst und die Ergebnisse dieser für Agri~Con gewertet.
Diese Arbeit und dieses Kapitel enden mit einem Abschnitt zum Ausblick, in welchem die zukünftige Nutzung der gewonnenen Erkenntnisse allgemein und bei Agri~Con erläutert wird.

\section{Nutzwertanalyse}%TODO titel erweitern
Aufbauend auf die Definition der Nutzwertanalyse in Abschnitt \ref{section:definitionnutzwertanalyse} wird diese erweitert, um die Ergebnisse aus den Tests in Kapitel \ref{chapter:tests} zu berücksichtigen.

%Wertungsmaßstab ergänzen und anpassen
\begin{table}[h!]
\centering
\begin{tabular}{l|l}
\textbf{Metrik} & \textbf{Gewichtung in \%} \\ \hline
Interoperabilität & 20 \\ \hline
Funktionsumfang & 20 \\ \hline
Dokumentation & 15 \\ \hline
Zeitverhalten & 40 \\ \hline
Modifizierbarkeit & 5
\end{tabular}
\caption{Neuer Wertungsmaßstab der einzelnen Metriken}
\label{table:Wertungsmassstab2}
\end{table}
Nach Erfassung der Testergebnisse der Leistung, wird in diesem Schritt das Zeitverhalten zusätzlich in einer erneuten Nutzwertanalyse bewertet.
Die Interoperabilität wird weiterhin mit zwölf Punkten bzw. 100\%{} Erfüllung bewertet, da die Ergebnisse der Funktionstests beweisen, dass beide Schnittstellen verwendet werden können.

Die Erkenntnisse, welche durch die Durchführung der Funktions- und Leistungstests entstanden sind, bedingen eine Änderung der Punkte für den Funktionsumfang.
Verschneidungsfunktionen sind wie in FT05 gezeigt vorhanden und einsetzbar, weshalb der Wert dafür auf vier erhöht wird.
Vermindert wird dagegen der Wert der Parallelität auf eins, da Funktionen des Lasttestes der Verarbeitung nicht gleichzeitig genutzt werden konnten.
Dies ergibt einen Wert von 53.
%Funktionsumfang muss schlechter bewertet werden, da wichtige Funktionen eines DBMS nicht vorhanden sind. Dies ist abe rnicht in der Nutzwertanalyse berücksichtigt.

Der Wert der Dokumentation wird um eins auf acht vermindert, da fehlender Funktionsumfang schlecht oder gar nicht dokumentiert ist.
Dies ist aber eine wesentliche Information für die Auswahl und Verwendung eines Frameworks.
Funktionsumfang in Dokumentation wird mit eins bewertet.

Das Zeitverhalten wurde mit den Lasttests in Kapitel \ref{section:leistungstests} ermittelt.
Die Laufzeit der Aggregation mit Postgres-XL liegt unter der mit PostgreSQL.
Dagegen ist die Verarbeitungsleistung gleich.
Entsprechend der Bewertungsfunktion auf Seite \pageref{bf:zeitverhalten} ergibt das die Bewertung drei und eins.
Das Zeitverhalten wird mit zwei gewertet.

Hinsichtlich der Modifizierbarkeit gab es keine neuen Erkenntnisse, somit ändert sich dessen Bewertung nicht.

\begin{table}[h!]
\centering
\small
\begin{tabular}{l|p{1.8cm}|c|p{3.1cm}|p{1.8cm}}
\textbf{Metrik} & \textbf{erreichter Wert} & \textbf{Erfüllung in \%} & \textbf{Kommentar} & \textbf{gewichteter Teilnutzen} \\ \hline
Interoperabilität & 12 & 100 & Analog des Ist-Standes. & 20 \\ \hline
Funktionsumfang & 53 & 87 & Mindestabdeckung erfüllt, jedoch sind Geostatistik und Versionierung nicht vorhanden. Außerdem keine parallele Nutzung einer Funktion. & 17 \\ \hline
Dokumentation & 8 & 69 & Dokumentation zu PostGIS ist sehr gut, zu Postgres-XL grob mit Mängeln bei fehlenden Funktionalitäten. Mindestabdeckung ist erfüllt. & 10 \\ \hline
Zeitverhalten & 2 & 67 & Bei Nutzung aller Coordinator besseres Zeitverhalten als PostgreSQL, jedoch bestehen hohe Kosten in der Hardwareanschaffung. & 27 \\ \hline
Modifizierbarkeit & 5 & 100 & Vollständige Abdeckung vorhanden. Möglichkeiten sind in SQL gegeben. & 5 \\
\end{tabular}
\caption{Neue Nutzwertanalyse von Postgres-XL}
\label{table:nutzwertanalyse2-postgresxl}
\end{table}
Entsprechend Tabelle \ref{table:nutzwertanalyse2-postgresxl} ergibt sich ein Nutzwert von 79.

\section{Zusammenfassung}
Die \titel{} am Beispiel des aktuellen Standes bei Agri~Con GmbH wird in diesem Abschnitt zusammengefasst.

Diese Arbeit begann mit den Grundlagen, welche für das Verständnis der darauf folgenden Ausführungen und das angewandte Vorgehen notwendig zu klären sind.
Dazu zählen Begriffe zu \Gls{dbms}, räumliche Datenverarbeitung und Alternativen zum relationalen Datenbankmodell.
Darin wurden im dritten Abschnitt Frameworks vorgestellt, welche in den anderen Kapiteln relevant sind.
Dazu zählen Postgres-XL, Rasdaman und GeoMesa.

Nach Sicherstellung der theoretischen Grundlagen folgte in Kapitel \ref{chapter:methodik} die Darlegung und Begründung der Methodik dieser Arbeit.
Diese Darlegung fand anhand einer Unterteilung des Themas in vier Unteraufgaben statt.
Dabei waren die Vorgehen zur Softwareauswahl und Leistungsbestimmung Schwerpunkte.
Es wurde die Nutzwertanalyse und Funktions- sowie Leistungstests als geeignete Mittel heraus gearbeitet.

Kapitel \ref{chapter:ausgangsszenario} legte dar, worauf und wie diese Methodik angewandt wird.
Das heißt, dass der Anwendungsfall mit Anforderungen festgelegt wurde.
%Dabei war Softwarequalität das Mittel 
Die Anforderungen wurden wissenschaftlich in Form von Softwarequalität festgehalten und mit Qualitätsmetriken messbar gemacht.
Die Softwarequalität wurde umfassend beschrieben und für die Untersuchung relevante Kriterien heraus gearbeitet.
Außerdem wurden Funktions- und Lasttests für den Anwendungsfall skizziert.
Das Kapitel endet mit einer Übersicht über relevante Literatur zum Thema dieser Arbeit.
Darin wird deutlich, dass keine thematisch vergleichbare Arbeit existiert und Teilprobleme in anderen Arbeiten zu finden sind.

Die erste Hälfte der Aufgabenstellung wird im Kapitel \ref{chapter:systemauswahl} gelöst.
Nach Definition der Nutzwertanalyse wurden Frameworks an Hand ihrer Spezifikation und der Nutzwertanalyse bewertet und Postgres-XL mit einem Nutzwert von 86 als geeignet ausgewählt.
Die Frameworks GeoMesa und Rasdaman erhielten dabei die Wertung 56 bzw. 51.

Postgres-XL erfuhr im darauf folgenden Kapitel eine Untersuchung hinsichtlich der allgemeinen Verwendung und der Möglichkeiten für den Einsatz bei Agri~Con.
So wurde das Vorgehen der Installation, die Nutzung der Schnittstellen und die Möglichkeiten der Verarbeitung erörtert, wobei die Schnittstellen und die Verwendung analog zu den Schnittstellen und der Verwendung von PostgreSQL sind.
Der Einsatz bei Agri~Con wurde heraus gearbeitet und eine tief greifende Integration in den Ist-Stand als notwendig ermittelt.
Doch auf Grund von fehlender Funktionalität ist die Integration von Postgres-XL im Rahmen dieser Arbeit nicht möglich.
Für das weitere Vorgehen wurde die Anpassung des Ist-Standes an Postgres-XL für die Untersuchung mit Funktions- und Leistungstests beschrieben.

Kapitel \ref{chapter:tests} enthält die Definition der Testumgebung sowie die Definition und die Ergebnisse der Funktions- und Leistungstests.
Als Testumgebung diente ein IBM Server, auf welchem mit virtualisierten Maschinen Postgres-XL und PostgreSQL installiert und miteinander vergleichbar konfiguriert wurden.
Die Funktionstests validierten die Funktionalität von Postgres-XL, wobei einzig der Test FT06 fehlschlug, da Speicher Befehle in SQL Funktionen Fehler verursachten und mit R berechneten Werte nicht validiert werden konnten.
Die Leistungsfähigkeit bezüglich der Aggregation und Verarbeitung von Daten wurde mit den Leistungstests ermittelt.
Diese Ermittlung fand mit Postgres-XL und PostgreSQL statt, um relative Aussagen treffen zu können.
In der Präambel fand eine Auseinandersetzung mit den Begriffen Leistung, Lastmessung, Auswertung, Datenverteilung und Verbesserungen in Postgres-XL, der Lastverteilung und der Skalierung statt.
Eine Zusammenfassung der Testergebnisse ergab geringere Laufzeiten von bis zu 16\%{} bezüglich der Aggregation mit Postgres-XL.
Die Verarbeitungsleistung unterscheidet sich um weniger als 1\%{}.
Diese Steigerung ging mit sechsfachen Hardwareaufwand einher.

\section{Wertung}
%Ist Stand sollte nicht ersetzt werden
%Postgres-XL ist nicht ausgereift - für standardfälle geeignet
%- verteilung der daten und query planning optimierung erneut nennen
%- distribute by wichtig
%- skalierung kaum vorhanden, neuere versionen validieren
%Unter realen Bedingungen (wie echte Netzwerk) höhere Kosten für Anfragen
%Für spezielle Teilausfgaben könnte ein Framework eingeschätz und eigesetzt werden
%Docu schema für cluster geeignet

Die Definition der Anforderungen kann für zukünftige Validierungen des Ist-Standes und zu analysierender Frameworks verwendet werden.
Lücken der Funktionalität in Postgres-XL bedingen Ergänzungen der Anforderungen bezüglich allgemeiner Funktionen von \Gls{dbms} zur umfassenden Bewertung eines Frameworks.
So ist Ordnungsmäßigkeit gesondert zu bewerten.
Die Berücksichtigung dieses Qualitätskriteriums hat eine Verminderung des Nutzwertes von Postgres-XL zur Folge.
Diese Berücksichtigung allgemeiner Qualitäten muss für eine umfassende und allgemeine Untersuchung eines solchen Frameworks verwendet werden.
Obwohl Postgres-XL nicht empfohlen werden kann, macht der hohe Nutzwert folgendes deutlich: 
Die Nutzwertanalyse ist eine Abstraktion der Bewertung von Softwarequalität.
Dabei wird eine festgelegte Menge von Qualitäten untersucht, der Umfang der Untersuchung ist somit eingegrenzt.
Außerdem führen fehlende essentielle Qualitätskriterien nicht zu einer null Wertung des untersuchten Systems.
Der hohe Wert spricht nichts desto trotz für eine stetige Berücksichtigung von Postgres-XL in solchen Untersuchungen, da es sich wegen des Umfanges und den Möglichkeiten um ein vielversprechendes Framework handelt.

Die in der Literaturrecherche aufgedeckte Lücke an wissenschaftlichen Dokumenten zu diesem Thema ist negativ zu bewerten.
Diese Arbeit schließt diese Lücke nicht, da die Untersuchung für ein Anwendungsszenario und nicht allgemein stattfand, da als Grundlage zur Softwareauswahl eine unbestätigte Liste an relevanten Frameworks verwendet wurde und da die Leistungstests die Skalierbarkeit nicht ausreichend bestimmten, um  die Skalierbarkeit für andere Knotenzahlen zu schätzen.

Die Übersicht und Bewertung der wichtigsten verteilten \Gls{gis} ist trotz fehlender Validierung von allgemeinem Interesse und zukünftig zu aktualisieren.
%Wer validiert eine solche liste?

Die Softwareauswahl und -bewertung erfolgte nachvollziehbar, weshalb diese Arbeit als Handlungsempfehlung für ähnliche Anwendungsszenarien verwendet werden kann.
Wird die Nutzwertanalyse um Ordnungsmäßigkeit erweitert, kann bei der Bewertung die Durchführung der Funktionstests entfallen.

Die durch eine genauere Untersuchung aufgedeckten Lücken in der Funktionalität von Postgres-XL sprechen gegen einen produktiven Einsatz des Frameworks im zu Grunde gelegten Anwendungsfall.
Es ist in der verwendeten Version 9.2.34 nicht ausgereift.
Darin werden keine Trigger und Sub-Transaktionen unterstützt, der Transaktionstyp internal ist in Prozeduren nicht verwendbar, Sequenzen unterscheiden sich auch bei Verteilung einer Tabelle per Replikation und die Verwendung von R Funktionen ist problematisch.
Lagert man die virtuellen Maschinen auf eigenständige physische Maschinen aus, muss mit zu berücksichtigenden Kosten für den Netzwerkverkehr und höherer Lese - und Schreibgeschwindigkeit des Festspeichers pro Knoten gerechnet werden.
In beiden Fällen ist der Leistungsgewinn durch Einsatz eines Clusters zu gering, als das sich ein sechsfacher Hardwareaufwand lohnt.
Ein Nutzen ergibt sich mit Postgres-XL, wenn Daten in wenigen Relationen verteilt gespeichert und gelesen werden müssen.
Besonders bei Datenmengen, welche die Größe von Festplatten übersteigen.
Bei Agri~Con wäre der Bereich Docu dafür geeignet.
So könnten alle Positionsdaten der Maschinen aller Betriebe unveränderlich abgelegt und mit kurzen Laufzeiten aggregiert werden.

Für Agri~Con ist das Ergebnis, dass der Ist-Stand mit der eingesetzten Technologie für diese umfangreichen Anforderungen am besten geeignet ist, sofern kostenlose Frameworks berücksichtigt werden.
Änderungen am Datenbankschema, an den Kostenwerten der Funktionen, an der Konfiguration des Query Planers und den Datenbank nahen Anwendungen erhöhen die Leistungsfähigkeit der PostgreSQL Installation um die gewünschte Verminderungen im Laufzeitverhalten zu erwirken.
Der Einsatz von anderen Frameworks ist dagegen für Teilaufgaben sinnvoll.
Beispielsweise würde sich Rasdaman bei wesentlicher Nutzung von Rasterdaten, welche unabhängig vom Datenbankschema sind, zur Speicherung, Verarbeitung und Bereitstellung eignen.


\section{Ausblick}
%- wenn trigger und subtransactions eingebaut sind und die Zuverlässigkeit erhöht wurde, kann der produktive EInsatz erneut validiert werden
%- auch Bezug auf Verarbeitung von ganzen Länderdaten mit dem System(en)
%- Darstellung als wichtige Komponente: Möglichkeiten und Performanz
%- Kosten/Aufwand/Nutzen für Erstellung eines eigenen Clusters mit PostgreSQL Mitteln sind zu berücksichtigen
%- zu geeigneten zeitpunkt erneute analyse relevanter frameworks
%- ist stand sollte optimiert werden

Agri~Con wird Änderung an der PostgreSQL Installation vornehmen, um die gewünschten Effekte zu erzielen.
Die in dieser Arbeit durchgeführte Untersuchung wird mittelfristig nicht erneut durchgeführt.
Bei Änderung des Szenarios werden Untersuchungen spezieller Frameworks für diese Änderungen erhoben.

Werden sich die Anforderungen bezüglich des Zeitverhaltens erhöhen, ist momentan die Erstellung eines Clusters mit PostgreSQL Instanzen das gewünschte Vorgehen.
Dabei wird das Datenbankschema aufgeteilt und in zwei PostgreSQL Instanzen integriert.
Es wird von einer annähernden Verdopplung der zusammengefassten Leistungsfähigkeit ausgegangen.
Neben der Veränderung des Schemas sind umfangreiche Änderungen in Programmen der Agri~Con durchzuführen.

%Gegenüberstellung verteilter GIS bleibt Forschungsziel - nicht nur für agricon interessant
%Neuentwicklungen bedingen auseinadnersetzung
%erst langfristig kokurrierende Produkte
Weiter- und Neuentwicklungen bedingen eine ständige Aktualisierung der Übersicht verteilter GIS.
Zwar ist langfristig mit einem zu PostgreSQL mit PostGIS konkurrierenden Framework zu rechnen, aber das Forschungsziel der Bewertung solcher Frameworks ist von stetigem Interesse.

\label{LastPage}